\chapter{Chapter 4: State Space Operators and the Quantum Observables Problem}
The previous chapter introduced the concept of Hilbert space vectors, called state vectors, representing physical states. The state space, like any infinite-dimensional vector space, has infinite bases, and this was claimed to be a motivating factor for the Hilbert space formulation in the first place: because it allows for the state represented by the state vector to be looked at from the perspective of different observables like momentum and position. To complete our understanding of how the state problem is approached in quantum mechanics we need to understand exactly how observables are incorporated into this linear algebraic vector space system, through very specific bases. This question will be answered by an investigation into \textbf{linear operators}. 
\section{Hilbert Space Operators}
An operator is a function, but it is the lexicon used to describe a sort of function which acts on functions or vectors as inputs-- here, in the state space, on state vectors. For a new student of quantum mechanics it is paramount to understand that the distinction between an operator and a function does not exist. The only reason the term operator is used in conjunction with vector spaces is to avoid exhausting the term function, since operators act on state vectors which are themselves linked already to wavefunctions. If unused to the term, there is a harmful tendency to give a mystic image to operators, but this must not be done. An operator, just like a function, takes an input and maps it to an output, and that is the extent of it. 
\\\\
For example, one operator the reader will have seen already is the differential operator:
$$
\frac{d}{dx}f(x)=f'(x).
$$
This takes some input function $f(x)$, and maps it to an output function $f'(x)$. In that sense, it is a function because it has an input and output, but it is an operator because it acts on functions. If the input was $f(x)=x^2$ then the output of the differential operator would be $2x$; if it were $e^x$ then its output would also be $e^x$. A scalar can also be an input of the operator: since an operator is a function of functions, we could just define the input function to be $f(x)\equiv k$ for that scalar $k$, and of course this would still be a function whilst sharing no differences to the scalar. Here, if such a scalar was an input of the differential operator, than the output would be the null function: or, $0$, but for other operators, this may be different.
\\\\
If this is understood, the technical requirement for an operator which is linear, here donated by $\Omega$, is that they are linear maps:
$$
\Omega: \mathscr{H} \mapsto \mathscr{H},\:\:\:\: \Omega(\Psi_{1}+\Psi_{2})\equiv \Omega\Psi_{1}+\Omega\Psi_{2}.
$$
There do exist non-linear operators, but in this book, and much of quantum mechanics, we will only ever encounter linear operators, and so we can take the latter linear distributive property to be a given for our work. We also have:
\begin{itemize}
    \item Associativity of scalar multiplication:
    $$
    (c\Omega)\Psi=c(\Omega\Psi).
    $$
    \item Distributivity:
    $$
    (\Omega_{1}+\Omega_{2})\Psi=\Omega_{1}\Psi + \Omega_{2}\Psi.
    $$
    \item Associativity in operators:
    $$
    \Omega_{1}(\Omega_{2}\Psi)=\Omega_{1}\Omega_{2}\Psi
    $$
\end{itemize}
We do not, however have commutativity:
$$
\Omega_{1}\Omega_{2}\Psi\neq\Omega_{2}\Omega_{1}\Psi
$$
for most pairs of operators $\Omega_{1}$ and $\Omega_{2}$ (some do by chance). Moreover, there is in general no useful relation between $\Omega_{1}\Omega_{2}$ and $\Omega_{2}\Omega_{1}$, unlike the inner product, which is not commutative, but which follows the much nicer relation 
$$
(\Psi_{1},\Psi_{2}) = (\Psi_{2},\Psi_{1})^{\ast}.
$$
The general irregularity of commutation relations between linear operators will turn out to be connected to an incredibly rich set of quantum phenomena!
\\\\
The rest of the operator facts again should, like the rules of vector spaces, come mostly naturally. A useful point to remember is that for any operator we work with in quantum mechanics, it maps from the state space $\mathscr{H}$ to the state space $\mathscr{H}$: in other words, performing that operator on any input state space vector will directly create a new state space vector! Thus 
$$
\Omega\Psi
$$
is a new state space vector, and so is
$$
\Omega_{1}\Omega_{2}\Psi=\Omega_{1}(\Omega_{2}\Psi)
$$
which is, the action of $\Omega_{1}$ on the state space vector $\Omega_{2}\Psi$ obtained after operating with $\Omega_{2}$ on an original state vector $\Psi$. We have a tendency, which is natural, to be suspicious when some arbitrary function acts on an immensely complex input vector. However, in the rules of quantum mechanics where linear operators dominate, we do not need this suspicion. Fluency should soon dictate that we see $\Omega\Psi$ as just another state space vector. 
\\\\
The caveat which must be noted is that general linear operators will not send a unit ray to another unit ray: i.e, if the input vector is normalised the transformation will usually give them a new unphysical phase factor. Yet we know the Hilbert space can contain these unphysically scaled state vectors, which are equivalent to the unit ray states, so we are fine, as we can renormalise them later. 
\\\\
The special class of operators which \textit{do} preserve the normalisation of unit ray states are called \textbf{unitary} operators. We will cover these later.
\\\\
Now, the much more important qualities of operators in quantum mechanics are that they possess quantities called eigenvalues and eigenvectors, which is where the quantum mechanical formalism will really start to come together.
\subsection{Eigenvalues and Eigenvectors}
We have just studied operators, which are fundamentally crucial to all quantum mechanics, but of the moment seem to have vague physical interpretations. The other side of the same coin is \textbf{eigenvalues}. Flip to the middle chapter of any quantum mechanics text and you might find a large mixture of words with the prefix ``eigen-": eigenvector, eigenfunction, eigenvalue, eigenenergy, eigenstate, eigenmomentum- et cetera. This will soon become natural, though it may be initially daunting. Understanding why these words are so commonplace will set us in a good position, so we do that now.
\\\\
Consider a three dimensional example first. We will pick a random vector, which in this case can be a column vector, 
$$
V=\begin{pmatrix}
\frac{47}{10}\\
5.69\\
242\\
\end{pmatrix}
$$
and a random operator, 
$$
\Omega \begin{pmatrix}
x_{1}\\
x_{2}\\
x_{3}\\
\end{pmatrix}=\frac{1}{2}\begin{pmatrix}
x_{2}\\
x_{3}\\
x_{1}\\
\end{pmatrix}
$$
Upon applying this operator to the vector $V$, we are likely not going to get an output vector which is anything like the original vector in terms of how corresponding components are related.
\\\\
We are however given the grace from mathematics of sets of vectors, belonging to specific operators, which behave much more stably. To each operator $\Omega$ there exists a set of vectors, called \textbf{eigenvectors}, such that:
$$
\Omega\omega = \lambda\omega
$$
for some constant $\lambda$ and vector $\omega$. In the above equation, which we call an eigenvalue equation, $\omega$ is an eigenvector of $\Omega$, and $\lambda$ is the corresponding eigenvalue: a complex constant. This equation corresponds to the operator $\Omega$ scaling the eigenvector $\omega$ by a scale factor $\lambda$: a relatively very trivial transformation compared to the nontrivial possibilities we expect for general vectors under the transformation. In the above example we just saw, for example, no such scaling exists at all, despite the transformation looking somewhat simple!
\\\\
Let us consider a basic operator and try to solve for all possible eigenvectors and eigenvalues. We could use the identity operator, which is extremely trivial:
$$
I\epsilon = \epsilon
$$ means that any vector $\epsilon$ is an eigenvector of the identity operator, with corresponding eigenvalue $1$. It is a rather uninteresting result, which pertains to a uniquely simple vector. 
\\\\
This eigenvalue equation was uniquely simple, but for more complicated operators, it is clear that trying to find eigenvectors by inspection will usually be futile. The more advanced methods of solving eigenvalue problems will come later in this book; for now, only the theory is important.
\\\\
A final note on the above: the prefix eigen-, which is derived from German and means `own' (hence, each operator has its `own' set of eigenvectors), is always used in mathematics when we are dealing with the above cases. We therefore have eigenvalues, eigenvectors, eigenfunctions: but also, later on, eigenenergies, eigenmomenta, and so on, when the eigenvalues are energy and momentum values respectively. The context and this explicit note should demystify such eigen- words in the future.
\subsection{Hermitian Operators}
The next important definition central to quantum mechanics is of \textbf{Hermitian} operators. Operators are Hermitian if they possess the property:
$$
\oip{\bm{\alpha}}{\Omega\bm{\beta}}=\oip{\Omega\bm{\alpha}}{\bm{\beta}}.
$$
for \textit{any} vectors $\bm{\alpha},\bm{\beta}$. This `Hermiticity' property is deceptively simple, but here are some profound consequent facts for hermitian operators following this definition.
\begin{enumerate}
    \item[H1.] Hermitian operators must have real eigenvalues. \\\\
    \underline{Proof:}\\\\
    For a hermitian operator the eigenvalue condition is the same
    $$
    \Omega \omega= \lambda \omega.
    $$
    We have, in taking the inner product with the eigenvector $\omega$:
    $$
    \oip{\omega}{\Omega\omega}=\oip{\Omega\omega}{\omega} 
    $$
    by the definition of Hermiticity, so
    $$
    \begin{aligned}
    \oip{\omega}{\lambda\omega}=\oip{\lambda\omega}{\omega} &\Rightarrow\:\: \lambda\oip{\omega}{\omega} = \lambda^{\ast}\oip{\omega}{\omega}\\
    \Rightarrow\:\: \lambda = \lambda^{\ast} &\Rightarrow \:\: \lambda \in \mathbb{R} \btab \square
    \end{aligned}
    $$
    \item[H2.] Eigenvectors $\omega_{1}$ and $\omega_{2}$ of the same hermitian operator corresponding to different eigenvalues are orthogonal to each other.\\\\
    \underline{Proof:}\\\\
    To prove that $\omega_{1}$ and $\omega_{2}$ are orthogonal we need to prove that $\oip{\omega_{1}}{\omega_{2}}=0$. We can do this by manipulating the hermitian property of the operator, here denoted $\Omega$.
    $$
    \begin{aligned}
    \oip{\omega_{1}}{\Omega\omega_{2}}&=\oip{\Omega\omega_{1}}{\omega_{2}}\\
    \Rightarrow\:\: \oip{\omega_{1}}{\lambda_{2}\omega_{2}}&=\oip{\lambda_{1}\omega_{1}}{\omega_{2}}\\
    \Rightarrow\:\: \lambda_{2}\oip{\omega_{1}}{\omega_{2}}&=\lambda_{1}^{\ast}\oip{\omega_{1}}{\omega_{2}}\\
    \end{aligned}
    $$
    but $\lambda_{1}$ is an eigenvalue of a hermitian operator so it is real: i.e, $\lambda_{1}^{\ast}=\lambda_{1}$. So above we had 
    $$
    \begin{aligned}
    \lambda_{2}\oip{\omega_{1}}{\omega_{2}}&=\lambda_{1}^{\ast}\oip{\omega_{1}}{\omega_{2}}\\
    \end{aligned}
    $$
    which is,
    $$
    \lambda_{2}\oip{\omega_{1}}{\omega_{2}}=\lambda_{1}\oip{\omega_{1}}{\omega_{2}}.\\
    $$
    Therefore, if the eigenvectors do not have the same eigenvalue then $\lambda_{2}\neq\lambda_{1}$ so the above implies that $\oip{\omega_{1}}{\omega_{2}}=0$ and so these eigenvectors must be orthogonal. $\btab\btab\square$
    \\\\
    There is a note for the above proof, however. The proof works on the assumption that for different eigenvectors their eigenvalues are also different. This is not always a correct assumption. Consider the identity operator $I$. It is clearly hermitian:
    $$
    \oip{\Psi_{1}}{I\Psi_{2}}=\oip{\Psi_{1}}{\Psi_{2}}=\oip{I\Psi_{1}}{\Psi_{2}}.
    $$
    However, it has infinite different eigenvectors but they all have the same eigenvalue: 1. Thus the proof above cannot apply to the identity operator. In this case, the collapse of the point should be obvious anyway-- all vectors are eigenvectors of the identity operator, but certainly not all vectors are orthogonal to each other. In general, an operator where different eigenvectors can share the same eigenvalue is said to be \textbf{degenerate}. In particular, we say particular eigenvalues are degenerate if they can correspond to multiple different eigenvectors-- but we do not say eigenvectors are degenerate because an eigenvector can never correspond to multiple different eigenvalues. 
    \\\\
    There are a few proofs of theorems in this book which involve the assumption that we are working with non-degenerate operators. These proofs, when we incorporate degeneracy, are usually different -- and unfortunately, more difficult. For every step where we assume non-degeneracy, it would still be within the reaches of this book to prove an alternative proof in the case of degeneracy, but at the same time these would take labour and space. Therefore, I will not include them in this book because they will not alter anything in the fundamental understanding of a reader. Should the reader want to find such proofs, they may turn to a more advanced textbook which has the space and desire to cover this technicality. It should not make a massive difference either way whether the reader is aware of the degenerate case proof, so long as they understand when degeneracy makes a difference to actual consequent theorem or result (and indeed when it does not, which is quite common here). I will highlight these cases at when they occur.
    \item[H3.] For an operator $\Omega$ with real eigenvalues $\lambda_{i}$ and eigenvectors $\bm{\alpha}_{i}$, if the eigenvectors constitute an orthonormal basis in the Hilbert space then the operator is hermitian.\\\\
    \underline{Proof:}\\\\
    Take the component expressions for the two arbitrary vectors $\Psi_{1}$ and $\Psi_{2}$. We know they can be both expressed as a linear combination of the eigenvectors $\bm{\alpha}_{i}$ since the set $\{\bm{\alpha}_{i}\}$ is stated in the conditions to be an orthonormal basis. So we have:
    $$
    \Psi_{1}=\sum_{\{i\}}c_{i}\bm{\alpha}_{i},\:\:\Psi_{2}=\sum_{\{j\}}\gamma_{j}\bm{\alpha}_{j}
    $$
    where the components $c_{i}$ and $\gamma_{j}$ are found by $(\bm{\alpha}_{i},\Psi_{1})$ and $\oip{\bm{\alpha}_{j}}{\Psi_{2}}$ respectively. Then 
    $$
    \begin{aligned}
    (\Omega\Psi_{1},\Psi_{2})&=\left(\Omega\sum_{\{i\}}c_{i}\bm{\alpha}_{i},\sum_{\{j\}}\gamma_{j}\bm{\alpha}_{j}\right)\\
    &=\left(\sum_{\{i\}}c_{i}\Omega\bm{\alpha}_{i},\sum_{\{j\}}\gamma_{j}\bm{\alpha}_{j}\right)
    \end{aligned}
    $$
    where the operator can be incorporated into the sum term as it is a linear operator. Then this becomes
    $$
    \begin{aligned}
    (\Omega\Psi_{1},\Psi_{2}) &= \left(\sum_{\{i\}}c_{i}\lambda_{i}\bm{\alpha}_{i},\sum_{\{j\}}\gamma_{j}\bm{\alpha}_{j}\right)\\
    &=\sum_{i,j}c^{\ast}_{i}\lambda_{i}^{\ast}\gamma_{j}(\bm{\alpha}_{i},\bm{\alpha}_{j})
    \end{aligned}
    $$
    but $\{\bm{\alpha}_{i}\}$ is an orthonormal basis so $(\bm{\alpha}_{i},\bm{\alpha}_{j})=\delta_{ij}$. The above then becomes $0$ except for when the two basis vectors are the same, so we are left with:
    $$
    (\Omega\Psi_{1},\Psi_{2})=\sum_{i}c^{\ast}_{i}\lambda^{\ast}_{i}\gamma_{i}.
    $$
    Now, considering $(\Psi_{1},\Omega\Psi_{2})$, we have very similarly:
    $$
    \begin{aligned}
    (\Psi_{1},\Omega\Psi_{2})&=\left(\sum_{\{i\}}c_{i}\bm{\alpha}_{i},\Omega\sum_{\{j\}}\gamma_{j}\bm{\alpha}_{j}\right)\\
    &=\left(\sum_{\{i\}}c_{i}\bm{\alpha}_{i},\sum_{\{j\}}\gamma_{j}\Omega\bm{\alpha}_{j}\right)\\
    &= \left(\sum_{\{i\}}c_{i}\bm{\alpha}_{i},\sum_{\{j\}}\gamma_{j}\lambda_{j}\bm{\alpha}_{j}\right)\\
    &=\sum_{i,j}c_{i}^{\ast}\lambda_{j}\gamma_{j}(\bm{\alpha}_{i},\bm{\alpha}_{j})=\sum_{i,j}c_{i}^{\ast}\lambda_{j}\gamma_{j}\delta_{ij}\\
    &=\sum_{i}c_{i}^{\ast}\lambda_{i}\gamma_{i}.
    \end{aligned}
    $$
    so we have 
    $$
    (\Omega\Psi_{1},\Psi_{2})=\sum_{i}c^{\ast}_{i}\lambda^{\ast}_{i}\gamma_{i},\:\:\:\:(\Psi_{1},\Omega\Psi_{2})=\sum_{i}c_{i}^{\ast}\lambda_{i}\gamma_{i}
    $$
    but we have conditioned that the eigenvalues $\lambda_{i}$ are real so we therefore see that $\lambda_{i}=\lambda_{i}^{\ast}$ and so
    $$
    (\Psi_{1},\Omega\Psi_{2})=(\Omega\Psi_{1},\Psi_{2})
    $$
    which is the definition of a Hermitian operator. This holds true for any arbitrary $\Psi_{1}$ and $\Psi_{2}$ so long as they are in the space spanned by the orthonormal basis and can subsequently be expressed as a linear combination of the orthonormal constituent vectors; therefore, any operator with real eigenvalues whose eigenvectors can form an orthonormal basis set is Hermitian. $\square$
    \\\\
    This proof in fact goes both ways: more significantly, any Hermitian operator possesses a set of eigenvectors which are an orthonormal basis set of the state space! The proof is rather technical,
    so it will be ignored-- but the profound consequences are clear. If an operator in the state space is Hermitian it has a basis consisting eigenvectors, called its \textbf{eigenbasis}, spanning the space; if we take any basis of the state space we can take its inner product over all the basis eigenvectors with any state vector to produce a wavefunction. These representations will prove massively helpful.
    \item[H4.] The action of all Hermitian operators whose eigenvectors form an orthonormal basis can be specified by their eigenvalues and eigenvectors.\\\\
    \underline{Proof:}\\\\
    For a Hermitian operator $\Omega$ with eigenvalues $\lambda_{i}$ and eigenvectors $\bm{\alpha}_{i}$, any vector can be expressed in the orthonormal basis:
    $$
    \Psi=\sum_{i}(\bm{\alpha}_{i},\Psi)\bm{\alpha}_{i}
    $$
    so 
    $$
    \Omega\psi=\Omega\sum_{i}(\bm{\alpha}_{i},\psi)\bm{\alpha}_{i}=\sum_{i}(\bm{\alpha}_{i},\psi)\Omega\bm{\alpha}_{i}=\sum_{i}(\bm{\alpha}_{i},\psi)\lambda_{i}\bm{\alpha}_{i}.
    $$
    This clearly requires no external knowledge other than understanding the sets $\{\lambda_{i}\}$ and $\{\bm{\alpha}_{i}\}$. Conversely: if we completely understand these sets then we can completely specify the operator given an input vector $\psi$ the operator is acting on. 
\end{enumerate}
With all this knowledge about operators in vector spaces, and clear signs that their eigenbases will be very useful, we now come to the Second Postulate of quantum mechanics.
\section{Observables in Quantum Mechanics}
The state problem is a question of information. The most relevant information to a physicist about a state is the value of its \textit{observables}. The problem, we have seen, is that unlike with the classical state, the quantum state cannot just be said to possess a value for any observable, as it is instead a superposition of many different possible states and it is thus far unclear which state will emerge upon measurement. This problem was dealt with by placing states in correspondence with vectors in a vector space which meant that all possible states could be summed together in a superposition to create a new state without forming something out of the space of possible states, and we learnt how to create more `tangible' wavefunctions from those state vectors. However, we still do not know why wavefunctions are so tangible. The genius of the formalism comes when we incorporate operators and their orthogonal eigenbases into the picture, where wavefunctions in eigenbases will answer our problem.
\\\\
\textbox{
\underline{\textbf{Postulate 2: Observables}}\\\\
To each physical observable there exists a corresponding hermitian operator. There exists an orthonormal eigenbasis of this operator which spans the state space: that is, for some observable operator $\hat{A}$ with eigenvalues $\{A_{i}\}$ corresponding to eigenvectors $\{\bm{\alpha}_{i}\}$,
$$
\forall\:\Psi, \:\: \Psi=\sum_{i}c_{i}\bm{\alpha}_{i}
$$
for some components $\setof{c_{i}}$. The only possible values for the observable whose operator is $\hat{A}$ which can be measured are the eigenvalues $\setof{A_{i}}$ and these correspond to the eigenstates $\setof{\bm{\alpha}_{i}}$. If there comes a condition 
$$
\hat{A}\Psi=A_{i}\bm{\alpha}_{i}
$$
then we say that the eigenvalue $A_{i}$ is the measured value of the physical observable and the new state vector is the eigenvector $\bm{\alpha}_{i}$.}
\\\\
The most essential outcome to us here is that we want real values for the results of physical measurements, as imaginary position or imaginary energy for example would be nonsense; since the set of eigenvalues $\setof{A_{i}}$ are the only possible results of measurements, we therefore require real eigenvalues. By the postulate, we have Hermitian operators representing observables, which therefore must have real eigenvalues. The postulate is that these eigenvalues are the only possible measurable results for that observable for any states, so that they are real is critical. It is from the fact we need the operator to have real eigenvalues if these are to be the measured values for a physical operator, combined with the fact it is postulated to have an orthonormal eigenbasis spanning the state space, which guarantees the operators representing observables must be Hermitian. That is precisely, we recall,
$$
\oip{\hat{A}\Psi_{1}}{\Psi_{2}}=\oip{\Psi_{1}}{\hat{A}\Psi_{2}}.
$$
We start a fuller discussion by noting the assertions of the postulate in short-form:
\begin{enumerate}
    \item[P2A1.] All physical observables are represented by hermitian operators whose eigenvectors form an eigenbasis which spans the state space. For brevity it is customary to call these operators which represent physical operators `observable operators' throughout the course of this book.
    \item[P2A2.] Each measurement of a physical observable must yield one of the real eigenvalues of the observable operator.
    \item[P2A3.] As it is possible for an operator to have a finite/and or discrete number of eigenvalues, so can a physical observable have a finite number of possible results after being measured if their operator has a finite number of eigenvalues. Such a situation is rarer than one would assume in quantum mechanics, considering we are working in an infinite dimensional vector space where it is relatively unlikely there are not infinite eigenvectors and eigenvalues to an arbitrary operator, but it is far from a non-existent possibility. The physical phenomenon resulting from discretely distributed eigenvalues is called quantization; its implications, most famously perhaps in the discrete energy levels of electrons, are important and may well have been already known to the reader.
\end{enumerate}
Now we must consider the importance of time. Crucial is that operators corresponding to physical observables never change with time, and there is only one operator corresponding to each physical observable.  That does not mean that the measured values of the observable will remain constant across any time period. That is because clearly the measured value of the observable depends on the state vector $\Psi$ representing the state of the system which is the input vector we the observable operator is operating on; we have already stated this state vector can evolve with time. The precise nature of all these temporal considerations will be covered in due course, but that observable operators do not evolve with time is surely a great relief, especially if we need to think about their eigenbases and eigenvalues and do not want to have to continually solve what are not trivial eigenvalue equations.
\\\\
The second part of the postulate gives us an interesting and significant link between the state vectors which represent states, the state space operators representing physical observables, and the eigenvalues of those observables representing possible results after measurement. 
\\\\
To start off, note that we expect that most state space vectors will be linear combinations of the eigenvectors of any observable operator, since the set of (infinite) eigenvectors of a hermitian operator constitutes an orthonormal basis of its space. We do not expect all possible states to be pure scalar multiples of single eigenvectors since there can be infinite linear combinations of the eigenvectors which are not pure scalar multiples of single eigenvectors. 
The postulate now states that the condition
\[
\hat{A}\Psi=A_{i}\bm{\alpha}_{i}\]
means that $A_{i}$ is the measured value of the physical observable represented by the operator $\hat{A}$. However, this condition is clearly very singular if we are working with a wavefunction which is a linear combination of the infinite eigenvectors of an observable operator-- since the state $\Psi$ does not naturally coincide with the eigenvector $\bm{\alpha}_{i}$ alone. What should be abundantly clear, however, is that, post-measurement, $\Psi$ has changed from a linear combination of eigenvectors to a multiple of only one of them, the eigenvalue corresponding to which is the measured value of the observable. So the act of measurement is clearly very important; indeed, it forms one of the central pillars of quantum mechanics and especially the mathematics which formulates it. We have seen this, already, in the Stern Gerlach experiment! There, measuring the $x$ spin led to irrevocable changes in the $y$ spin even without physically affecting it in that dimension. This measurement problem is the final component of the quantum mechanical solution to the state problem, and pulls everything together in an understandable way. Thus to complement Postulate 2 on observables, we have Postulate 3, on Measurements.
\subsection{Measurements}
\underline{\textbf{Postulate 3: Measurements}}\\\\
After a measurement of a physical observable, the state vector is forced into a specific eigenvector corresponding to the eigenvalue measured for that observable. The probability that the (normalised) state vector is forced into a state represented by a state vector $\bm{\alpha}_{i}$, which is called an eigenstate, is given by
$$
P(\bm{\alpha}_{i})=|\oip{\bm{\alpha}_{i}}{\Psi}|^2,
$$
which is therefore also the probability of measuring the eigenvalue $A_{i}$ as the final result of the measurement for the observable.
This Postulate now provides great meaning to the discourse immediately preceding this section. By \textit{forcing} a state vector into specific eigenvector of an observable operator after a measurement of that specific observable, we guarantee several things:
\begin{itemize}
    \item We do not restrict the state vector to being a pure scalar multiple of one single eigenvector prior to measurement. This is important as by Postulate 1, all state space vectors represent physical states, and there certainly should be infinite Hilbert space vectors which are linear combinations of any orthonormal basis vectors which span it: including when that orthonormal basis is the eigenbasis of an observable operator. This is the superposition of different possible states which exists before a measurement.
    \item We guarantee that after measurement, Postulate 2 has meaning: since the measurement forces the wavefunction into a specific eigenstate, we will indeed achieve after measurement
    $$
    \hat{A}\Psi\to \hat{A}\bm{\alpha}_{i}=A_{i}\bm{\alpha}_{i}
    $$
    and therefore we guarantee that a measurement will always yield one single value-- the eigenvalue $A_{i}$: regardless of what superposition of states it was in previously. 
\end{itemize}
This is all well and good, but what gives order to the chaos is the probabilistic link of the postulate. Without it, we would be wondering what to do in any arbitrary superposition in states, since intuition tells us that just because we have a superposition it does not mean that all the measurements must have equal probabilities. Fortunately, we have the postulate:
$$
P(\bm{\alpha}_{i})=|\oip{\bm{\alpha}_{i}}{\Psi}|^2
$$
where $P(\bm{\alpha}_{i})$ is the probability of measuring the state vector to be in the state $\bm{\alpha}_{i}$. Here we make a clarification of similar type as that regarding the distinction between state and wavefunction: the reader must understand that the state vector being in an eigenstate $\bm{\alpha}_{i}$ is not so interesting itself as is the fact that when it is in that eigenstate we know $A_{i}$ is the eigenvalue is the measured result for the observable. Thus when we say the state is measured to be the eigenstate $\bm{\alpha}_{i}$ we really allude to the fact that a measurement will yield $A_{i}$ as the value. The reason we do not write $P(A_{i})$, the probability of measuring $A_{i}$, is due to the fact that in the face of degeneracy (say, eigenvalue $A_{1}$ corresponding to two different eigenstates $\bm{\alpha}_{1}$ and $\bm{\alpha}_{2}$), we have the following problem:
$$
P(A_{1})\neq\oip{\bm{\alpha}_{1}}{\Psi}\neq\oip{\bm{\alpha}_{2}}{\Psi}
$$
in fact, here it would be 
$$
P(A_{1})=\oip{\bm{\alpha}_{1}}{\Psi}+\oip{\bm{\alpha}_{2}}{\Psi}.
$$
So we see that defining the probability of a wavefunction being in an eigenstate is slightly easier and more consistent. Next,
consider the state 
$$
\Psi=\bm{\alpha}_{n}.
$$
In this state a measurement will yield value $A_{n}$ with probability 
$$
P(A_{n}):=|\oip{\bm{\alpha}_{n}}{\Psi}|^2=|\oip{\bm{\alpha}_{n}}{\bm{\alpha}_{n}}|^2=1
$$
since the eigenvectors $\bm{\alpha}_{n}$ are assumed to be normalised. So if the state vector is a pure eigenstate then the eigenvalue corresponding to the eigenstate it is in will be measured with probability 1. When do pure eigenstates occur for state vectors? They may occur organically for some arbitrary physical state which happens to be a pure eigenstate of a physical observable, though we expect this to be comparatively rare. More importantly: they also occur after measurements, since by the first part of the postulate a measurement will force a state vector into an eigenvector -- a pure eigenstate -- of the observable operator. This now explains why instantaneous successive measurements must yield the same answer: the first measurement forces the state vector into a pure eigenstate $\bm{\alpha}_{n}$ corresponding to the eigenvalue $A_{n}$ measured, and then the second measurement will give the same eigenvalue $A_{n}$ with probability
$$
P(A_{n})=|\oip{\bm{\alpha}_{n}}{\bm{\alpha}_{n}}|^2=1
$$
since the state vector is now the pure eigenstate $\bm{\alpha}_{n}$ after being forced into this eigenstate by the first measurement. We saw this intuitive consequence in the Stern Gerlach experiment, where successive magnetic fields in the same axis yielded the same spin results each time!
\\\\
The disturbance to classical intuition comes when we make the same observation we have already made. By Postulate 1, all state space vectors represent physical states, and there are infinite state space vectors which are linear combinations of any orthonormal basis vectors which span it. Thus if the orthonormal basis is the eigenbasis of a physical observable operator, there are infinite state vectors which are not pure eigenvectors of the observable operator, but rather linear combinations of the corresponding eigenvectors. Then,
$$
\Psi=\sum_{i}\oip{\bm{\alpha}_{i}}{\Psi}\bm{\alpha}_{i},
$$
by the expansion of Hilbert Space vectors in an orthonormal basis. But then, for any $A_{n}$ in a non-degenerate state (similar reasoning holds for degenerate states),
$$
\begin{aligned}
P(A_{n}):=|\oip{\bm{\alpha}_{n}}{\Psi}|^2=|\oip{\bm{\alpha}_{n}}{\sum_{i}\oip{\bm{\alpha}_{i}}{\Psi}\bm{\alpha}_{i}}|^2
\end{aligned}
$$
so if this probability was zero then that would imply the component $\oip{\bm{\alpha}_{i}}{\Psi}=0$. In a nontrivial linear combination of eigenvectors, there will exist more than one eigenvector $\bm{\alpha}_{i}$ for which this is not true, and therefore more than one eigenvalue $A_{i}$ which can be measured with nonzero probability.
\\\\
This is the famous probabilistic nature of quantum mechanics encapsulated through our postulates. Do there exist such state vectors which are linear combinations of multiple eigenvectors of an observable operator? Certainly yes, by Postulate 1. Yet in such cases multiple eigenvalues can be measured with nonzero probability: that is, multiple values can be obtained for the same measurement. Thus the wording of the postulate-- the state vector is forced into a specific eigenvector-- is relevant: the state vector in these cases does not possess a single fixed value for an observable which must be revealed upon measurement so it cannot be said, technically, to have a position or momentum or any other observable value. We can only say that an eigenvalue is the measured value of this specific measurement: or, the eigenvector the state vector has been forced into was not necessarily the state vector before at all; in another scenario, with defined probability, the state vector may well have been forced into a different eigenstate and then yielded a different value for an observable. 
\\\\
We end this section with a summary on quantum states and inherent probability:
\begin{itemize}
    \item In a pure state with respect to a physical observable the state vector is made up solely of one eigenvector of the observable operator (up to some phase factor). A measurement will therefore yield the eigenvalue corresponding to that eigenvector with probability 1. In such cases (a rarity) a deterministic prediction can be made about the results of a measurement.
    \item \textit{After} measurement a state vector is forced into one of the constituent pure eigenstates $\bm{\alpha}_{i}$ with probability $|\oip{\bm{\alpha}_{i}}{\Psi}|^2$. Thereon the above determinism of a pure state applies for successive measurements of the same observable unless the system experiences external perturbation which moves it out of the pure eigenstate.
    \item A state vector is in a mixed state with respect to a physical observable if the state vector is a non-trivial linear combination of more than one eigenvector of the observable operator. In those cases the strongest predictive statement about the result of a measurement is that a specific eigenstate $\bm{\alpha}_{i}$ has probability $|\oip{\bm{\alpha}_{i}}{\Psi}|^2$ of being measured. We cannot make any deterministic predictions at all, and we do not really think of a mixed state as having a value for that specific observable. Most naturally occurring states in quantum mechanics are indeed mixed states.
\end{itemize}
\section{Probability Mass Functions}
We conclude this chapter on measurements, and indeed conclude the fundamental postulates dealing with the quantum state problem, with a formalisation of how discrete wavefunctions encapsulate probabilities as probability mass functions.
\\\\
We have already stated that the discrete wavefunction, which we denoted ${\psi}_{\bm{\alpha}}(x)$ with a domain of orthonormal eigenvectors, is exactly the function which stores the components corresponding to the eigenvectors we input. We therefore call it a \textbf{probability mass function}. This is a formal name for a very simple idea: it stores probabilities of discrete events- here, the event is the state vector being forced into a certain eigenstate by a measurement-- and can be extracted as an output of the probability mass function when we input the event (eigenstate). We know that these components are probabilities, because of the measurement postulate and the common expansion we have already proved! 
\\\\
If the discrete wavefunction is a probability mass function, then necessarily the modulus squared of its outputs (the probabilities of the state taking each pure eigenstate) must sum to exactly $1$. Thus as we must get some result for a measurement, the sum of probabilities is the probability that some result will occur, and thus 1. There is an important clarification to make to prove that our formalism works.
\\\\
\textit{Claim: The modulus squared components of a normalised state vector must sum to 1 in a discrete basis.}
\\\\
The importance of this claim is clear, since it is equivalent to the statement that the sums of the different probabilities for all the possible measurements of an observable must sum to $1$, which must be true if they are to be considered probabilities in the first place.
\\\\
\underline{Proof:}
\\\\
For some state vector
$$
\Psi:=\sum_{\{i\}}c_{i}\bm{\alpha}_{i}
$$ 
in some orthonormal basis $\setof{\bm{\alpha}_{i}}$, we need to prove that 
$$
\sum_{\{i\}}|(\bm{\alpha}_{i},\Psi)|^2=1
$$
given that the state vector is normalised. Well we know that 
$$
(\Psi,\Psi)=1
$$
so we know that 
$$
\biggl(\sum_{\{i\}}c_{i}\bm{\alpha}_{i},\sum_{\{i\}}c_{i}\bm{\alpha}_{i}\biggr)=1.
$$
Then, by the rudimentary expansion this is 
$$
\biggl(\sum_{\{i\}}(\bm{\alpha}_{i},\Psi)\bm{\alpha}_{i},\sum_{\{j\}}(\bm{\alpha}_{j},\Psi)\bm{\alpha}_{j}\biggr)=1.
$$
Due to linear distributivity this means that we get sum terms of the form
$$
\oip{\bm{\alpha}_{i}}{\Psi}^{\ast}\oip{\bm{\alpha}_{j}}{\Psi}\oip{\bm{\alpha}_{i}}{\bm{\alpha}_{j}}
$$
for some $i,j$. However, due to the orthonormality of the basis, all terms when $i\neq j$ disappear, so we have 
$$
\biggl(\sum_{\{i\}}(\bm{\alpha}_{i},\Psi)\bm{\alpha}_{i},\sum_{\{j\}}(\bm{\alpha}_{j},\Psi)\bm{\alpha}_{j}\biggr)=\sum_{\{i\}}\oip{\bm{\alpha}_{i}}{\bm{\alpha}_{i}}^{\ast}\oip{\bm{\alpha}_{i}}{\bm{\alpha}_{i}}=1.
$$
But then this is simply
$$
\sum_{\{i\}}|\oip{\bm{\alpha}_{i}}{\Psi}|^2=1.
$$
and our proof is complete.
\\\\
Thus indeed, we have the result that the square modulus components of the discrete state vector, which is, the square modulus of the outputs of its discrete wavefunctions, are valid probabilities of measurements. In this way, the encapsulation component of the state problem is much better understood. The full bijections are:
\begin{itemize}
    \item In a basis-independent bijection (by the state postulate),
    $$
    \text{Physical States} \leftrightarrow \text{State Vector}
    $$    
    \item In a basis-dependent bijection (by the properties of expansions in vector spaces):
    $$
    \text{State Vectors} \leftrightarrow \text{Component Maps}\leftrightarrow\text{Wavefunctions}
    $$
    \item By reinterpreting the coefficients (according to the measurement postulate):
    $$
    \text{Wavefunctions} \leftrightarrow \text{Probability Mass Functions} 
    $$
\end{itemize}
\section{Expecation Values and Ehrenfest's Theorem}
With this probabilistic formalism complete, we can now determine physically relevant qualities of a state given some state vector. For example, one of the most important physically relevant values is the \textbf{expectation value} of an observable.
\\\\
The expectation value is simple compared to what we have already discussed thus far. For a discrete random variable $X$, the expectation of $X$ would be defined and denoted
$$
\langle X \rangle := \sum_{{i\in \mathcal{I}}}x_{i}P(X=x_{i})
$$
if $\{x_{i}\}_{i\in \mathcal{I}}$ are the complete set of possible values for $X$. So there is no difference at all in the quantum mechanical case, where we treat the value of a measurement on a state $\Psi$ as the random variable in question:
$$
\langle A \rangle = \sum_{i\in\mathcal{I}}A_{i}P(\bm{\alpha}_{i})
$$
since $A_{i}$ is the (eigen)value for the observable $\mathcal{A}$ measured when the state vector is forced into the eigenstate $\bm{\alpha_{i}}$ in the event $\Psi \to \bm{\alpha}_{i}$. But this is then 
$$
\langle A \rangle = \sum_{i\in\mathcal{I}}A_{i}|(\Psi,\bm{\alpha}_{i})|^{2}.
$$
Why is this physical quantity relevant? Well, for one, while quantum states are strange on a quantum level (at small length scales), they may not appear strange on a macroscopic level, where one cannot distinguish between small differences between observable eigenstates.
\\\\
That is, even though all superpositions are technically possible quantum states, in practice most superpositions will consist of heavy weights around a specific small interval of values. One will essentially never get an electron with $50\%$ chance of being in a coordinate in California and $50\%$ chance of being in some coordinate in Paris. Reality is strange, but not unboundedly so.
\\\\
Does that mean that quantum phenomena are only valid on small scales? Not at all. Again, the only difference is we might not be able to \textit{notice} them on larger scales. It is still anticlassical for an electron to be in a superposition of many different possible positions, but if these are within the $1/1000$th of the radius of an atom, the human eye will detect no difference even if different measurements would yield different eigenstates.
\\\\
Hence, we can use the expectation value as a substitute for the actual measured value if we are considering a quantum system at a higher level, whether that be within an ensemble of other states, or at a more macroscopic level. We clearly do not \textit{always} have to be concerned with the exact superposition a state is in.
\\\\
Now, we introduce the connection between microscopic, quantum scales and macroscopic, classical scales. This is \textbf{Ehrenfest's Theorem}, at a very nonrigorous level.
\\\\
\underline{\textbf{Ehrenfest's Theorem}}:
\\\\
The laws of classical physics relating observables to each other hold for their quantum expectation values. 
\\\\
For example, the kinetic energy, denoted here by $T$, satisfies
$$
T = \frac{1}{2}mv^{2}
$$
which is, substituting the momentum $p$:
$$
T = \frac{p^2}{2m}.
$$
Then, Ehrenfest's Theorem says that for the Kinetic Energy Operator $\hat{T}$ and the Momentum Operator $\hat{P}$ (whose exact forms we do not know yet),
$$
\langle T \rangle = \frac{\langle P \rangle^{2}}{2m}.
$$
And since the quantum expectation is defined regardless of the scale of the system and the initial superposition, this is a \textit{deterministic} statement, not a probabilistic one. We may not know what value the measurement of $\mathcal{A}, \mathcal{B}$ will take, but we know what relation the expectation values $\langle A \rangle, \langle B \rangle$ should take.
\\\\
This theorem is usually introduced rather late in undergraduate quantum textbooks, because it requires notions we have not reached yet to prove it. Since this is an expository primer, I will not prove it, since the reader can in the future look for a proof themselves, and the proof is not particularly instructive.
\\\\
However, the theorem is clearly satisfying for us, because it provides a concrete bridge between quantum phenomena and classical phenomena. The reason why Newton's Principles apply to macroscopic objects like balls and tables is because of Ehrenfest's Theorem: the balls may be in quantum superpositions, but their expected values will follow Newtonian mechanics. And since at a microscopic scale we cannot detect microscopic differences in eigenstates, the expectation value will nearly always be the value we observe. 
\\\\
This is why it took so long for quantum mechanics to be developed. Without developing the experimental sophistication to probe systems at a microscopic level, it is difficult to observe quantum phenomena. This is also why you do not need to wonder if your fridge is in the kitchen or not even though by quantum superposition it may not take a concrete value until you have observed it!
\section{Basis Dependence of  Observable Operators}
When we refer to an \textbf{observable space}-- for example, ``position space" or ``momentum space", we mean the state space with the eigenbasis of a physical observable set as the fixed basis we are expanding and considering our vectors in.
\\\\
A large component of problem solving in quantum mechanics consists of being able to apply two concepts related to bases:
\begin{enumerate}
    \item Staying flexible without committing to a specific base and considering the state vector without a basis first.
    \item Switching into a basis which is useful for our specific needs and algebraic manipulations. Finally, calculating relevant physical values of the state in that basis (eg, probabilities, or expectation value).
\end{enumerate}
It is clear, even from the very early section on orthonormality, that not a bases are created equal (orthonormal bases are far cleaner and thus superior to nonorthonormal bases). Some are simply better selected because algebraic manipulation becomes a lot easier when a problem is considered in those bases. Out of such bases, the eigenbases of physical observables are the most significant. Thus we reach this definition of `observable spaces'. We understand that all these observable spaces are not actually new vector spaces: they are all this \textit{same} state space we have been studying, but simply with different eigenbases spanning the space and therefore where the same vectors will take different expansions. 
\\\\
All quantum mechanics problems are solved in the end in one of these spaces. Rarely, we might solve one part of a problem in one observable space and then switch bases (more in Chapter 7), after taking what we have learnt from that part, to solve the rest of the problem. Most of the time -- especially overwhelmingly in this book -- we work in position space where possible. This is not only because it makes sense to work with time and position as the two most important variables, as in our physical reality, but also because there are important variables to consider in the time evolution problem -- in particular the potential $V(x)$, a function of position-- which are easier to express in the position basis than any other. Momentum space is by far the other main counterpart in quantum mechanics, and usually we might work in momentum space in problems where the conditions or potential are easily expressed in terms of the momentum. Only rarely will we see energy space in action, and any other observable space will essentially be inferior to position and momentum space because the observables of position and momentum are far more important than any other.
\\\\
Now, there is an important conceptual detail to understand. Recall when we first were introduced to hermitian operators; we proved that each operator's action can be uniquely specified in its own eigenbasis if we have its eigenvalues. Consider some vector $\Psi$ in a the state space spanned by the eigenbasis $\setof{\omega_{i}}$ with eigenvalues $\setof{\lambda_{i}}$ of an operator $\Omega$. We have, by the common inner product expansion:
$$
\Psi=\sum_{i\in\mathcal{I}}\oip{\omega_{i}}{\Psi}\omega_{i}.
$$
Then the action of the operator $\Omega$ is 
$$
\Omega \Psi = \Omega\sum_{i\in\mathcal{I}}\oip{\omega_{i}}{\Psi}\omega_{i}
$$
which, as it is a linear operator,
$$
\Omega\Psi=\sum_{i\in\mathcal{I}}\oip{\omega_{i}}{\Psi}\Omega\omega_{i}
$$
which is,
$$
\sum_{i\in\mathcal{I}}\oip{\omega_{i}}{\Psi}\lambda_{i}\omega_{i}.
$$
So quite simply, we get 
$$
\Omega \Psi = \sum_{i\in\mathcal{I}}\oip{\omega_{i}}{\Psi}\lambda_{i}\omega_{i}.
$$
    Of course, it is very clear that these eigenvalues $\lambda_{i}$ are not obtained if we are not working in the space spanned by the eigenbasis of $\Omega$. This in turn shows us the very important fact: that the action of the operator-- not just the forms of the state vectors-- is \textbf{basis dependent}.  
\\\\
It is important to understand this idea, just as it is with different wavefunctions, which represent that expansions are different in different bases. So we will often talk about ``the momentum operator in position space" or ``the position operator in momentum space".   
\section{Summary}
In this chapter, we discussed the action of Hilbert space linear operators, and the special property of Hermiticity. We used these to define operators associated with physical observables, and in turn, we used the eigenvectors and eigenvalues of these operators to define physical measurements and pure states. The measurement postulate then taught us how to calculate numerically probabilities associated with the superposition, and we also learnt the famous property of ``the collapse of the wavefunction", which is that superpositions are collapsed into pure eigenstates when a measurement is made. Finally, we discussed the link between the microscopic quantum world and the macroscopic classical world, to provide a satisfactory explanation for why quantum phenomena do not occur in everyday life.
% \section{Unitary Transformations}
% We see above why we originally stated that actually the true bijection between physical states and state vectors is:
% $$
% \text{Physical States} \leftrightarrow \text{Hilbert Space Unit Rays}
% $$
% This is precisely because if the state vector is not a unit ray, that is, if it does not have norm $1$, then it cannot be interpreted as a probability mass function in a given basis, because the total probabilities of some value being measured upon measurement would not be equal to $1$.
% \\\\
% Now, we alluded to a useful definition. Consider the case of a general Hilbert space linear operator $U$ if:
% $$
% U^{\dagger}U = \bm{1}
% $$
% Note that $\bm{1}$ is the symbol usually used for the identity operator: a product of two operators should be an operator, not a scalar. Well then, if we take as an input some unit ray state $\Psi$:
% $$
% \Psi \text{\:\:is a unit ray} \iff (\Psi,\Psi) = 1
% $$
% (again, note the difference between the scalar $1$ and the operator $\bm{1}$). Then, applying $U$ to $\Psi$:
% $$
% \tilde{\Psi} := U\Psi
% $$
% satisfies
% $$
% (\tilde{\Psi},\tilde{\Psi}) = 
% $$





% \section{Simultaneous states}
% The question of ``simultaneous states'' deals with the segue from the previous section onto this one. We want to investigate which states (state vectors) can contain information on multiple observables at a time, and which observables these are. We have seen in the Stern Gerlach experiment that for example $x$ and $y$ spin simultaneous states are impossible, so this is a relevant, fully quantum in nature, problem.
% \\\\
% It turns out that the ability for different observables to have information represented in the same state vectors depends strongly on the relationship between their observable operators, as these in turn relate their orthonormal eigenvectors. To see this, there is one sweeping but simple theorem on operators for observables which can be measured simultaneously, and one dramatically anticlassical one for observables which cannot.
% \subsection{The Compatibility Theorem}
% Consider an unperturbed system, two physical observables, and three measurements ordered chronologically. The first and third measurements are for the first physical observable, but the second measurement is for the second observable. We know from the Measurement Postulate that:
% \begin{itemize}
%     \item The first measurement forces the wavefunction into a pure eigenstate of the first physical observable operator. 
%     \item The second measurement forces the wavefunction into a pure eigenstate of the second physical observable operator.
%     \item If the second measurement of the different observable did not exist, then we would have successive measurements of the same state (which is, the operator acting on the same eigenstate the starting state vector was forced into following the first measurement) and we would expect the same measurement for the observable as we originally obtained.
% \end{itemize}
% The question is therefore whether or not this second measurement changes the result of the third. This is a profound question, because if it does, then we would conclude the simple act of measuring the second observable has moved the state vector out of the pure eigenstate it was in after the first measurement; that would then imply the second measurement is in itself a perturbation to the system: a confusing result. Indeed-- the reader will recognise that this is exactly the class of behaviour which appeared so shockingly in the Stern Gerlach experiment.
% \\\\ 
% We shall see that the determining factor is what the relationship between the two observable operators is. To start off, we will use the notations $\mathcal{A}$ and $\mathcal{B}$ as shorthand to distinguish between the two observables, so we do not have to name them. We define $\mathcal{A}$ and $\mathcal{B}$ to be \textbf{compatible observables} if the first and third measurements yield the same value regardless of the starting state and the value of the second observable measured in the second measurement. If we call the values measured $\mathcal{A}^{(1)}$, $\mathcal{B}^{(1)}$, $\mathcal{A}^{(2)}$, then observable $\mathcal{A}$ and $\mathcal{B}$ are compatible iff 
% $$
% \forall\:\Psi,\:\mathcal{B}^{(1)},\:\:\mathcal{A}^{(1)}=\mathcal{A}^{(2)}.
% $$
% \textbf{Compatibility Theorem}: Two observables $\mathcal{A}$ and $\mathcal{B}$ are defined compatible if they possess a common eigenbasis or their operators commute. These three conditions in fact all imply each other.
% \underline{Proof:}
% \\\\
% First we prove that $\hat{A}$ commutes with $\hat{B}$ iff they possess a common eigenbasis. Consider two observable operators which commute, and define their eigenbases to be $\{\bm{\alpha}_{i}\}$ and $\{\bm{\beta}_{i}\}$. Now take an arbitrary eigenvector $\bm{\alpha}_{i}$ of $\hat{A}$ with eigenvalue $A_{i}$. We have
% $$
% \hat{A}\hat{B}=\hat{B}\hat{A}
% $$
% so we get 
% $$
% \hat{A}\hat{B}\bm{\alpha}_{i}=\hat{B}\hat{A}\bm{\alpha}_{i}=\hat{B}A_{i}\bm{\alpha}_{i}.
% $$
% However, we can now pull the constant eigenvalue out:
% $$
% \hat{A}(\hat{B}\bm{\alpha}_{i})=A_{i}(\hat{B}\bm{\alpha}_{i})
% $$
% so clearly $\hat{B}\bm{\alpha}_{i}$ is an eigenvector of $\hat{A}$ corresponding to eigenvalue $A_{i}$. Assuming that the eigenvalues are nondegenerate this implies that $\hat{B}(\bm{\alpha}_{i})$ coincides with $\bm{\alpha}_{i}$ as the eigenvalue $A_{i}$ has only one distinguishable eigenvector. The fact that 
% $$
% \hat{B}\bm{\alpha}_{i}\equiv\bm{\alpha}_{i}
% $$
% means we must have
% $$
% \hat{B}\bm{\alpha}_{i}=c\bm{\alpha}_{i}
% $$
% for some scalar multiple $c$. This means that $\bm{\alpha}_{i}$ is an eigenvector of $\hat{B}$ corresponding to eigenvalue $c$. So we can say that $\forall\:i, \bm{\alpha}_{i}$ is an eigenvector of $\hat{A}$ and $\hat{B}$: which means that they have the same eigenbasis. This isn't of course, to say, the eigenvalues are the same for $\hat{B}$ and $\hat{A}$ even though it may correspond to the same eigenvector (above, they are not the same unless $A_{i}=c$), since we expect the operators to be formulated differently so there will still be different values measured for each observable. Yet at the same time this is clearly helpful: if we know two physical observable operators commute and we have the eigenbasis of one then we automatically have the eigenbasis of the other. 
% \\\\
% Now, we prove it the other way around. Assume $\hat{A}$ and $\hat{B}$ both possess the eigenbasis $\{\gamma_{i}\}$. We want to prove they commute. As they possess the same eigenbasis with eigenvalues $\{A_{i}\}$ and $\{B_{i}\}$ respectively, we can write
% $$
% \hat{A}\hat{B}\gamma_{i}=\hat{A}B_{i}\gamma_{i}=B_{i}\hat{A}\gamma_{i}=B_{i}A_{i}\gamma_{i}
% $$
% and the exact same applies for $\hat{B}\hat{A}\gamma_{i}$:
% $$
% \hat{B}\hat{A}\gamma_{i}=\hat{B}A_{i}\gamma_{i}=A_{i}\hat{B}\gamma_{i}=A_{i}B_{i}\gamma_{i}.
% $$
% Clearly, as $A_{i}$ and $B_{i}$ are constant eigenvalues, 
% $$
% A_{i}B_{i}\equiv B_{i}A_{i}.
% $$
% So this easily proves that two observable operators possessing the same eigenbasis must commute. Thus the implication works both ways and therefore two observable operators commute iff they share a common eigenbasis.
% \\\\
% Now to look at the practical definition: we are probably more interested in the concept of compatibility, as it concerns whether or not a measurement of a second observable in between measurements of a first observable will alter the measured results from the first measurement, effectively  forcing the state vector out of the pure eigenstate it was forced into. Let's first prove that two observables having common operator eigenbases is necessary and sufficient for the above defined definition of compatibility to hold.
% \\\\
% Start by considering two observables $\mathcal{A}$ and $\mathcal{B}$ represented by operators $\hat{A}$ and $\hat{B}$ respectively. Define the measurements to be $\mathcal{A}^{(1)},\mathcal{B}^{(1)}$, $\mathcal{A}^{(2)}$. For the observables to be compatible we need $\mathcal{A}^{(1)}$ to be the same as $\mathcal{A}^{(2)}$ regardless of the starting state and $\mathcal{B}^{(2)}$. Assume to begin with that the two operators $\hat{A}$ and $\hat{B}$ have the common eigenbasis $\{\gamma_{i}\}$. By definition the first measurement of $\mathcal{A}$ must force the state vector into a single eigenvector in the eigenbasis of the operator $\hat{A}$: that is, some $\gamma_{i}$ such that the measured value is for observable $\mathcal{A}$ the eigenvalue $A_{i}$. Next, measurement ${{B}}^{(1)}$ is the action of the operator $\hat{B}$ on the eigenvector $\gamma_{i}$. But by the Measurement Postulate of quantum mechanics,
% $$
% P(\bm{\alpha}_{i})=|\oip{\bm{\alpha}_{i}}{\Psi}|^2
% $$
% That is, the probability that the arbitrary operator $\hat{A}$ forces the state vector into an arbitrary eigenvector $\bm{\alpha}_{i}$ from its eigenbasis. Here, then, since the state vector has been forced into the eigenstate $\gamma_{i}$ by the first measurement, the probability the second measurement of the other observable $\mathcal{B}$ forces the state vector into the same eigenstate is:
% $$
% P(\gamma_{i})=|\oip{\gamma_{i}}{\gamma_{i}}|^2=1
% $$
% where we assume as per usual that the eigenvectors $\gamma_{i}$ have been normalised. So we can say that measurement B will not alter the eigenstate the state vector is in and therefore the third measurement will follow the same logic to yield the exact same value, the eigenvalue $A_{i}$ corresponding to $\gamma_{i}$. Thus, if two observable operators possess the same eigenbasis, they are compatible observables.
% \\\\
% If the observables are compatible then this implies their operators have the same eigenbasis. The proof for this is simple. If the observables $\mathcal{A}$ and $\mathcal{B}$ are compatible then for the successive measurements $\mathcal{A}^{(1)},\mathcal{B}^{(1)},\mathcal{A}^{(2)}$ the measured values for $\mathcal{A}^{(1)}$ and $\mathcal{A}^{(2)}$ must be the same. The measurement $\mathcal{A}^{(1)}$ must have forced the wavefunction into an eigenvector of $\hat{A}$, some arbitrary $\bm{\alpha}_{i}$. Then, the measurement $\mathcal{B}^{(1)}$ must force the wavefunction into some arbitrary eigenvector $\bm{\beta}_{i}$ of the operator $\hat{B}$. However, the final measurement must yield the same result as the first if the observables are compatible, which is, the same eigenvalue corresponding to the same eigenvector $\bm{\alpha}_{i}$ of operator $\hat{A}$ as it originally was in. The probability that the measurement forces the wavefunction, currently in the eigenstate $\bm{\beta}_{i}$ of $\hat{B}$ as the measurement $\mathcal{B}^{(1)}$ has just been performed, into the same eigenstate $\bm{\alpha}_{i}$ as originally measured is:
% $$
% P(\bm{\alpha}_{i})=|\oip{\bm{\alpha}_{i}}{\bm{\beta}_{i}}|^2.
% $$
% However, if these observables are to be compatible, the final measurement must with certainty yield the eigenvalue $A_{i}$ again and therefore the above probability of measurement $\mathcal{A}^{(2)}$ forcing it back into the original eigenstate must be 1. So 
% $$
% |\oip{\bm{\alpha}_{i}}{\bm{\beta}_{i}}|^2=1 \Rightarrow\:\: \bm{\alpha}_{i}\equiv\bm{\beta}_{i}
% $$
% and therefore their eigenbases must be the same as the above holds true for any arbitrary $\bm{\alpha}_{i}$ and corresponding $\bm{\beta}_{i}$ from the measurements.
% \\\\
% The Compatibility Theorem is now complete. We have shown that:
% \begin{itemize}
%     \item Two operators commuting is necessary and sufficient for them to possess a common eigenbasis.
%     \item Two operators possessing a common eigenbasis is necessary and sufficient for the two observables they represent to be compatible.
%     \item Therefore, two observable operators commuting is also necessary and sufficient for them to represent compatible observables.
% \end{itemize}
% The logical implications of these facts all run three ways.
% \\\\
% While we have now seen facts about compatible observables, an example of incompatible observables sticks in our mind-- that of the Stern Gerlach experiment. We saw exactly that $x$ and $y$ spins were incompatible, because measuring the $x$ spin in between two $y$ measurements stopped the second $y$ measurement from being the same as the first with certainty-- which is to say, we now know, that the measurement of $x$ spin forced it out of the eigenstate of $y$ spin it had been previously forced into. All the questions about the quantum state raised by the Stern Gerlach experiment will finally come to an end with this section. We would like to formalise our understanding of how incompatibility affected the experiment. To explain it all, we witness-- and prove ourselves!-- one of Physics' most groundbreaking and shocking theorems.
% \section{The Heisenberg Uncertainty Principle}
% The idea of commuting observable operators being necessary and sufficient for the two observables they represent to be compatible is a very important one for the question of simultaneous states, and has been shown above. Now we must surely consider when two observable operators do not commute: in other words, when they represent \textbf{incompatible} observables. One of the most important and dramatic results of all quantum mechanics, the Heisenberg Uncertainty Principle, results when we carry out some elegant mathematics to investigate this problem. Before we begin the statement and proof, let us define the commutator between two operators to be 
% $$
% [\hat{A},\hat{B}]:=\hat{A}\hat{B}-\hat{B}\hat{A}
% $$
% so that if we have two commuting operators $\hat{A}$ and $\hat{B}$, then 
% $$
% [\hat{A},\hat{B}]=\hat{A}\hat{B}-\hat{B}\hat{A}=0
% $$
% since $\hat{A}\hat{B}=\hat{B}\hat{A}$ iff they commute. For operators which do not commute, their commutator may take a wide variety of forms: which is why it is useful under universal convention to have this shorthand.
% \begin{tcolorbox}
% \textbf{\underline{Heisenberg Uncertainty Principle}}\\\\
% For any state $\Psi_{t}$,
% $$
% \Delta A_{t}\Delta B_{t}\geq\frac{1}{2}|\oip{\Psi_{t}}{[\hat{A},\hat{B}]\Psi_{t}}|
% $$
% where $\Delta A_{t}$ is the standard deviation of measurable values of observable $\mathcal{A}$ at time $t$: which is therefore a measure of uncertainty for these variables.
% \end{tcolorbox}
% \underline{\textbf{Proof:}}\\\\
% We will continue to refer to arbitrary observables $\mathcal{A}$ and $\mathcal{B}$ for the proof; all the proof is relevant at any instant of time and so time subscripts will be eschewed. The notation $\Delta A$ refers to the standard deviation of the measurements of observable $\mathcal{A}$; this standard deviation is no different from the statistical definition:
% $$
% \Delta A=\sqrt{\langle \hat{A}^2\rangle-\langle \hat{A}\rangle^2}
% $$
% where the symbol $\langle X\rangle$ is the expected value of the variable $X$, as seen in the probability preliminary. First we note that this principle is valid for compatible observables: as compatible observables, their operators must commute. Thus
% $$
% [\hat{A},\hat{B}]=0 \Rightarrow\:\: \Delta A_{t}\Delta B_{t}\geq\frac{1}{2}|\oip{\Psi_{t}}{[\hat{A},\hat{B}]\Psi_{t}}| = \frac{1}{2}|\oip{\Psi_{t}}{0} |=0. 
% $$
% So for compatible observables, 
% $$
% \Delta A_{t}\Delta B_{t}\geq 0
% $$
% which is neither interesting nor invalid at all since the standard deviation of any measurement can never be negative. Now, we will prove this for all physical operators, regardless of whether they commute.
% \\\\
% \underline{\textbf{Lemma 1:}}\\
% Any operator $\hat{X}':=\hat{X}-\qexp{\hat{X}}$ where $\hat{X}$ is a Hermitian physical operator is also Hermitian.\\\\
% \underline{\textbf{Proof:}}\\\\
% Recall that the definition for an expected value of a variable is the sum of its possible values multiplied by the probabilities of the variable taking those values. Therefore, we can say that, over the eigenbasis $\{\xi_{i}\}$ of $\hat{X}$ with eigenvalues $\{X_{i}\}$, 
% $$
% \qexp{\hat{X}}=\sum_{\{i\}}P(\xi_{i})X_{i},
% $$
% but by our knowledge of the previous postulates we can describe the probability more precisely: the measurement postulate defines this to be
% $$
% \qexp{\hat{X}}=\sum_{\{i\}}X_{i}|\oip{\xi_{i}}{\Psi}|^2.
% $$
% Our job is to prove that the operator $\hat{X}':=\hat{X}-\qexp{\hat{X}}$ is hermitian if $\hat{X}$ is hermitian for all quantum operators. That is, we need to prove that:
% $$
% \oip{\Psi_{1}}{\hat{X}'\Psi_{2}}=\oip{\hat{X}'\Psi_{1}}{\Psi_{2}}
% $$
% for all Hilbert space functions $\Psi_{1}$ and $\Psi_{2}$. The operator $\hat{X}$ must be hermitian as $\hat{X}$ is defined to be a quantum operator corresponding to a physical observable. Meanwhile, the expectation value
% $$
% \qexp{\hat{X}}=\sum_{\{i\}}X_{i}|\oip{\xi_{i}}{\Psi}|^2.
% $$
% is clearly a real scalar, as the probabilities, which are square moduli, will all be real numbers and so will each eigenvalue of the hermitian operators. Therefore, 
% $$
% \oip{\hat{X}\Psi_{1}}{\Psi_{2}}\equiv\oip{\Psi_{1}}{\hat{X}\Psi_{2}}
% $$
% and 
% $$
% \oip{\qexp{\hat{X}}\Psi_{1}}{\Psi_{2}}\equiv\oip{\Psi_{1}}{\langle\hat{X}\rangle\Psi_{2}}\equiv\qexp{\hat{X}}\oip{\Psi_{1}}{\Psi_{2}}
% $$
% so for any physical operator $\hat{X}$ the defined operator $\hat{X}'$ is the sum of two hermitian operators. So
% $$
% \begin{aligned}
% \oip{\hat{X}'\Psi_{1}}{\Psi_{2}}&=\oip{[\hat{X}-\langle\hat{X}\rangle]\Psi_{1}}{\Psi_{2}}=\oip{\hat{X}\Psi_{1}}{\Psi_{2}}-\oip{\langle\hat{X}\rangle\Psi_{1}}{\Psi_{2}}\\
% &=\oip{\Psi_{1}}{\hat{X}\Psi_{2}}-\oip{\Psi_{1}}{\langle\hat{X}\rangle\Psi_{2}}\\
% &=\oip{\Psi_{1}}{\hat{X}'\Psi_{2}}
% \end{aligned}
% $$
% using the linear properties of the inner product. Thus, the operator $\hat{X}'$ is Hermitian for any physical operator. Therefore, defining $\hat{A'}:=\hat{A}-\qexp{\hat{A}}$ and $\hat{B}':=\hat{B}-\qexp{\hat{B}}$ for the purpose of the problem also gives us two hermitian operators. $\square$
% \\\\
% The commutator in the generalised principle might give pause with regards to the development of these new operators, but, importantly,
% $$
% [\hat{A}',\hat{B}']=[\hat{A},\hat{B}].
% $$
% This fact can be proved quite simply:
% $$
% \begin{aligned}
% [\hat{A}',\hat{B}']&= \hat{A}'\hat{B}'- \hat{A}'\hat{B}'\\
% &= (\hat{A}-\qexp{\hat{A}})(\hat{B}-\qexp{\hat{B}})-(\hat{B}-\qexp{\hat{B}}) (\hat{A}-\qexp{\hat{A}})\\
% &=(\hat{A}\hat{B}-\hat{A}\qexp{\hat{B}}-\qexp{\hat{A}}\hat{B}-\qexp{\hat{A}}\qexp{\hat{B}})-(\hat{B}\hat{A}-\hat{B}\qexp{\hat{A}}-\qexp{\hat{B}}\hat{A}-\qexp{\hat{B}}\qexp{\hat{A}})
% \end{aligned}
% $$
% but as the expectation values $\qexp{\hat{A}}$ and $\qexp{\hat{B}}$ are real scalars it is clear that $\qexp{\hat{A}}\qexp{\hat{B}}=\qexp{\hat{B}}\qexp{\hat{A}}$, and $\qexp{\hat{A}}\hat{B}=\hat{B}\qexp{\hat{A}}$ and vice versa swapping the $A$ and $B$ around. So the terms cancel out and we are left with
% $$
% [\hat{A}',\hat{B}']=\hat{A}\hat{B}-\hat{B}\hat{A}:=[\hat{A},\hat{B}]. \:\:\square
% $$
% Now, one last important lemma:\\\\
% \underline{\textbf{Lemma 2:}}\\
% $$\oip{\hat{A}'\Psi}{\hat{A}'\Psi}=(\Delta\hat{A})^2$$
% \\\\
% \underline{\textbf{Proof:}}\\
% By the Hermiticity of $\hat{A}'$,
% $$
% \oip{\hat{A}'\Psi}{\hat{A}'\Psi}=\oip{\Psi}{([\hat{A}']^2\Psi}.
% $$
% Expanding the definition,
% $$
% \begin{aligned}
% \oip{\hat{A}'\Psi}{\hat{A}'\Psi}&=\oip{\Psi}{\hat{A}'^2\Psi}\\
% &=\oip{\Psi}{[\hat{A}-\qexp{\hat{A}}][\hat{A}-\qexp{\hat{A}}]\Psi}\\
% &=\oip{\Psi}{[\hat{A}^{2}]\Psi-2\qexp{\hat{A}}\hat{A}\Psi+\qexp{\hat{A}}^2\Psi}\\
% &=\oip{\Psi}{[\hat{A}^{2}]\Psi}-2\qexp{\hat{A}}\oip{\Psi}{\hat{A}\Psi}+\qexp{\hat{A}}^2\oip{\Psi}{\Psi}\\
% &=\langle\hat{A}^2\rangle\oip{\Psi}{\Psi}-2\qexp{\hat{A}}\qexp{\hat{A}}\oip{\Psi}{\Psi}+\qexp{\hat{A}}^2\oip{\Psi}{\Psi}\\
% &= \langle\hat{A}^2\rangle-2\qexp{\hat{A}}\qexp{\hat{A}}+\qexp{\hat{A}}^2\\
% &=\langle\hat{A}^2\rangle-\qexp{\hat{A}}^2\\
% &=(\Delta\hat{A})^2 \:\:\:\:\:\:\square
% \end{aligned}
% $$
% Now we can use these lemmas to prove the problem. We want to prove that 
% $$
% \Delta{A}\Delta{B}\geq\frac{1}{2}|\oip{\Psi}{[\hat{A},\hat{B}]\Psi}|
% $$
% at all times $t$. We start by replacing $[\hat{A},\hat{B}]$ with $[\hat{A}',\hat{B}']$. Then, we have,
% $$
% \oip{\Psi}{[\hat{A},\hat{B}]\Psi}=\oip{\Psi}{[\hat{A}',\hat{B}']\Psi}=\oip{\Psi}{[\hat{A}'\hat{B}'-\hat{B}'\hat{A}']\Psi}.
% $$
% This is, 
% $$
% \oip{\Psi}{[\hat{A},\hat{B}]\Psi}=\oip{\Psi}{\hat{A}'\hat{B}'\Psi}-\oip{\Psi}{\hat{B}'\hat{A}'\Psi}
% .$$
% We can rearrange this by the hermiticity of $\hat{A}'$ and $\hat{B}'$:
% $$
% \oip{\Psi}{[\hat{A},\hat{B}]\Psi}=\oip{\hat{A}'\Psi}{\hat{B}'\Psi}-\oip{\hat{B}'\Psi}{\hat{A}'\Psi}=\oip{\hat{A}'\Psi}{\hat{B}'\Psi}-\oip{\hat{A}'\Psi}{\hat{B}'\Psi}^{\ast}
% $$
% so this is
% $$
% \oip{\Psi}{[\hat{A},\hat{B}]\Psi}=2i\text{Im}\left(\oip{\hat{A}'\Psi}{\hat{B}'\Psi}\right)
% $$
% according to rudimentary arithmetic of complex numbers. Then, the expression we need is
% $$
% \frac{1}{2}|\oip{\Psi}{[\hat{A},\hat{B}]\Psi}|\leq\frac{1}{2}\times2|\oip{\hat{A}'\Psi}{\hat{B}'\Psi}|=|\oip{\hat{A}'\Psi}{\hat{B}'\Psi}|.
% $$
% This is because of the above expression for $\oip{\Psi}{[\hat{A},\hat{B}]\Psi}$ and the fact that the modulus of the imaginary part of a scalar cannot be greater than the modulus of the scalar (Exercise 1.3.2a). Then, by Lemma 2
% $$
% \oip{\hat{A}'\Psi}{\hat{A}'\Psi}=(\Delta\hat{A})^2\Rightarrow\:\:\Delta\hat{A}=\sqrt{\oip{\hat{A}'\Psi}{\hat{A}'\Psi}}.
% $$
% So 
% $$
% \Delta\hat{A}\Delta\hat{B}=\sqrt{\oip{\hat{A}'\Psi}{\hat{A}'\Psi}}\sqrt{\oip{\hat{B}'\Psi}{\hat{B}'\Psi}}.
% $$
% By Cauchy-Schwartz, 
% $$
% \sqrt{\oip{\hat{A}'\Psi}{\hat{A}'\Psi}}\sqrt{\oip{\hat{B}'\Psi}{\hat{B}'\Psi}}\geq|\oip{\hat{A}'\Psi}{\hat{B}'\Psi}|
% $$
% and so, conclusively,
% $$
% \Delta\hat{A}\Delta\hat{B}=\sqrt{(\hat{A}'\Psi,\hat{A}'\Psi)}\sqrt{(\hat{B}'\Psi,\hat{B}'\Psi)}\geq|\oip{\hat{A}'\Psi}{\hat{B}'\Psi}|\geq\frac{1}{2}|\oip{\Psi}{[\hat{A},\hat{B}]\Psi}|
% $$
% so 
% $$
% \Delta\hat{A}\Delta\hat{B}\geq\frac{1}{2}|\oip{\Psi}{[\hat{A},\hat{B}]\Psi}|.
% $$
% This proves Heisenberg's Uncertainty Principle. $\square$
% \\\\
% This general form we have above is still difficult to interpret, but if we consider a few examples we will realise this is a very important result. One of the most famous iterations comes with considering simply the two central operators of quantum mechanics: the position and momentum operators, which we have not yet introduced but will for now just use for calculation purposes. We can calculate the commutator:
% $$
% \begin{aligned}
% &[\hat{X},\hat{P}]=\hat{X}\hat{P}-\hat{P}\hat{X}\\
% \Rightarrow\:\:&[\hat{X},\hat{P}]\Psi(x)=-xi\hbar\frac{\partial}{\partial x}\Psi(x)--i\hbar\frac{\partial}{\partial x}\biggl(x\Psi(x)\biggr)\\
% \Rightarrow\:\:&[\hat{X},\hat{P}]\Psi(x)=-i\hbar x\frac{\partial \Psi}{\partial x}--i\hbar\biggl[\frac{dx}{dx}\Psi(x)+x\frac{\partial\Psi}{\partial x}\biggr]\\
% \Rightarrow\:\:&[\hat{X},\hat{P}]\Psi(x)=-i\hbar\biggl[x\frac{\partial\Psi}{\partial x}-\Psi(x)-x\frac{\partial\Psi}{\partial x}\biggr]\\
% \Rightarrow\:\:&[\hat{X},\hat{P}]\Psi(x)=i\hbar\Psi(x)\\
% \Rightarrow\:\:&[\hat{X},\hat{P}]\equiv i\hbar.
% \end{aligned}
% $$
% After this, if we plug this into the Generalised Uncertainty principle and assume that the wavefunction is normalised,
% $$
% \Delta{\hat{X}}\Delta{\hat{P}}\geq\frac{1}{2}|\oip{\Psi}{i\hbar\Psi}| \Rightarrow \Delta{\hat{X}}\Delta{\hat{P}}\geq\frac{1}{2}|i\hbar\oip{\Psi}{\Psi}|=\frac{1}{2}|i\hbar|=\frac{1}{2}\sqrt{-i\hbar\times i\hbar}=\frac{\hbar}{2}.
% $$
% The key end result is that 
% $$
% \Delta{\hat{X}}\Delta{\hat{P}}\geq\frac{\hbar}{2}.
% $$
% This is the most well known form of the Uncertainty Principle, but we can see that the Generalised Uncertainty Principle can be applied more broadly than just to the two observables of position and momentum. 
% \\\\
% Returning to our considerations of the physical results of trying to measure two incompatible observables, it is clear how bizarre this result is. Consider if we have just made a measurement for the position of a particle. Then we have forced its wavefunction into a position eigenstate and therefore we can say that the uncertainty in the position is now $0$: we know the successive measurement must yield the same position value with probability 1. However, if we plug in $\Delta{\hat{X}}$ into the Uncertainty Principle we get 
% $$
% 0\times\Delta\hat{P}\geq\frac{\hbar}{2}
% $$
% which implies somehow that the uncertainty in momentum must be infinite! So if we know the value of the position with certainty we are completely unable to distinguish between infinite possibilities for the momentum. The relationship works both ways so the same applies for the momentum: if we know the momentum of a particle then we necessarily have infinite uncertainty in the position of the particle and we have not a clue where it is. This is undoubtedly one of the most anti-classical results in quantum mechanics, and yet it results beautifully from the mathematics we have defined (and has never been experimentally refuted). If nothing else, it should now be clear that the mathematical manipulations of quantum mechanics are rich and impactful.
% \\\\
% The same is manifested, of course, in the Stern-Gerlach experiment. By knowing $x$ spin, we had infinite uncertainty in $y$ spin- with absolutely no way to tell if an electron would be up or down spin. By knowing the $y$ spin, we had infinite uncertainty in the $x$ spin. This is one example of an experimental verification of the Heisenberg Uncertainty Principle.
% \section{Probability Mass Functions}
% We conclude this section on measurements, and indeed conclude the state problem, with a formalisation of how discrete wavefunctions encapsulate probabilities as probability mass functions.
% \\\\
% We have already stated that the discrete wavefunction, which we denoted ${\psi}_{\bm{\alpha}}(x)$ with a domain of orthonormal eigenvectors, is exactly the function which stores the components corresponding to the eigenvectors we input. We therefore call it a \textbf{probability mass function}. This is a formal name for a very simple idea: it stores probabilities of discrete events- here, the event is the state vector being forced into a certain eigenstate by a measurement-- and can be extracted as an output of the probability mass function when we input the event (eigenstate). We know that these components are probabilities, because of the measurement postulate and the common expansion we have already proved! 
% \\\\
% If the discrete wavefunction is a probability mass function, then necessarily the modulus squared of its outputs must sum to exactly $1$. This is naturally because the modulus squared of its outputs are the modulus squared of the probability amplitudes, which are probability densities and must sum to $1$, therefore. There is an important clarification to make to prove that our formalism works.
% \\\\
% \textit{Claim: The modulus squared components of a normalised state vector must sum to 1 in a discrete basis.}
% \\\\
% The importance of this claim is clear, since it is equivalent to the statement that the sums of the different probabilities for all the possible measurements of an observable must sum to $1$, which must be true if they are to be considered probabilities in the first place.
% \\\\
% \textbf{\underline{Proof:}}
% \\\\
% For some state vector
% $$
% \Psi:=\sum_{\{i\}}c_{i}\bm{\alpha}_{i}
% $$ 
% in some orthonormal basis $\setof{\bm{\alpha}_{i}}$, we need to prove that 
% $$
% \sum_{\{i\}}|(\bm{\alpha}_{i},\Psi)|^2=1
% $$
% given that the state vector is normalised. Well we know that 
% $$
% (\Psi,\Psi)=1
% $$
% so we know that 
% $$
% \biggl(\sum_{\{i\}}c_{i}\bm{\alpha}_{i},\sum_{\{i\}}c_{i}\bm{\alpha}_{i}\biggr)=1.
% $$
% Then, by the rudimentary expansion this is 
% $$
% \biggl(\sum_{\{i\}}(\bm{\alpha}_{i},\Psi)\bm{\alpha}_{i},\sum_{\{j\}}(\bm{\alpha}_{j},\Psi)\bm{\alpha}_{j}\biggr)=1.
% $$
% Due to linear distributivity this means that we get sum terms of the form
% $$
% \oip{\bm{\alpha}_{i}}{\Psi}^{\ast}\oip{\bm{\alpha}_{j}}{\Psi}\oip{\bm{\alpha}_{i}}{\bm{\alpha}_{j}}
% $$
% for some $i,j$. However, due to the orthonormality of the basis, all terms when $i\neq j$ disappear, so we have 
% $$
% \biggl(\sum_{\{i\}}(\bm{\alpha}_{i},\Psi)\bm{\alpha}_{i},\sum_{\{j\}}(\bm{\alpha}_{j},\Psi)\bm{\alpha}_{j}\biggr)=\sum_{\{i\}}\oip{\bm{\alpha}_{i}}{\bm{\alpha}_{i}}^{\ast}\oip{\bm{\alpha}_{i}}{\bm{\alpha}_{i}}=1.
% $$
% But then this is simply
% $$
% \sum_{\{i\}}|\oip{\bm{\alpha}_{i}}{\Psi}|^2=1.
% $$
% and our proof is complete.
% \\\\
% Thus indeed, we have the result that the square modulus components of the discrete state vector, which is, the square modulus of the outputs of its discrete wavefunctions, are valid probabilities in themselves of measurements. In this way, the encapsulation component of the state problem is much better understood: a state vector represents a physical state, we can then transform that state vector into wavefunctions in another bijection to the state and state vector, and in the discrete case the modulus squared of the outputs of these wavefunctions are the probabilities we need if we want to make predictions about measurements.
% \\\\
% Now, we are ready to move onto time evolution.

\section{Exercises from Chapter 4$\ast$}
\begin{enumerate}
    \item 
    \item
    \item
    \item
    \item
    \item
    \item
    \item
    \item
    \item
\end{enumerate}