\chapter{Chapter 2: The Stern Gerlach Experiment}
In the Exoskeleton to this book, I presented the idea of modelling very explicitly to the reader, because I think that a more authoritative overview on the \emph{procedure} of what we are trying to do is important and often neglected by compendium-style textbooks. Considering this is a mathematics textbook, we have opened up with a whole lot of words. However, this is, to my estimation, the best way, and the result is that the reader should get a `bird's-eye' view of the subject which will be a helpful guide throughout, before we open the algebra and our horizons immediately become narrowed down for some time.
\\\\
To complete our `bird's-eye' view over quantum mechanics, then, we will now investigate one of the radical experiments of the early 20's which necessitated the birth of quantum mechanics to take the mantle from a classical model which was suddenly proven woefully inadequate. I have chosen the Stern Gerlach experiment, involving electron deflection in a magnetic field. Other practical experiments, such as the Davisson-Germer modification of Young's Double Slit or the Mach-Zehnder Inferometer, are also often used as opening preludes to the quantum model, but the Stern Gerlach experiment is amongst these the easiest-accessible exhibition of the phenomena we will be trying to mathematically model, without much additional Physics knowledge required. What we herein are trying to understand is what exactly the demands-- from physical reality-- quantum mechanics had to meet were. It is a model, and models are valuable only based on how well they meet the demands of the predictions they need to be capable of making. With these extraordinary experimental results, quantum mechanics very much was given a deeply challenging and novel set of demands, so its rules -- the Postulates -- can be said to be directly motivated by them.
\\\\
Thus, understanding these motivations provides a stabilising force of unification between the otherwise unexplainable haphazard mathematical abstractions which will follow. The author strongly suggests understanding Planck's (discretised) solution to the blackbody radiation problem, Bohr's new atom model, Young's double slit experiment, De Broglie's matter waves, and Born's radical idea of ``probability clouds" before beginning to read this book. These need not be learned out of textbooks or with any length or depth of study at all-- only the most superficial familiarity with the concepts is needed-- but are very useful to have a basic grasp of in order to understand what new issues quantum mechanics had to address when it was conceived. We cannot cover all the above physics experiments in any true detail, because doing so not only violates the mathematical focus of this book-- dissipating the reader's attention with phenomenological distractions-- but also requires far more complex physics than can fit into this textbook. 
\\\\
Now, then, if the reader feels introduced to the ideas of inherent probability and superposition, we will press on with a very concrete example to see these principles in real world practice.
\section{The Stern-Gerlach experiment}
The idea of inherent probability has been introduced and is well covered even by non-mathematical media on quantum mechanics. If the reader has familiarised themselves with these background ideas of inherent probability, we can investigate this further, and the consequences it has on the quantum state, by turning to the Stern-Gerlach experiment. The reason this experiment is so useful is because it allows us to focus on a particular \textit{observable}-- that is, something we measure in an experiment-- which can only take two possible values. This allows greater lucidity in how we organise a systematic investigation into the strange quantum mechanical phenomena of probabilities.
\\\\
The setup deals with the magnetic properties of silver (Ag) electrons. Specifically, we consider what happens when we pass them through a magnetic field. What we expect is for deflection to occur; this is due to a property of electrons, called spin, which is directly influenced by the magnetic moment exerted on the electrons by the magnetic field. A problem with spin is that it has no classical analogue- that is, we have not seen any such quality with our own eyes in our macroscopic world; indeed, the physicist Pauli describes spin as something `not describable classically'. Nevertheless, all the reader needs to know is that this quality determines the different ways an electron is deflected by the magnetic field, and that it is a measurable quality. We can reduce each electron to be considered as a single point particle with a particular spin, by performing some net cancellations. So we will make that assumption that the particles we are working with in the experiment are single point particles with some spin values.
\\\\
Now, after electrons are deflected by this magnetic field, they are detected by an electron detector on the other side, which we can visualise as a simple screen which shows where the electrons land. The expectation classically is that there is a large patch of electrons formed on the detector screen, bounded by the furthest deflections achieved (and possible) during the experiment. This is because we expect that spin is some continuous value which may differ from electron to electron, resulting in slightly different respective deflections by the magnetic field: thus forming in this patch on the detector screen.
\\\\
However, the results which are achieved are rather different. There are only two positions at which electrons are detected after deflection on the screen. If we proffer that this quality of spin is the only facet which changes the deflection-- other than the magnetic field, which is the same for every electron passing through it and constant throughout-- then this necessarily means we have observed the \textit{quantisation} of spin: the property of an observable having a finite set of discrete values-- here, two. There are two possible deflections, and therefore there must be two possible spins given that nothing else is changing.
\\\\
In the above we could have made a subtle semantic adjustment and said: ``there are two possible deflections, and so the electrons must have one of two possible spins before passing through the magnetic field". We will show that, while this might seem sensible to classical intuition, it is in fact impossible for us to say this because it is impossible for an electron to have a specific value of spin before passing through the magnetic field! This is a dramatic statement, because under classical intuition every object we study must have some value for any observable at all times. We can change the momentum of a baseball by striking it, but that doesn't mean it did not have a momentum before we struck it and measured the result. Somehow, quantum mechanics tells us that with spin, the electron can exhibit a certain spin value after it passes through the magnetic field, but does not necessarily have one before it does so. This is a prime example of purely quantum behaviour, and was indeed why the Stern Gerlach experiment, which strongly suggests this fact, was so seminal as one of the experiments which necessitated the birth of quantum mechanics.
\\\\
To understand how the results of the experiment show us this strange behaviour must exist, we will engage in some sort of proof by contradiction. We will assume the opposite is true: that the electron must have a spin value at all times, and show that the results of the experiment lead to a contradiction if this assumption is true. By doing this, we will thereby prove our assumption is incompatible with the results of experiment, and that therefore the electron cannot in fact have a spin value at all times.
\\\\
        In our work, we will focus on the two-dimensional Stern-Gerlach experiment. The primary dimension we focus on will be the $y$-dimension-- or, as we will think of it, the ``vertical" direction. Here, the two possible spins in our experiment are ``up spin", and ``down spin". More interesting things come when we consider a second dimension, so we will use the $x$-dimension as our secondary dimension, where the two spins are ``right spin", and ``left spin". Corresponding to these dimensions there are the $y$-dimension magnetic fields and $x$-dimension magnetic fields, which deflect electrons in those dimensions alone.
\\\\
Let us now take our classical assumption on until we reach a contradiction: we will assume that the electron must already have some value of spin before passing through the magnetic field. Thus begins our proof by contradiction.
\\\\
Let us first consider the purposes of the magnetic fields. It might seem like they are performing a physical action on the electrons- which they are: they are deflecting them. However, the real purpose of the magnetic fields in this experiment is to \textit{reveal} the spin value of electrons which we assume they possess. By passing a single electron through the $y$-dimension magnetic field it is either deflected down, if it had down spin, or it is deflected up, if it had up spin. In this way, the action is of passing an Ag electron through the magnetic field is the action of measuring the spin it has, because we reveal the spin through observing the direction of deflection. 
\\\\
The second quality of the magnetic field is that it separates a group of electrons into two new groups corresponding to the two possible spins in its dimension. Note that it can do both of these functions at the same time: it deflects down spin electrons down and up spin electrons up, which both splits them into distinguishable groups, and measures (reveals) their spin. After passing a random beam of electrons through the magnetic field, we will get these two resulting groups split by their initial spin: surely, a useful thing to have. 
\\\\
If this is the case, then we expect successive $y$-dimension magnetic fields to yield a clear result when applied to a group of electrons. The first $y$-dimension magnet separates the group of electrons into electrons with up spin and electrons with down spin. Suppose we allow for enough distance to distinguish the two separate deflections of up and down, and place two new magnetic fields at that distance, one in the up deflection position, and one in the down deflection position. Then, we therefore expect the second layer of magnetic fields to perform the same measurement function, but with different results since the original magnetic field originally separated the inital beam into two groups based on the two possible spins. Specifically, the second $y$-dimension magnetic field in the up-deflected position should measure all its electrons to have up spin, and the second $y$-dimension magnet in the down-deflection position should measure all its electrons to have down spin. This is of course because they have already been separated into two distinct groups and measuring the spin of all the electrons in those distinct groups will only yield one result, under our assumption that the electrons hold a spin value at all times. We did not know the spin value of the electrons before we passed them through the first magnetic field, but after we do all our revelations are done.
\\\\
This is indeed what we achieve when we do actually perform the experiment. The first $y$-dimension magnet splits a group of electrons into two groups corresponding to each spin, and an infinite succession of $y$-dimension magnetic fields on each of these deflected trajectories will continue to measure the same spin by continuing to deflect them in the same direction. The same holds for successive $x$-dimension magnetic fields: the first separates the electrons into two groups with left and right spin respectively, and the subsequent magnetic fields will continue to measure the same spins if applied successively along the two initial trajectories obtained by the first magnet.
\\\\
Of course, physicists were more inspired than just applying the same magnetic field over and over again on the same electrons. We might wonder what would happen if we applied some sequence of $y$ and $x$ magnetic fields. Let us note that electrons passing through a magnetic field in a certain dimension are not deflected in anyway in any other dimension. We implicitly see this in the fact that electrons passing through the $y$ magnetic field follow a straight trajectory with respect to the $x$-dimension and therefore still all hit the detector screen a certain $x$ distance away from the field.
\\\\
Our assumption is that an electron at all times holds a value for its spin. We therefore also posit that an electron may at one time hold two simultaneous properties of $x$ spin and $y$ spin, which are independent of each other: just like a ball may have a $y$ axis momentum value and an $x$ axis momentum value at the same time. We test the relationship between the $x$ and $y$ spins of an electron: is an up spin electron more likely to be a right spin electron, for example? Understandings like this are physically relevant, of course- if a electron passing through `unbiased' magnetic fields was more likely to be deflected right and up or left and down then this would surely have physical implications. 
\\\\
Suppose we isolate a certain $y$-spin and only consider electrons with that $y$-spin: say, up spin. To do this experimentally is simple-- pass a group of electrons through a $y$-magnetic field, give sufficient distance for the deflected trajectories to diverge, and block the down-deflected path so that none of those electrons can keep moving forwards and the only ones moving forwards are the up spin electrons. Let us suppose we did this. Then, we can investigate what happens when we place an $x$-magnetic field in the path of the up-deflected electrons. The point of blocking out the down-deflected electrons in this way is to assure us we will past this point never have to deal with down spin electrons during our considerations.
\\\\
The $x$-magnetic field on any ensemble of electrons will not show considerable bias towards any spin, in fact. We obtain two new deflected trajectories, therefore: up-left and up-right. Again, this all seems good, until we add another $y$ magnetic field, after which a bizarre phenomenon occurs.
\\\\
If we add another $y$ magnetic field to the trajectory of the up spin electrons which we have just deflected left or right with the $x$ magnetic field, we would expect all the electrons to be deflected upwards, since these are after all electrons with up spin. We know that we have blocked out all the down spin electrons far back, so they cannot escape to intervene with the results of our current experiment.
\\\\
Such is not the case. What we get instead is that both the up-left and up-right ensembles of electrons have subgroups of electrons which are deflected downwards by the new $y$ magnet! This by all means should absolutely be impossible, given we have filtered out all the down spin electrons already. Clearly, something is wrong. 
\\\\
To keep track of everything following this revelation, we will use a new notation: $\uspin$, $\dspin$, $\lspin$, $\rspin$ for different spins, as well as two-fold notations such as $\urspin$ to record when an electron has had its $y$ spin and $x$ spin measured. This is in a sense a crude version of a `state notation',
as we are attempting to label the state of a given electron-- at least with respect to its spin properties. The reason this is so useful is, among other things, simply that it saves us a lot of time by avoiding having to write ``up-right spin electron" all the time.
\\\\
This state denotation model is just some arbitrary and meaningless labelling, so one should not carried away with its importance other than for making things more concise. Nevertheless, let's take this state denotation one final step further by adding scalars to the mix. Specifically, we will use scalars to show how many electrons are in each state for a given set. For example, we could have 
$$
1000\elec=500\uspin + 500\dspin.
$$
That is: we have some 1000 electrons, 500 of which are up spin and 500 of which are down spin. Similarly, we could say 
$$
500\uspin = 250\ulspin + 250\urspin
$$
or even, 
$$
500\uspin \implies 250\ulspin + 250\urspin + 0\dlspin + 0\drspin.
$$
That is, out of some sample of 500 up spin electrons, we have measured 250 to have up left spin and 250 to have up right spin. We can use the $\implies$ arrow to show that there is a change after we place the electrons in a magnetic field. With this interlude we can now clearly express what is strange about our above result again-- given our original classical assumption-- by tracking what information was revealed \underline{after} each step:
\begin{enumerate}
    \item Pass the electron beam through a $y$-magnetic field:
    $$
    1000\elec \implies 500\uspin+500\dspin
    $$
    assuming we start with 1000 electrons and the sample is indeed random.
    \item Block out the down-deflected channel:
    $$
    500\elec \implies 500\uspin+0\dspin
    $$
    reflecting that we now have 500 electrons since 500 down spin electrons have all been blocked out from the experiment.
    \item Pass the electron beam through an $x$-magnetic field:
    $$
    500\elec \implies500\uspin+0\dspin=250\ulspin+250\ulspin+0\dspin
    $$
    assuming again an equal distribution which is experimentally realistic.
    \item Block the left spin channel.
    $$
    250\elec\implies0\ulspin+ 250\urspin + 0\dspin
    $$
    \item Finally, pass the electron beam through a $y$-magnetic field.
    $$
    250\urspin\implies125\uspin+125\dspin
    $$
\end{enumerate}
The scalars themselves are really not important at all: one needn't think about numbers of electrons, as they are irrelevant in these scenarios. What is relevant is what happens if we track what happens to the $\dspin$ count between these above steps.
\begin{enumerate}
    \item $500\dspin$
    \item $0\dspin$
    \item $0\dspin$
    \item $0\dspin$
    \item $125\dspin$
\end{enumerate}
We have followed one single trajectory! So we should not be counting 500, 0, 0, 0, 125 unless down spin electrons have materialised out of thin air (this is impossible, rest assured). We might therefore guess that some of the $0$ counts are false. We can consider this:
\begin{itemize}
    \item In step 2, we have just blocked out all the down-spin electrons. The 0 count surely therefore must be accurate.
    \item In step 3, we have just passed the electron beam through an $x$-magnetic field. We might offer 1 possible guess: that the $x$-magnetic field actually influences the $y$-spin and that it is here therefore that the subsequent 0 count was wrong.
    \item In step 4, we blocked out all the left-spin electrons. If the 0 count here is false, that means that either blocking out left-spin electrons switched the $y$-spin of some of the electrons, or that there already were down spin electrons by this point before we blocked out the left-spin electrons. If there were down spin electrons before step 4 was implemented then this necessarily means again that after step 3 the 0 count was already wrong.
\end{itemize}
So everything points to the only reasonable explanation being that the 0 count after step 3 of the $x$-magnetic field was false: and this would only occur if the $x$-magnetic field actually flipped the $y$-spin of some electrons. We first remember that the counter detector has only actually measured the number of left spins and right spins after step 3, since an $x$-magnetic field does not make any discernible $y$-deflections. So it is certainly true that if the $x$-magnetic field did flip some of the electron $y$-spins we would not have detected this since it would not have manifested in a deflection.
\\\\
To test this one lifeline for our assumption, there is a cunning modification to the apparatus. We recall that what we did originally was place successive magnetic fields a relatively large distance away from each other such that different deflected trajectories could be given sufficient distances over which to diverge from each other and therefore be discernibly different according to the measurements of detectors. This makes sense if we simply consider the fact that we are considering magnetic deflections of electrons, which are obviously not significant at all in scale.
\\\\
On the other hand, we can also well imagine that we can let both deflected paths pass through the same magnetic field, since the angle of deflection is small. If we try this, we continue to add to the list of complications. We can use our step by step denotation to track things again:
\begin{enumerate}
    \item Pass the electron beam through a $y$-magnetic field:
    $$
    1000\elec \implies500\uspin+500\dspin
    $$
    assuming we start with 1000 electrons and the sample is indeed random.
    \item Block out the down-deflected channel:
    $$
    500\elec \implies500\uspin+0\dspin
    $$
    reflecting that we now have 500 electrons since 500 down spin electrons have all been blocked out from the experiment.
    \item Pass the electron beam through an $x$-magnetic field:
    $$
    500\elec=500\uspin+0\dspin\implies250\ulspin+250\urspin
    $$
    assuming again an equal distribution which is experimentally realistic.
    \item Pass both the up right and up left electrons through a new $y$ magnetic field.
    $$
    500\elec=250\ulspin+ 250\urspin\implies 500\uspin
    $$
\end{enumerate}
So if we compare the results of adding a new $y$ magnetic field to the experiment (step 5 in the previous experiment and step 4 here):
\begin{itemize}
    \item If we remove one of the right or left channels of $x$-deflected electrons, then both up and down channels emerge from the $y$-magnetic field applied to the remaining beam.
    \item If we keep both of the right and left channels, then only one $y$ deflection emerges from the $y$-magnetic field applied to the remaining beam.
\end{itemize}
The above effectively cancels out any idea that the $x$-magnetic field is flipping the $y$ spin of some of the electrons. We had guessed that in the first experiment that the reason why down spin electrons were appearing seemingly out of nowhere (having been blocked out, we thought, earlier) was because the $x$-magnetic field was flipping these electrons. If this was the case, then adding left spin electrons would not change anything, since the spin of one electron most definitely does not effect the spin of another. Therefore, if the $x$ field really did flip electrons when the left spin electrons were still absent, adding them would simply mean we still have down spin electrons (perhaps more, since presumably some left spin electrons would be flipped too if right spin electrons are). Yet adding these left spin electrons eliminates rather than augments the number of down spin electrons. Since the spin of one electron does not affect the spin of another electron, we thereby infer our hypothesis that an $x$-magnetic field can flip a $y$ spin is false.
\\\\
This now undoubtedly therefore suggests a problem with our original assumption that an electron has pre-determined spin values, within which there can be no explanation for certain empirically observed phenomena. Therefore, the assumption that electrons have a spin value at all times leads to a contradiction with empirical results and must be false.
\section{Remodelling the Stern Gerlach Experiment}
Within the next two chapters from this one, the exact reason for our bizarre results will become quite clear. However, the gist is already apparent: it seems like it is impossible for an electron to have pre-determined spin values at every point in the experiment: and that they only have pre-determined values at some very specific points in the experiment. We need to ensure this could be consistent with experimental data, so let us do this.
\\\\
One clue comes when we repeat the exact same experiment, but with $x$, $y$, $x$ magnetic fields rather than $y$, $x$, $y$ fields. The exact same set of  phenomena occurs, but simply with the sequence of spin dimensions switched around to reflect the new sequence of magnetic fields. The explanation we now proffer-- which is finally the correct explanation-- is that we cannot know the $x$ and $y$ spins of the electrons simultaneously. This is obviously not a classical explanation, and is our first experience of a quantum phenomenon in action- that there can be observables which cannot simultaneously both hold values for a given state was certainly shocking to physicists of that time and should be equally shocking to us. 
\\\\
This does however, become consistent with all the phenomena we have been getting in the experiments explained. When we have both left and right channels going into the same magnet, out from the magnet emerges one single channel (in our case, up). This is because, when both channels enter the same magnetic field we can't tell which of them have left and which of them have right spin-- and therefore we can tell afterwards what their $y$ spin is. However, if we pass only one $x$-deflected channel through the magnet, then we can tell what the $x$ spin of the input electrons are by what channel we have chosen, and therefore some are deflected down and some are deflected up-- we can no longer tell what their $y$ spin is. 
\\\\
The facts above, however, are presented to imply another question which someone trying to cling onto classical common sense would reasonably ask. It may be true that if we stand above the apparatus and only look at detector readings we cannot tell which electrons have which spin when we are passing all of them into the same magnetic field. However, if we were to focus in on electrons one by one, tracking which electrons have which spin, surely we could for example track the $x$ spins of each of the electrons passing into the $y$ field and then each of the $y$ spins when they come out of it: and thereby know the $x$ and $y$ spins simultaneously?
\\\\
Investigating this leads us to try and repeat the experiment with as few electrons as possible.
The first thing we realise, when doing so, is that 10 electrons and $10^8$ electrons exhibit the same types of phenomena when we study them, though naturally the groups of electrons are different in quantity for the two scenarios. In other words-- intensity (a measure of the number of electrons per unit area in the beam) does not affect the experiment. Well this is good news, since that means we can reduce to single electrons in our electron gun and once and for all understand the phenomenon!
\\\\
All seems well. The first $y$ magnetic field, let's say, gives the electron to have up spin. The second $x$ magnetic field, might give a right spin. So we know both of the $x$ and $y$ spins simultaneously, we think: seemingly invalidating our quantum hypothesis that we could not do this. To confirm, we put a new $y$ magnetic field to our up-right spin electron. And finally, we get a nasty shock, because (sooner or later, if we repeat the single electron analysis multiple times) the given electron which we measured to have up and then right spin suddenly exhibits a downwards deflection, showing down spin.
\\\\
\textbf{This is the probabilistic state}. We have already proved that the magnetic fields cannot flip any spins, and only reveal the spin in the relevant dimension where the magnet is set to. Yet this electron has exhibited up and down spin in successive measurements. This leads us to the only possible conclusion. Our overarching point: that an electron cannot have a spin value at all times, must be true. This would be in line with our other quantum hypothesis that we cannot know simultaneously the $x$ and $y$ spins. The electron \textit{cannot} have a spin value at all times, and this means that there are times it does not have a fixed spin value, and in particular we now know that these times where the spin value cannot be known at least include all the times when the spin value in another dimension is known.
\\\\
Why do electrons still get deflected, if they do not have spin values? The logical explanation is the correct one this time: an electron might not \textit{have} a spin value, but it can \textit{take} a spin value. It turns out that it takes a spin value with specific probabilities, which we will soon learn how to calculate. So what we achieved in the first two measurements on this single electron: an up and right spin, is simply a result of the electron taking some the up value for $y$ spin with some probability ($\frac{1}{2}$ in our example) and then taking the left value for $x$ spin with some probability (also $\frac{1}{2}$ in our example). However, if we measure $y$ then $x$, then trying to measure $y$ spin again results in a new probabilistic trial: we \textit{could} get up again, but we could also get down this time. The $y$ spin measurement will be a completely new outcome when we try to measure it again, and therefore we never did know the two spins simultaneously!
\\\\
A few things are maintained in this world of chaos, fortunately. We recall that applying the same magnetic field to an electron beam over and over again gives the same spin every time. This would not be true if for each magnet we were simply experiencing a probability of $\frac{1}{2}$ that it exhibits up spin and $\frac{1}{2}$ of down spin, since an infinite sequence of magnetic fields would eventually of course invoke that $\frac{1}{2}$ probability of the other spin to the one we are measuring every time. This therefore proves that if we measure the spin of an electron it \textit{takes} a state- here, with probability $\frac{1}{2}$ of taking an up spin state and $\frac{1}{2}$ of taking a down spin state- and stays in this state until we force it out of that measured state with some type of action. The example we have seen for one type of action which takes an electron out of the original state it has taken, shockingly, is trying to measure its $x$ spin. By measuring its $x$ spin we force it into either a left or right state (here, also with probabilities $\frac{1}{2}$), but also revert it into an unmeasured state with respect to its $y$ spin in the process of measuring its $x$ spin.
\\\\
This term `unmeasured state' is a clumsy denomination. Quantum physicists came up instead with the term \textbf{superposition} of states-- literally, the idea of different possible states being added onto each other to make a new state. Here, the superposition is of two states (up and down in the $y$ spin, or left and right in the $x$ spin). However, if we were considering different observables with more than two possibilities- such as position, which has infinite possibilities- then we could have a superposition of infinite possible states. Of course, some of these states in the superposition might have a greater probability of being measured than others: which is why the idea of superpositions and inherent probability are inextricably linked when we are considering the states of quantum systems. Through our long proof by contradiction, we have thus established that:
\begin{itemize}
    \item A system may exist in a superposition of states.
    \item A measurement on a system in a superposition forces it into one of the possible states in that superposition.
    \item There is some probability associated with forcing that superposition into one of the constituent states.
    \item There are certain observables, like $x$ and $y$ spin, which get in each other's ways when one tries to force a system out of the respective superpositions.
\end{itemize}
By the end of the next three chapters, the reason for these things will all be clear. 
\section{Conclusion}
This chapter is actually more of a pedagogical exercise than a strict reportage of phenomena. That is why we kept offering false hypotheses instead of the correct answer from the start. The point is that the reader should get a feeling of classical mechanics and their own intuition starting to fail, and the logical links we must make to try and offer explanations for these strange quantum phenomena. The physical set-up described in this chapter will, for the most part, be swept under the rug, but the ideas it has introduced to the reader will on the other hand be central. The most important two ideas, which the reader must remember, especially for the next chapter, are superposition of states, and inherent probabilities.
\\\\
These will provide good motivation for the mathematical innovations we see in the next chapter, which will begin our journey of quantum mechanics formally.