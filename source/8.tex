\chapter{Chapter 8: Continuous Spectra}
In this section, we take a huge leap forwards towards full physical reality by finally loosening the discrete spectrum restraint which we have been working under for the entire book so far. If we recall, eigenvalues of observable operators represent physically measurable valuables for those observables represented by those operators. Now, quantum mechanics does show us that some observables can be discretely distributed: most notably, energy as in the free particle solution. However, we also know that some observables must never be able to be discretely distributed. In particular, position being discretely distributed would be a huge issue in any physical model, because it would mean that us and any object moves by teleporting between discrete positions with increments between them which we somehow cannot move into. This is clearly nonsense; position must be continuous. Thus, we must also be prepared to deal with continuous spectra in quantum mechanics because, if nothing else, position eigenvalues must always be in a continuous spectrum. In the end, position is not the only observable which exhibits continuous spectra at all, but the illustration is already there.

\section{A Conceptual Definition of Continuity}
Clearly, as just established, some observables like position must take infinite continuously distributed eigenvalues. Now, if we take the eigenvalue definition:
$$
\Omega\ket{\omega}=\lambda\ket{\omega}
$$
then we can come to a clear realisation that we must also have infinite eigenkets as well. This is a result of the fact that the number of eigenvectors (eigenkets) must be greater than or equal to the number of eigenvalues. Otherwise, if there were fewer eigenvectors than eigenvalues, at least one eigenvector would correspond to multiple different eigenvalues. Then this would be:
$$
\Omega\ket{\omega}=\lambda\ket{\omega}
$$ 
and 
$$
\Omega\ket{\omega}=\lambda'\ket{\omega}
$$
but 
$$
\lambda\neq\lambda'\implies\lambda\ket{\omega}\neq\lambda'\ket{\omega}
$$
so 
$$
\Omega\ket{\omega}
$$
would simultaneously be two different kets, even though it is the same operator acting on the same input ket. Then, $\Omega$ simply could not be a linear operator, as it would have two outputs for a single input. 
\\\\
So now that we have established the number of eigenvectors is greater than (the degenerate case) or equal to (the non-degenerate case) the number of eigenvalues, and we know that some observable operators take infinite eigenvalues, we must have indeed infinite eigenvectors as well in such cases.
\\\\
Now, consider what practical function a eigenket has. It is an abstraction, so the answer is nothing in itself! The reason why an eigenket is important is because it is interpreted as the eigenstate where a corresponding eigenvalue has probability 1 of being measured. Thus, as already described in our section on inner labelling when we started with Dirac notation, we often just label eigenkets by that eigenvalue they are connected to. 
\\\\
Considering position eigenkets now, we would label them by 
$$
\ket{x_{0}}
$$
being the eigenstate where a measurement of position would yield position $x_{0}$ with certainty. We might also have for example
$$
\ket{0}, \ket{L} 
$$
if we were working bounded in some length $L$ where a position could not be further than distance $L$ away from the other end we labelled position $0$. Now, we would never write:
$$
\ket{L} > \ket{L/2} > \ket{0}
$$
because kets are abstract vectors and it makes no sense to compare their magnitudes with a `greater than' sign. Yet at the same time, that $\ket{L}>\ket{L/2}>\ket{0}$ would be easy to mistake as somewhat valid, because $L$ is a positive distance so indeed the values these kets represent themselves do satisfy $L>L/2>0$. We can see that inner label confusion can occur when it comes to magnitudes, because if we a label a ket by a specific value: usually an eigenket by an eigenvalue, we can be tempted to compare the magnitudes of those inner labels.
\\\\
What this pedantic clarification is to say is that
$$
\ket{L} > \ket{0}
$$
is mathematically void, but there is some concept of comparing the values labelling kets when it is a situation where values are labelling kets. Specifically, this means that when we write 
$$
\ket{x_{0}+dx},
$$
we can claim it is `infinitesimally close' to $\ket{x_{0}}$ as $dx\to0$. Now, that technically would not be correct syntax, considering kets cannot be infinitesimally close if they do not have magnitudes and therefore the definitions of `close' and `infinitesimal' do not exist. However, the idea is that $\ket{x_{0}+dx}$ and $\ket{x_{0}}$ may be infinitesimally close in that they represent eigenstates of positions which are infinitesimally close to each other! This is another case of the unfortunate semantic clarifications which result when we need to postulate so many abstractions in bijections, but at the same time, now that this is said, the reader should no longer scratch their heads when I or other texts say ``$\ket{x_{0}+dx}$ is infinitesimally close to $\ket{x_{0}}$'' rather than ``$\ket{x_{0}+dx}$ represents a position which is infinitesimally close to the position represented by $\ket{x_{0}}$''. Most compendia would not make such clarifications, but I think it is useful here.
\\\\
Now, then we can finally understand how continuity occurs in quantum mechanics and in the abstract state spaces we are working in. While kets cannot exhibit continuity, they can represent values exhibiting continuity, and this can be seen as one and the same.
\\\\
The mathematical definitions of continuity are:
\begin{itemize}
    \item A variable $Z$ is continuous over an interval $[a,b]$ if:
    $$
    \forall\stab z=Z\in[a,b], \stab dz\to 0, \mtab z+dz\in[a,b].
    $$
    This is probably already intuitive and not much more needs to be said: simply, there is a continuum of values which exist such that no matter how infinitesimal a scale we go down to, we will not be able to distinguish values by a discernible increment. 
    \item A function $f(x)$ is continuous over $x$ if $x$ is a continuous variable and
    $$
    \forall x, \lim_{dx\to 0}f(x+dx)=f(x).
    $$
    This means that there are no sudden jumps over infinitesimal intervals, and the graph of the function is \textit{smooth}.
\end{itemize}
\subsection{Infinities}
A discussion on continuity necessarily involves a thought about infinities. The issue herein is that the mathematics of infinities is rather inaccessible; in fact, all we primarily need to understand is this extremely simplistic idea of $\infty + 1 \ngtr \infty$. I want to stray as far from these sums including infinities and arithmetic numbers as I can, and this is achievable to us in quantum mechanics. Quite simply, the less thought here, the better.
\\\\
The one clarification I would like to make is on the comparison between discrete and continuous spectra in an infinite dimensional vector space.
\\\\
If a vector space is infinite dimensional, then, we know that the basis of that space-- that is, the set of linearly independent vectors in that space-- must have infinite cardinality (there are infinite basis vectors), by definition of dimensionality. The state space, we know, is an example of a vector space of infinite dimensions, and is critical to our study. Now, if we compare discretely distributed and continuously distributed eigenbases, we must make the clarification of how an infinite dimensional space can be spanned by both discrete bases and continuous bases. A reader might be surprised by this. If dimensionality is to do with the number of linearly independent basis vectors we can accommodate in the space, how could it be that one basis could be discrete and still have all the maximum number of basis vectors, while another basis is continuous? There is a sense, which is not unfounded, that the very point of continuous sets is that they have infinite elements between discrete points, and so they must have more elements than any discrete set.
\\\\
This idea is countered by the question ``How many integers are there?''. Of course, we know there are infinite integers-- that is, after all, the original point of having the entity infinity in the first place: if we keep on going through the consecutive positive integers we reach infinity. Can we say that there are more decimals than integers? In feeling, perhaps, but this is not mathematically rigorous. The idea is in fact of \textbf{countable and uncountable infinities}.
\\\\
A countable infinity is an infinity where one can hypothetically number each of those infinite elements. The positive integers are natural for demonstrate this, because they are already numbered- the integer 1 is the first element, the integer 2 the second, and so on, but yet they are definitely infinite in number. On the other hand, all decimal numbers constitute an uncountable infinity- we cannot number them since we could by definition always produce at least 1 decimal between the decimal we have numbered to be the first and the decimal we have numbered to be the second.
\\\\
This answers our question of how discrete and continuous basis can both span the same vector spaces. If a continuous basis spans a space, there must be infinite basis vectors, which is why the space is infinite dimensional. However, we could also span the space with a discrete basis, which has a \textit{countably} infinite number of basis vectors rather than an uncountably infinite one. A continuous basis cannot span a finite dimensional vector space, but a discrete basis can span both finite and infinite dimensional vector spaces depending on what bases we are considering. Discrete and finite are not synonyms, even though we might consider them to be. Thus what might appear confusing in regards to these discrete versus continuous bases is in fact answered by the idea that uncountable infinities are not larger in magnitude than countable infinities since both of them are infinite! Perhaps this section may be simplified, but quite frankly I am not interested in entering such fringe discussions which are extremely unimportant to our learning of quantum mechanics, and doubt the reader is either, so we can close this discussion.
\section{Continuous Wavefunctions}
Having a continuous case in our work will necessitate that some of our old summation and matrix notation seems defunct, because thinking of summing infinite terms or infinitely large matrices is not natural. However, the generalisation to infinite dimensions is in fact quite fine. Starting with infinitely sized matrices, We can consider the position basis for example to be column kets represented by 
$$
\ket{x_{i}}\duac
\begin{bmatrix}0 \\ 0 \\ \vdots \\ 1 \\ \vdots \\ 0 \\ \end{bmatrix}
$$
with the unity in the $i$'th row, $0$ elsewhere, and infinite rows. Clearly, they constitute an orthonormal basis and the actual number of rows of the vector is already not of much concern. 
\\\\
Meanwhile, in continuous cases, we can simply replace the sum terms we have been using with integrals-- infinite sums over continuous variables-- instead! This is an idea which generalises both in continuous cases and to infinities: the idea of an integral is that we are taking the sum of values separated by increments when that increment approaches $0$: thereby an infinite sum over infinitesimally different (continuous) values.
\\\\
For the inner product, we have thus far used sums for our multiplications in Dirac notation: we set up a column entity, which we called a ket, and a row entity, a bra, and we simply multiplied them together, but matrix multiplication in finite dimensions is an algorithm which is essentially a sum of $n$ values:
$$
\begin{bmatrix}
a_{1} & a_{2} & \dots & a_{n}
\end{bmatrix}\times \begin{bmatrix}
b_{1}\\
b_{2}\\
\vdots\\
b_{n}\\
\end{bmatrix}=\sum_{i=1}^{n}a_{i}b_{i}.
$$
The inner product we have been so far using in the Dirac notation sections is therefore the exact same idea, but with a conjugate transpose of column vector kets rather than just a transpose as the bras. That is, 
$$
\ip{\alpha}{\beta}=\sum_{i=1}^{n}a^{\ast}_{i}b_{i}
$$
with components $\setof{a_{i}}$ and $\setof{b_{i}}$ for $\ket{\alpha}$ and $\ket{\beta}$ respectively in an $n$-dimensional space.
\\\\
The problem we now face is that, while we wish to integrate kets in some way to create a continuous inner product, we cannot do this currently as kets are in no way related to any functions, which are our natural inputs for integrals-- kets are abstract vectors. However we have seen this problem of getting by an abstract ket: or, state vector, already! All we need to generate a function which exists in a bijection with state kets. We require this function to be continuous so we can integrate it.
\\\\
The function which makes sense, then is the same component function we worked with before
$$
\ket{\Psi}\to \psi(x).
$$
This should be governed by:
\begin{itemize}
    \item Input: Basis ket.
    \item Output: Component corresponding to the input basis ket.
    \item Domain: All kets in the basis.
    \item Range: Complex numbers (the components).
\end{itemize}
which changes with the input position $x$ and gives as an output the component corresponding to basis eigenket $\ket{x}$. 
\\\\
Well, this is the wavefunction, we know, because we have done exactly this already in the discrete case. Here, the argument $(x)$ represents that we are inputting position values, so this is the position wavefunction, which is only possible now because the position observable exhibits a continuous spectrum. One problem we do have with this is that, if we are trying to denote the value of the function evaluated at position $0$, for example, the form becomes
$$
\psi(0)
$$
 where it is now unclear whether or not it is a position wavefunction with position value $0$ or momentum wavefunction with momentum value $0$ or so on. Unfortunately, if we want to leave space for a subscript identifying separate wavefunctions apart from each other, for example in 
$$
\ket{\Psi_{1}}\duac\psi_{1}(x),\stab \ket{\Psi_{2}}\duac \psi_{2}(x)
$$
we have to sacrifice the ability to tell when the argument is absent which basis a wavefunction is expressed in. The other side of the coin is that first of all, context should always make it abundantly clear which basis everything is expressed in anyway, and secondly, reserving the same letter $\psi$ representing wavefunctions is more valuable than alternative possibilities such as identifying different observable wavefunctions with different letters, which would be just confusing.
\\\\
Now we can see easily that the position wavefunction is a continuous function since, 
$$
\forall x_{0}, \stab \lim_{dx\to 0} \psi(x_{0}+dx)=\psi(x_{0}).
$$
%# RAYS 
This intuitively makes sense, since the component corresponding to the position $x+dx$ should approach the component as we rotate to the direction corresponding to position $x$. A final convincing of this should be completed by the fact that the component relates the probability of measuring a certain position, by the measurement postulate: thus as we approach one position the component (probability) of measuring positions approaching that specific position should become progressively closer to the probability of measuring that specific position. Note also that we could have 
$$
\psi(p)
$$
which changes depending on input momentum-- assuming we are working in instances of continuous momentum, which do exist-- to give the corresponding component of the eigenket which is the eigenstate with that input momentum. This would simply be used if we were working in momentum space.
\\\\
Now, the components of a ket $\ket{\Psi}$ in a given basis $\setof{\ket{x}}$, which we will use since it is most natural, are given by 
$$
\ip{x_{i}}{\Psi}=c_{i}.
$$
Thus the position wavefunction is
$$
\psi(x):=\ip{x}{\Psi}.
$$
which is, the component of the ket $\ket{\Psi}$ in the direction of the basis vector $\ket{x}$.
\\\\
Now, we pause, because the above definition in ket form is prone to confusion, especially on passing glances, so it is important the next clarification is understood thoroughly before we move on.
\\\\
The reason the function definition we have just reached looks so absolutely wrong is because on the right hand side we have what appears exactly as an inner product, which we instantly associate (rightly) with a scalar constant, but on the left we have the position wavefunction we have just learnt about as a continuous function. What is very important here is that the inner label of the bra $\bra{x}$ in the inner product is meant to represent a varying index, rather than just an inner tag on a static bra obtained from a static ket. So we would have
$$
\psi(0)=\ip{0}{\Psi}
$$
which is, the component of the state $\ket{\Psi}$ corresponding to position $0$ which is represented by the ket $\ket{0}$ the corresponding bra of which is $\bra{0}$. We could also have 
$$
\psi(L)=\ip{L}{\Psi}
$$
which is, the component according to position $L$. 
\\\\
It is not wrong for the reader to find the above somewhat unusual, since a single inner product is always a scalar constant. It therefore might seem like a bit of an abuse of notation to write
$$
\ip{x}{\Psi}
$$
to be a varying quantity, since before our inner labels were stationary orders for kets and nothing more. Unfortunately, this is one point in quantum mechanics which convention has not seen fit to bother with changing notation to accommodate. The best way to deal with it is that in this book I will always use $\ket{x}$ to represent a variable position ket, $\ket{p}$ a variable momentum ket and $\ket{E}$ a variable energy ket. Otherwise, for static single position kets I will use notation such as $\ket{x_{0}}$ or $\ket{x_{1}}$ to denote these fixed points -- usually, just $x_{0}$ if we only require one fixed point to compare to, as this notation is quite common in quantum mechanics. Now we can fully enunciate the action of this wavefunction is through the following steps:
\begin{enumerate}
    \item Take the input observable value.
    \item Find the eigenket corresponding to that observable value, which is a pure eigenstate with probability 1 of obtaining that value.
    \item Find the inner product product of that eigenket with the state ket.
    \item Output that inner product value.
\end{enumerate}
\section{Dirac Delta Function}
That notational liberty of variable kets is aesthetically cumbersome, so we should demand it to be at least useful to us. It is useful, in fact, since we can finally define the integral for our abstract kets and bras:
$$
\ip{\Psi_{1}}{\Psi_{2}}=\int_{-\infty}^{\infty}\psi_{1}^{\ast}(x)\psi_{2}(x)\,dx .
$$
Finally, we have the continuous inner product! We indeed see that the above is very much related to the discrete inner product, since instead of discrete components we now have a continuous wavefunction storing the components to deal with the fact there are infinitely many continuous eigenkets, and finally we integrate just as an integral is a continuous sum. The same varying input variable $x$ ensures that the coefficients are matched up for both states $\ket{\Psi_{1}}$ and $\ket{\Psi_{2}}$ since each time the functions $\psi_{1}(x)$ and $\psi_{2}(x)$ give the components corresponding to the eigenket representing the inputted position, by their definition.
\\\\
The integral bounds can become definite, and often do, in physical problems. An example of this fact might be if we are considering position functions on a string-- in which case the variable $x$ might be considered, and the bounds might be $0$ to $L$-- the other end of the string of length $L$ from the end we defined to be position $0$. It is clear that in this case it does not make sense to take an integral from $0$ to $70000$km displaced from $0$, for example, if a string is say $50$cm long, as we do not expect any particles we consider to be any further than $50$cm (in the one dimension we are considering) from the point on the other end which we defined to be position $0$. This is some sort of way to saying that we can ignore the bounds of the integrals in continuous quantum mechanics until a physical problem sets them for us.
\\\\
With this given, the orthogonal condition is the same: two kets $\ket{x_{0}}$ and $\ket{x_{1}}$ (which, by the aforementioned convention we are using, represent static kets corresponding to positions $x_{0}$ and $x_{1}$ respectively) are orthogonal if 
$$
\ip{x_{0}}{x_{1}}=0.
$$
However, the normalisation condition is different. To understand this, consider the completeness relation, which we expect to be:
$$
\infint \sop{x}\,dx=1
$$
where we have continued the convention of $x$ without any subscripts representing a variable inner label. Left multiplying by $\bra{x_{0}}$ and right multiplying by any non-basis ket $\ket{\alpha}$ gives us
$$
\infint dx\,\ip{x_{0}}{x}\ip{x}{\alpha}=\ip{x_{0}}{\alpha}
$$
which is, the component $\psi_{\alpha}(x_{0})$ of the ket $\ket{\alpha}$in the direction of eigenket $\ket{x_{0}}$ corresponding to position value $x_{0}$.
\\\\
Denote now the inner product between basis vectors $\ket{x}$ and $\ket{x_{0}}$ as $$\ip{x}{x_{0}}:=
\delta(x-x_{0}).$$ The way of presenting the two arguments $x$ and $x_{0}$, which are, the inputs,
of the function should not throw one off: it is convention, which has a not so significant but still very much logical reason. We know that 
$$
\forall x\neq x_{0}, \btab \delta(x-x_{0})=0 
$$
since the position kets are all orthogonal to each other as they are non-degenerate eigenkets of a hermitian operator. Now, we can look at the above expression again:
$$
\infint \ip{x_{0}}{x}\ip{x}{\alpha}\,dx=\ip{x_{0}}{\alpha}\implies \infint \delta(x_{0}-x)\psi_{\alpha}(x)\,dx=\psi_{\alpha}(x_{0})
$$
The range over negative infinity to positive infinity seems expansive, but in fact, an infinite part of it is redundant; this is because the inner product $\delta(x-x_{0})$ is 0 for all $x\neq x_{0}$! Now it is clear we need to continue to consider infinities since we are working with continuous variables rather than discrete ones. Consider an infinitely small region $[x_{0}-\Delta x,x_{0}+\Delta x]$ for an infinitesimal difference $\Delta x$, which centres at $x_{0}$. In this region we can consider the integral because it is only here where we can consider $\delta(x-x_{0})$ to not definitively be $0$ since if $x$ is infinitely close to $x_{0}$ we cannot just easily say $x\neq x_{0}$. With these new bounds, 
$$
\int_{x_{0}-\Delta x}^{x_{0}+\Delta x}\delta(x-x_{0})\psi_{\alpha}(x)=\psi_{\alpha}(x_{0})
$$
since this region is the only area where we cannot say the inner product $\delta(x-x_{0})$ is $0$. As all values of $x$ in the integral lie in this infinitesimal region, we can as we assume in the limit that the components $\psi_{\alpha}(x)=\psi_{\alpha}(x_{0})$ since the input direction $\ket{x}$ is infinitesimally different from the fixed direction $\ket{x_{0}}$. This then is a specific value (not a function, despite the notation), so we can pull it out as a constant:
$$
\int_{x_{0}-\Delta x}^{x_{0}+\Delta x}\delta(x-x_{0})\psi_{\alpha}(x)\,dx=\psi_{\alpha}(x_{0})\int_{x_{0}-\Delta x}^{x_{0}+\Delta x}\delta(x-x_{0})\,dx=\psi_{\alpha}(x_{0}).
$$
That is, 
$$
\int_{x_{0}-\Delta x}^{x_{0}+\Delta x}\delta(x-x_{0})\,dx_{0}=1.
$$
What implications does this have for the inner product $\delta(x-x_{0})$, which we call the (Dirac) delta function? The most intuitive answer comes from using the conventional visualisation of an integral as a way to measure the area under a smooth function. We know the delta function $\delta(x-x_{0})$ is $0$ until it reaches an infinitely small interval around the value $x_{0}$. Yet the integral of the whole function with respect to $x$ is $1$. So if we draw a horizontal axis for varying $x$ and a vertical axis for the value of $\delta(x-x_{0})$ (with $x_{0}$ fixed), we will get a flat line for all infinity until we get infinitely close to $x_{0}$. Yet, this as a whole must have area $1$! So the infinitely small width interval close to $x$ is the only region which contributes any area to the integral, and this whole area is $1$. So, in the picture we have created, the value $\delta(x-x_{0})$ in this region is the height which contributes $1$ to the area despite having an infinitely small width. The only explanation therefore is that at the infinitesimally small region, the delta function has infinite value. Otherwise, the infinitely small width interval could not have any area which is not infinitely small! Then the infinitesimally small domain for $x$ around $x_{0}$ can simply be reduced to $x=x_{0}$, so $\delta(x_{0}-x_{0}):=\delta(0)$ is infinity!
\\\\
We summarise with the following:
$$
\delta(x-x_{0})=
\begin{cases}
0\stab\text{if}\stab x\neq x_{0}\\
\infty\stab\text{if}\stab x=x_{0}\\
\end{cases}
$$
Of course, we do not like writing the infinity symbol as if it is a value very often in mathematics, so this is better put
$$
$$
$$
\delta(x-x_{0})=
0\stab\text{if}\stab x\neq x_{0}, \btab\btab
\int\delta(x-x_{0})\,dx=1
$$
where the bounds of integration do not matter since the function is $0$ anyway until we get infinitesimally close to $x_{0}$. This will be commonplace any time we consider a continuous basis. Another way to summarise this is also in the framework of viewing the delta function as a \textit{sampling function}, which means that 
$$
\int\delta(x-x_{0})f(x)\,dx=f(x_{0})
$$
That is- because the range where $x$ and $x_{0}$ are completely distinct vanishes, the integral only selects the value of $f(x_{0})$, which changes over a continuously varying $x_{0}$, which is the same as that at point $x$. Finally, it is crucial to know that the delta function is real! This means that
$$
\delta(x-x_{0})=\ip{x}{x_{0}}=\ip{x_{0}}{x}^{\ast}=\delta(x_{0}-x)^{\ast}
$$
but as the delta function is real, 
$$
\dd{x_{0}}{x}^{\ast}=\dd{x_{0}}{x}
$$
which altogether means that 
$$
\dd{x}{x_{0}}=\dd{x_{0}}{x}
$$
-an important point, of course. It is also very important to know that the minus sign in the function is meant to be taken literally (technically, the Delta function is a function of the difference between its two arguments, but this point is not important at all). So we will often see expressions like 
$$
\delta(x)=\dd{x}{0}=\dd{0}{x}
$$
or 
$$
\delta(0)=\dd{x}{x}
$$
and even 
$$
\delta(k)=\dd{x+k}{x}
$$
in shorthand. A reader should not get confused by this, and should also remember that 
$$
\delta(k)=\delta(-k)
$$
since $\delta(x+k-x)=\delta(x-(x+k))$.
\\\\
\section{Position and Momentum}
In order to understand the importance of orthonormalising to the Dirac delta function as the continuous analogue of orthonormalising to the Kronecker delta in the discrete case we have beforehand been working on, we will undergo the necessary algebraic steps to work with the two canonical quantum mechanical operators. Through this, one can practice both the algebra and rationale of the Dirac delta function.
\subsection{Position and Momentum Operators}
The first step we can take using the Dirac Delta function is to confirm the action of the position operator-- the operator in quantum mechanics which represents the physical variable of position-- in \textbf{position space}, the space spanned by the continuous position eigenkets. Take some function $f(x):=\ip{x}{f}$ represented by the ket $\ket{f}$ and define
$$
\hat{X}\ket{f}:=\ket{F}
$$
with 
$$
F(x_{0}):=\ip{x_{0}}{F}.
$$
Now in the usual way we can determine the action of an operator on a ket by left multiplying it by a basis bra:
$$
\optrip{x_{0}}{X}{f}=\ip{x_{0}}{F}
$$
We can then insert the continuous completeness relation:
$$
\int\sop{x}\,dx=I\implies\optrip{x_{0}}{X}{f}\equiv \int\optrip{x_{0}}{X}{x}\ip{x}{f}\,dx=\int x\ip{x_{0}}{x}\ip{x}{f}\,dx.
$$
In Delta notation, this is 
$$
\int\optrip{x_{0}}{X}{x}\ip{x}{f}\,dx=\int x\dd{x}{x_{0}}f(x)\,dx.
$$
As per usual, the delta function vanishes all the integrated terms except for when $x=x_{0}$. Therefore we can assume the $x$ variable is only relevant when $x=x_{0}$ and thus pull it out as $x_{0}$. To finalise the result, we simply get the sampling property of the delta function.
$$
\int x\dd{x}{x_{0}}f(x)\,dx\equiv x_{0}\int\dd{x}{x_{0}}f(x)\,dx = x_{0}f(x_{0}).
$$
So we have $$
\optrip{x_{0}}{X}{f}=\ip{x_{0}}{F}
=F(x_{0})=x_{0}f(x_{0})
$$
So
$$
\hat{X}f(x_{0})=F(x_{0})=x_{0}f(x_{0}).
$$
Since this holds true for all $x_{0}$, we can generalise this for any input of the position variable $x$ as simply:
$$
\hat{X}f(x)=xf(x)
$$
in the position basis; indeed, this is the position operator we are familiar with. The matrix elements of $\hat{X}$ in the position basis are trivial:
$$
\optrip{x_{1}}{X}{x_{0}}=x_{0}\ip{x_{1}}{x_{0}}=x_{0}\dd{x_{1}}{x_{0}}.
$$
\\\\
The position and momentum operators in any basis are related solely by the commutation relation
$$
[\hat{X},\hat{P}]=i\hbar.
$$
The choice one makes for the momentum operator in position space is therefore 
$$
\hat{P}:=-i\hbar\nd{}{x}.
$$
Solving its eigenvalue problem in the continuous position space will prove a more challenging task, as it contains a derivative. We start by considering its action on a function in the position basis:
$$
\optrip{x_{0}}{P}{f}\equiv -i\hbar\bip{x_{0}}{\nd{f}{x}}=-i\hbar f'(x)
$$
where $f'(x)$ is the component in the $x$ direction of the derivative of $f$ with respect to $x$. 
%Hmm, this line rather shows up the problem with the prime notation. I think you need to state very explicitly here that the two primes mean completely different things. Either that, or switch to $x_1$, $x_2$ or similar.
Expanding again with the completeness relation, we also have the alternate equations
$$
\optrip{x_{0}}{P}{f}=\int\optrip{x_{0}}{P}{x}\ip{x}{f}\,dx=\int\optrip{x_{0}}{P}{x}f(x)\,dx
$$
So this means that
$$
-i\hbar f'(x)=\optrip{x_{0}}{P}{f}=\int\optrip{x_{0}}{P}{x}f(x)\,dx.
$$
The form we expect is that 
$$
\optrip{x_{0}}{P}{x}=-i\hbar\dd{x}{x_{0}}\nd{}{x}.
$$
Indeed, this form works with the delta function. We get:
$$
\optrip{x_{0}}{P}{f}=\int-i\hbar\dd{x}{x_{0}}\nd{}{x}f(x)\,dx=-i\hbar\int\dd{x}{x_{0}}f'(x)\,dx
$$
and this is, by the sampling property,
$$
-i\hbar f'(x)\biggl\vert_{x=x_{0}}=-i\hbar f'(x_{0})
$$
so we get 
$$
\optrip{x_{0}}{P}{f}=\hat{P}f(x_{0})=-i\hbar f'(x_{0})
$$
as required. So we have the matrix element 
$$
\optrip{x_{0}}{P}{x}=-i\hbar\dd{x}{x_{0}}\nd{}{x}.
$$
Interestingly, this is in fact 
$$
\optrip{x_{0}}{P}{x}=-i\hbar\delta'(x-x_{0}),
$$
that is, $-i\hbar$ multiplied by the derivative of $\dd{x}{x_{0}}$ with respect to its first argument (which here is $x$)! The rule for the derivatives of Delta functions, which are too mysterious and challenging to warrant their own proofs for now but which must be stated, is that 
$$
\frac{d^{(n)}\dd{x}{x_{0}}}{dx^{(n)}}=\delta'^{(n)}(x-x_{0})=\dd{x}{x_{0}}\frac{d^{(n)}}{dx^{(n)}}
$$
where the exponent $n$ represents the order of the derivative. Do note that the derivative is with respect to $x$ here because here $x$ is the first argument of the delta function and the argument which is variable. 
\\\\
It is equally important to know that a very different definition for the operators and their matrix elements occurs if we are in a different basis- for example, the momentum space. Our original definition in Postulate 6 was that the position operator was $\hat{X}=x$ and the momentum operator was $\hat{P}=-i\hbar\nd{}{x}$, but this only holds true in position space. Position space is generally the space used, as aforementioned in chapter 5, as it is in physical reality, and it usually is easier to express keep the potential $V(x)$, a function of position, in the position basis than it is to find its form in momentum space. Nevertheless, momentum space can also for example be considered.
\\\\
Now observe the way in which we derived the action of the position operator in position space. It should be convincing to the reader that if we were to replace the position operator with the momentum operator, and multiply it by a momentum eigenbra, and use the completeness relation but with momentum eigenkets, we will observe the exact same result! Nothing at all is contributed by the fact that it was the position operator we were discussing, other than that it was the position operator whose eigenbasis was spanning the space above while we were trying to express its action on constituent kets. Therefore, if we consider any other continuous operator in a space spanned by its orthonormal basis, we should obtain the exact same result-- but simply expressed for a different physical variable!
\\\\
If one is convinced that we \textit{derived} the action of the position operator in its own space simply by using continuous relations and manipulation with its own eigenkets, they should then be convinced that in momentum space we have:
$$
\hat{P}f(p)=pf(p).
$$
Its matrix elements in this basis are easy as well:
$$
\optrip{p'}{P}{p}=p\ip{p'}{p}=p\dd{p'}{p}.
$$
\subsection{Position and Momentum eigenfunctions}
Fruitful (but difficult) discussions come out of considering the eigenfunctions of position and momentum. We will work in the position space, as that is sufficient to provide the necessary discourse and is the most common space to work in.
\\\\
One might assume that the eigenvalues of $\hat{X}$ in its own space are natural.
We can denote an eigenvector as $\xi(x)$ and the corresponding eigenvalue as $x_{0}$. Then we have
$$
\hat{X}\xi(x)=x\xi(x)=x_{0}\xi(x).
$$
Consider this now carefully. We have 
$$
x\xi(x)=x_{0}\xi(x)
$$
where $x$ is a continuous variable and $x_{0}$ is a single constant eigenvalue! Clearly something is wrong. It is impossible for a single eigenvalue to be multiple values of $x$ at the same time; if it was a function instead, it would not be viable as an eigenvalue and therefore $\xi(x)$ would also not be an eigenvector.
\\\\
It turns out this problem is fixed if we make the eigenvector $\xi(x)$ a special type of function. Of course, by now one should anticipate this is the Dirac delta function (we can see how useful it is), with arguments $x_{0}$ and $x$. That way, if we have
$$
\hat{X}\dd{x_{0}}{x}=x\dd{x_{0}}{x}=x_{0}\dd{x_{0}}{x}
$$
we will see that the function does the work for us! We get a crude $0=0$ for all the values of continuous variable $x$ which are not $x_{0}$, and then for $x=x_{0}$ clearly the eigenvalue and eigenvector conditions are satisfied. We will rewrite this as 
$$
\hat{X}\dd{x}{x_{0}}=x_{0}\dd{x}{x_{0}}
$$
as it is more conventional to write $\dd{x}{x_{0}}$ than $\dd{x_{0}}{x}$, even though they are the same thing. 
\\\\
The momentum eigenvalue problem in position space is more important, and difficult. The problem is
$$
\hat{P}\phi(x)=-i\hbar\nd{}{x}\phi(x)=p\phi(x)
$$
for an arbitrary eigenvector $\phi(x)$ corresponding to eigenvalue $p$. We can tell that immediately the problem we had with the position eigenvectors in position space isn't going to be prominent here simply due to the form of the operator; indeed, we will not need the delta function. Instead, what one finds is that this is very simply a recognisable first order differential equation.
$$
\begin{aligned}
p\phi(x)&=-i\hbar\nd{}{x}\phi(x)\\
\Rightarrow\stab\nd{\phi}{x}&=ip/\hbar\phi(x)
\end{aligned}
$$
and this is an equation of the form 
$$
\nd{y}{x}=ky
$$
with general solution $y=Ae^{kx}$. So matching these two equations we have the comparisons $y=\phi(x)$, $x=x$ and $k=ip/\hbar$. Therefore, the solution to the eigenvector problem is 
$$
\phi(x)=Ae^{ipx/\hbar}.
$$
for some arbitrary constant $A$. It should be noted that $i,\hbar$ are both constants, and $p$ is the eigenvalue corresponding to the eigenvector, which is not a single constant unless we are only concerned with a single eigenvector. Therefore, $x$ is the only variable which changes the value of the position space momentum eigenfunction as it is itself varied, which is why we can call it a function of the position variable $x$. Now, it might seem like our job is done in this section, but the form of the momentum eigenvectors bring significant imperfections to the Hilbert space we have so far seen as quite perfect for our job of describing physical reality. We must deal with these in any valid discussion of quantum mechanics.
\\\\
As mentioned in the section on infinities, an exponential function rises exponentially to infinity as $x$ goes to infinity (when its exponent is positive, as it is here). The reader should therefore realise that, no matter what constant $A$ we try to multiply it with, we will never be able to gain a finite integral over all space and therefore position $x$. As there is no normalisation which produces a finite norm for a momentum eigenstate, we can only do the next best thing, which is try to  normalise it in a way which encapsulates this infinite behaviour while being still mainly serviceable for the needs of algebraic manipulations. Of course, this can be best done through the Dirac delta function. We have seen how useful it is, in that its derivatives are defined, it has secondary properties like sampling which greatly simplify algebra with it, and it is very commonly seen across what we have already done. Thus, for such non-finite norm vectors we want to use the process of \textit{normalising to the Dirac delta function}, the continuous analogue of normalising to unity!
\\\\
In the case of momentum eigenvectors, a constant $A$ is already set up in the general solution. So we can try to find a constant $A$ which normalises the momentum eigenvectors to the Dirac delta function. In arbitrary terms, we have, still in the position basis,
$$
\ip{p}{p'}=\int\ip{p}{x}\ip{x}{p'}\,dx
$$
and we would like $\ip{p}{p'}$ to be equal to $\dd{p}{p'}$ after we have modified $\ket{p}$ (we can use the variable label $p$ since it should hold for all eigenkets) by multiplying it by some constant. Currently the above is 
$$
\begin{aligned}
\ip{p}{p'}&=\int\ip{p}{x}\ip{x}{p'}\,dx =\int\ip{x}{p}^{\ast}\ip{x}{p'}\,dx\\
&=\int\psi_{p}^{\ast}(x)\psi_{p'}(x)\,dx = \int A^{\ast}e^{-ipx/\hbar}Ae^{ip'x/\hbar}\,dx\\
&=|A|^{2}\int e^{-ix(p-p')/\hbar}\,dx.
\end{aligned}
$$
Unfortunately, at this point we cannot use any mathematics we know or will be able to quickly learn to move forwards, as the answer lies in a relationship given by Fourier transforms, which are too advanced for this book. Therefore a relationship will just have to be stated and taken as given. This is the relationship
$$
\frac{1}{2\pi}\int_{-\infty}^{\infty}e^{-ik(x-x_{0})}\,dk=\delta(x-x_{0}).
$$
It looks haphazard, but at the heart of it lies the connections between complex numbers, exponentials and pi, of which an example is the Euler formula. Any further explanation would be meaningless without the reader understanding Fourier transforms (where from the relationship is derived) in the first place, and that is hardly assumed knowledge here.
\\\\
This is reasonably closely related to our momentum eigenstate normalisation question, with a bridge step. We had:
$$
|A|^{2}\int e^{-ix(p-p')/\hbar}\,dx
$$
for some normalisation constant $A$ we are trying to find. We can rewrite this as 
$$
|A|^{2}\int e^{-i(\frac{p}{\hbar}-\frac{p'}{\hbar})x}\,dx,
$$
which is extremely close to the expression above. The only lemma we need to is that $\delta(\frac{x}{a})=a\delta(x)$ for some constant $a$. Now we can use the relations to get 
$$
|A|^{2}\int e^{-ix(p-p')/\hbar}\,dx=\frac{|A|^2}{2\pi}\delta\left(\frac{1}{\hbar}(p-p')\right)
$$
and by the lemma this is
$$
\frac{|A|^2\hbar}{2\pi}\delta(p-p').
$$
Finally, we can see what constant we want $A$ to be! Simply $A=\frac{1}{\sqrt{2\pi\hbar}}$ will work, and, retracing our steps and replacing $A$ with $(2\pi\hbar)^{-1/2}$ we will get
$$
\ket{p}\duac\frac{1}{\sqrt{2\pi\hbar}}e^{ipx/\hbar}\implies\ip{p}{p'}=\delta(p-p').
$$
This has taken a lot of labour and a couple of steps we can only take for granted without more advanced mathematical tools: but one should now be able to fully understand that orthonormalising to the Dirac delta function is the continuous analogue of orthonormalising to the Kronecker delta in the discrete case, and that both are necessary, possible, and useful.
\section{Probability Distribution Functions}
The state problem was said at its introduction in this book to be a problem of encapsulation and extraction. The question of extraction is answered by observable operators and their eigenbases, and elaborated on by the Compatibility Theorem and Heisenberg Uncertainty Principle, which result from those observable and measurement postulates. The question of encapsulation has been already largely answered by state vectors, but both elegance and historical justice can be demonstrated with the following formalisation, which will show that the continuous wavefunction can be interpreted as a probability distribution function which therefore encapsulates not only all the possible measurements, but also their probabilities. This is naturally a direct mirror image of the probability mass function we obtained for the discrete case wavefunction. After that, the only difficulty-- though it will prove itself to be much more diverse and challenging than any other-- would be to solve for the state vector in some basis given the conditions of a physical problem we are given.
\\\\
We know how to extract probabilities from our representation of a state:
$$
P(\alpha_{i})=|(\alpha_{i},\Psi)|^{2}
$$
if $\alpha_{i}$ is some orthonormal eigenvector. It is important to remember that there are bases of the state space which do not consist of eigenvectors of some operator, but since we have already seen that we are exclusively interested in observable eigenbases when we work in quantum mechanics, we might often write eigenvector instead of vector with the implicit assumption that the basis vector we are using would be an eigenbasis vector. We also know that 
$$
(\alpha_{i},\Psi)=c_{i},
$$
the component of the state vector in the orthonormal basis $\setof{\alpha_{i}}$. Therefore the components of the state vector in some observable basis are the links to probabilities in the state problem. We call them \textbf{probability amplitudes}, and the modulus squared of these probability amplitudes are called \textbf{probability densities}. If we work in discrete cases, probability densities, called probability masses in the discrete case, are simply synonymous with probabilities, as the postulate shows. Very minor differences exist in the continuous case, which we now have the correct definitions to tackle as well.
\\\\
The generalisation to continuous dimensions of probability mass functions are well covered in mathematics already. The problem we have is that continuity implies infinite eigenstates, and therefore, from the perspective of probability, infinite events. If there are infinite events, then all of them must have technically have $0$ probability because we cannot pin a probability down to a single value when there is a value which corresponds to shifting that value by an infinitely small increment.
\\\\
Again, this is pure mathematics rather than quantum mechanics speaking. It is convention to call the probability distribution functions \textbf{probability density functions} in the continuous case. Since it is a continuous function, we cannot take a certain value for a probability as discussed, but we can integrate it between certain bounds!
\\\\
The integral 
$$
\int_{x=a}^{x=b}|\psi(x)|^2\,dx
$$
with a continuous wavefunction, here the position wavefunction, is the probability that the state is measured to be somewhere between the eigenstates $\ket{a}$ and $\ket{b}$ corresponding to position values $a$ and $b$. Thus we have a way to find the probabilities we need now for the continuous case as well, and see why wavefunctions are probability density functions with respect to certain measurements of observables.
\\\\
The state problem, which we left as finished for the discrete case, is now complete for the continuous case as well. The physical state corresponds to the state vector, which can be transformed into wavefunctions by eigenvector inner products. The resulting functions are either probability mass functions, in the discrete case, or probability distribution functions, in the continuous case. Finally, I am justified in calling wavefunctions `tangible' when I sounded strange beforehand. More importantly, we are done with all the conceptual learning we will be covering in this book!
\section{Exercises from Chapter 7}
\begin{enumerate}
    \item 
    \item
    \item
    \item
    \item
    \item
    \item
    \item
    \item
    \item
\end{enumerate}