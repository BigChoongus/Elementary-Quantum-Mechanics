\chapter{Problems in Quantum Mechanics}
This chapter, and the next, will practice all the concepts we have learnt with physical problems. The algebraic labour involved will be extensive, but how could it be otherwise? I have kept all the solutions in Dirac notation, because this is conventional and because manipulation with Dirac notation is more powerful and fruitful. 
\section{The Propagator}
We have mentioned how one can use energy eigenstates to conveniently find solutions for the Schr\"{o}dinger equation. However, beyond this algebraic method there is another method, the method of the propagator, which is used ubiquitously by those more advanced in quantum mechanics. In this book, it might seem even at the end counterintuitive to have learnt both the energy eigenstate method and the propagator method, but the crucial point is that for more difficult problems only the latter is viable; while we might not have the full mathematical tools to realise how useful it is, no mature treatment of quantum mechanics is complete without it, and we can still use it for the simpler problems we cover even if its formulation might seem to contribute nothing for these basic examples. It is also, mathematics aside, a conceptually valuable thing to study even at an expository level.
\\\\
At time $t_{0}$ the state must be able to be represented by a ket in the ket space. At any other time $t$ the state must also be able to be represented by a ket in the ket space. One therefore might be led to consider whether there is an operator which governs the mapping of one state ket to another state ket given some input time $t-t_{0}$ which has passed.
\\\\
There is, and it is called the propagator (or time evolution operator). More importantly, the operator turns out to be the same over all time \textbf{given the same Hamiltonian} (i.e, given the system is not perturbed). We can try to formulate it, as we know some conditions we expect to be fulfilled.
\\\\
Denote the operator $\prop{t}{t_{0}}$ to be the propagator which carries a state ket from time $t_{0}$ to a ket at time $t$. The first condition we expect the propagator to satisfy is the property of composition:
$$
\forall\: t_{0}\leq t_{1} \leq t_{2}, \btab \prop{t_{2}}{t_{0}}=\prop{t_{2}}{t_{1}}\prop{t_{1}}{t_{0}}.
$$
This is because the propagator of a system carrying a ket from $t_{0}$ to $t_{1}$ to $t_{2}$ should be an equivalent transformation to carrying a ket from $t_{0}$ to $t_{2}$ directly since the start and end kets are the same.
\\\\
The second condition is subtle but important. Take a ket $\ket{\Psi,{t_{0}}}$, We expect that for a proper physical ket it is normalised, so that the total sum of probabilities is always $1$ for any measurements of a given observable. Now, if we consider
$$
\prop{t}{t_{0}}\ket{\Psi,{t_{0}}}:=\ket{\Psi,{t}},
$$
then the norm at time $t$ of the state ket becomes
$$
\ip{\Psi,{t}}{\Psi,{t}}= \bra{\Psi,{t_{0}}}\:{\propdagg{t}{t_{0}}\prop{t}{t_{0}}}\:\ket{\Psi,{t_{0}}}.
$$
We expect that the ket at times IS still normalised, or the sums of probabilities for measurements will not equal $1$.
%There something funny going on in the last sentence.
Therefore we have 
$$
\sip{\Psi,t_{0}}=\sip{\Psi,t}=1=\bra{\Psi,{t_{0}}}\:{\propdagg{t}{t_{0}}\prop{t}{t_{0}}}\:\ket{\Psi,{t_{0}}}.
$$
This means we must have:
$$
{\propdagg{t}{t_{0}}\prop{t}{t_{0}}}=1.
$$
Otherwise stated, the propagator is unitary! In fact this is a requirement for operators which map a physical state onto another physical state, due to the necessity of \textbf{conservation of probability}-- really, conservation of normalisation--, which unitary operators always fulfil because they preserve the norm of any ket they act on. 
\\\\
Finally, we expect that 
$$
\lim_{dt\to0}\prop{t_{0}+dt}{t_{0}}=1
$$
(the identity operator), due to the continuity of time. Now, we do know that the Schr\"{o}dinger Equation must still apply- in other words, we know that
$$
i\hbar\frac{\partial}{\partial t}\ket{\Psi,{t}}=\hat{H}\ket{\Psi,{t}}.
$$
Assume the state $\ket{\Psi,t_{0}}$ was a precedent state to $\ket{\Psi,t}$. We therefore have
$$
i\hbar\frac{\partial}{\partial t}\prop{t}{t_{0}}\ket{\Psi,{t_{0}}}=\hat{H}\prop{t}{t_{0}}\ket{\Psi,{t_{0}}}
$$
so we can equate the two operators since it holds true for any $\ket{\Psi,t_{0}}$:
$$
i\hbar\frac{\partial}{\partial t}\prop{t}{t_{0}}=\hat{H}\prop{t}{t_{0}}.
$$
This is called the Schr\"{o}dinger Equation for the time evolution operator.
\\\\
We would like there to be a very simple way to find the propagator, because having an explicit representation of the propagator is sufficient to solving the time evolution problem (we can apply it to any initial state $\ket{\Psi,t_{0}}$). We can try to induce its form through looking at equations we know must hold, since we understand its usage and what properties it must have.
\\\\
The reader might be relieved to know our starting point is the energy eigenfunction method of solving the Schr\"{o}dinger Equation, which is not incorrect at all (we derived it very soundly from the Sch\"{o}dinger Equation, and the reader may go back to review that proof if they need do so before continuing) and need not be discarded. In our function formulation, we had 
$$
\psi_{t}^{(n)}:=e^{-iE_{n}t/\hbar}\varepsilon_{n}
$$
as the stationary states of any given system- in other words, where time evolution does not change the states at all. In our Dirac notation this is
$$
\ket{\Psi_{n},t}=e^{-iE_{n}t/\hbar}\ket{E_{n}}
$$
where we have labelled kets by their eigenvalues as is customary. The derivation 
$$
\Psi_{t}=\sum_{n}(\varepsilon_{n},\Psi_{0})\psi_{t}^{(n)}
$$
was made where $\Psi_{t}$ is any arbitrary state at time $t$ and $\Psi_{0}$ was the state at time $0$. We translate this to Dirac notation as well:
$$
\ket{\Psi, t}=\sum_{n}\ket{\Psi_{n}}\ip{E_{n}}{\Psi,0}=\sum_{n}e^{-iE_{n}t/\hbar}\ket{E_{n}}\ip{E_{n}}{\Psi,0}.
$$
However, this allows us to induce the form of the propagator:
$$
\ket{\Psi, t}=\sum_{n}e^{-iE_{n}t/\hbar}\ket{E_{n}}\ip{E_{n}}{\Psi,0}=\biggl(\sum_{n}e^{-iE_{n}t/\hbar}\ket{E_{n}}\bra{E_{n}}\biggr)\ket{\Psi,0}
$$
and 
$$
\ket{\Psi,t}=\prop{t}{0}\ket{\Psi,0}
$$
so therefore
$$
\prop{t}{0}=\sum_{n}e^{-iE_{n}t/\hbar}\ket{E_{n}}\bra{E_{n}}\stab.
$$
This form of the propagator of course applies when the energy eigenkets are discrete: a phenomenon which does often occur in problems, as we will see shortly. It should be noted that $e^{-iE_{n}t/\hbar}$ is not a constant, as its value depends on the eigenvalue $E_{n}$ which varies over the different indexes $n$, so trying to pull it out of the sum term is invalid. Furthermore, the continuous analog (for continuous energies, which also occur) should be equally clear, as we can simply take an integral:
$$
\prop{t}{0}=\infint e^{-iE_{n}t/\hbar}\ket{E_{n}}\bra{E_{n}}\,dn.
$$
In reality, using $n$ as an index for a continuously varying entity, and integrating with respect to that contrived $n$, is somewhat hideous. However, the idea is that we integrate over the changing values of $E_{n}$, which correspond to the labelled kets $\ket{E_{n}}$. Finally, we will not consider time varying Hamiltonians as they are extremely difficult.
\subsection{The Free Particle Propagator}
We can now formulate the propagator for the free particle, whose solution we studied without a propagator already in chapter 5. The energy eigenkets of the Hamiltonian, which commutes with the momentum operator, are of the form
$$
\ket{E_{n};+}=\ket{p=\sqrt{2mE_{n}}\:}
$$
and
$$
\ket{E_{n};-}=\ket{p=-\sqrt{2mE_{n}}\:}.
$$
The $+$ and $-$ inner labels summarise the fact that these energy eigenvalues are degenerate: they apply to two distinct eigenkets- but with different momenta corresponding to them (the reader should probably revise the section on the Free Particle to recall this). We therefore label the eigenkets by their nondegenerate momenta instead of using the natural $E$ label. This means for the propagator:
$$
\prop{t}{0}=\infint e^{-iE_{n}t/\hbar}\ket{E_{n}}\bra{E_{n}}\,dn
$$
we can write it for the free particle as 
$$
\prop{t}{0}=\infint e^{-ip^{2}t/2m\hbar}\ket{p}\bra{p}\,dp.
$$
Note that, while we before were unscrupulously using an index $n$ to vary over the continuous eigenkets since it made the analog from discrete to continuous clearer, we have replaced that with the continuous index $p$, which makes far more sense in any case. The term $p^2/2m$ has replaced the term $E_{n}$ since it is the energy value corresponding to the eigenstate with eigenmomentum $p$.
\\\\
We now may choose a basis to work in to evaluate the propagator elements. Evaluating the position space matrix elements
$$
\bra{x}\prop{t}{0}\ket{x_{0}}
$$
makes more sense than evaluating the momentum space matrix elements, 
$$
\bra{p}{\prop{t}{0}}\ket{p'}
$$ since the latter clearly is going to involve delta functions and we might prefer to avoid this for an operator we want to readily apply. In the position space the elements are
$$
\bra{x}\prop{t}{0}\ket{x_{0}}=\infint e^{-ip^{2}t/2m\hbar}\ip{x}{p}\ip{p}{x_{0}}\,dp
$$
and this can be readily written in terms of the position space momentum eigenfunctions:
$$
\bra{x}\prop{t}{0}\ket{x_{0}}=\infint e^{-ip^{2}t/2m\hbar} \psi_{p}(x)\psi^{\ast}_{p}(x_{0})\,dp.
$$
The eigenfunctions are given by 
$$
\psi_{p}(x)=\frac{1}{\sqrt{2\pi\hbar}}e^{ipx}
$$
so the above is
$$
\begin{aligned}
\bra{x}\prop{t}{0}\ket{x_{0}}&=\infint e^{-ip^{2}t/2m\hbar} \psi_{p}(x)\psi^{\ast}_{p}(x_{0})\,dp\\
\bra{x}\prop{t}{0}\ket{x_{0}}&=\infint e^{-ip^{2}t/2m\hbar} \biggl(\frac{1}{\sqrt{2\pi\hbar}}e^{ipx/\hbar}\biggr)\biggl(\frac{1}{\sqrt{2\pi\hbar}}e^{-ipx_{0}/\hbar}\biggr)\,dp\\
\bra{x}\prop{t}{0}\ket{x_{0}}&=\frac{1}{2\pi\hbar}\infint e^{-ip^{2}t/2m\hbar} e^{ip(x-x_{0})/\hbar}\,dp\\
\bra{x}\prop{t}{0}\ket{x_{0}}&=\frac{1}{2\pi\hbar}\infint e^{-(it/2m\hbar)p^{2}+(i(x-x_{0})/\hbar)p}\,dp\\
\end{aligned}
$$
While to the inexperienced mathematician this integral is not straightforward, in the wider scope of things this integral is in fact trivial, because of the relatively simple fact we can quote that 
$$
\infint e^{-ax^2+bx}\,dx=e^{b^2/4a}\sqrt{\frac{\pi}{a}}.
$$
Plugging in $a:=it/2m\hbar$ and $b:=i(x-x_{0})/\hbar$, and changing the integral to be with respect to $p$ rather than $x$, we get:
$$
\begin{aligned}
\bra{x}\prop{t}{0}\ket{x_{0}}&=\frac{1}{2\pi\hbar}\infint e^{-(it/2m\hbar)p^{2}+(i(x-x_{0})/\hbar)p}\,dp\\
\bra{x}\prop{t}{0}\ket{x_{0}}&=\frac{1}{2\pi\hbar}e^{i(x-x_{0})/\hbar)^2/4(it/2m\hbar)}\sqrt{\frac{\pi}{it/2m\hbar}}\\
\bra{x}\prop{t}{0}\ket{x_{0}}&=\biggl(\frac{1}{2\pi\hbar}\biggr)e^{-m(x-x_{0})/2it\hbar}\sqrt{\frac{2m\pi\hbar}{it}}\\
\bra{x}\prop{t}{0}\ket{x_{0}}&=\sqrt{\frac{m}{2i\pi\hbar t}}\:e^{im(x-x_{0})/2\hbar t}\stab.
\end{aligned}
$$
It appears laborious, but actually requires neither advanced mathematical skills nor obscure mathematical facts to evaluate. The benefit is now we can plug in any state vector $\Psi$ and we will have a perfect understanding of how it will evolve with time. Thus such a process (with proof of the momentum energy relationship) would be sufficient to solve the problem of the free particle. 
\\\\
To prepare for the next section, we will look at a small modification of the free particle question, and see that it vastly changes the result we get so far as to even discretise the energy spectrum. We will from now onwards continue to use the propagator, as it is ubiquitous in post-expository quantum mechanics texts.
\section{Particle on an Ellipse\ast}
Let's start with an interesting modification to the free particle problem. Consider the ellipse below (which could be a circle-- the difference does not here matter):
$$
\\
\\
\\
$$
\begin{center}
\begin{tikzpicture}
\draw (1,1) ellipse (4 and 2);
\filldraw[black] (1,-1) circle (4pt) node[anchor=south east]{};
\node at (1,-1.75) {$x$};
\end{tikzpicture}
\end{center}
\\
Say that a free particle $\mathfrak{p}$ is confined to be on that ellipse at all times. Consider position $x$ as the particle's starting point, and imagine the particle moving anticlockwise around the ellipse infinitely many times. Now we can draw a helpful visual representation of the linear distance $\mathfrak{p}$ has travelled:
\begin{center}
\begin{tikzpicture}
\draw (0,1) -- (0,-0.25);
\draw (0,0) -- (11,0);
\draw (3,0) -- (3,-0.25);
\draw (6,0) -- (6, -0.25);
\draw (9,0) -- (9, -0.25);
\node at (3, -0.5) {$L$};
\node at (6, -0.5) {$2L$};
\node at (9, -0.5) {$3L$};
\node at (0, -0.5) {$0$};
\end{tikzpicture}
\end{center}
\\
...and so on infinitely many times. Take L to be the perimeter of the ellipse. Then it is clear that after moving in a perfect cycle for the length of one perimeter of the ellipse the particle will end back at position $x$ once more. In other words, we write:
$$
x \sim x+L
$$
This means that the position $x$ is equivalent to position $x+L$, and the relation carries on until infinity as $x+L\sim x+2L$ and so on.
\\\\
Now, considering the the system using Schrodinger's equation, we expect
$$
\psi(x) = \psi(x+L) 
$$
for all positions $x$. Now, the free particle time independent Schr\"{o}dinger Equation is familiar to us:
$$
\hat{H}\Psi = \frac{\hat{P}^2}{2m}\Psi = E\Psi
$$
and we also know $\psi(x)=\psi(x+L)$. Note that the $\psi(x)$ indicates we are working in position space, which makes perfect sense considering the interesting cyclic behaviour of varying position along the ellipse meaning we want $x$ to be the central changing variable.
\\\\
We can come to a solution for the particle on an ellipse quite organically. We first note that due to the form of the Hamiltonian energy and momentum must still be compatible observables, since neither operators are changed by the periodic behaviour bestowed by the elliptical setup. Therefore, the energy eigenkets we derived before are still correct:
$$
\ket{E;P^{+}}=\ket{p=\sqrt{2mE}\:}, \btab \ket{E;P^{-}}=\ket{p=-\sqrt{2mE}\:}
$$
for each $E$. We again choose the nondegenerate momentum eigenkets to represent the energy spectrum, though we do not know whether it is discrete or continuous yet. The solution comes from noting facts we already know. The momentum eigenkets, which are also energy eigenkets, correspond to:
$$
\ket{p}\duac\frac{1}{\sqrt{2\pi\hbar}}e^{ipx/\hbar}
$$
and this seems innocuous until we remember the boundary condition: 
$$
\Psi(x)=\Psi(x+L)
$$
That is, due to there being no external potential, we do not expect the wavefunction to change when it travels a length of the perimeter of the ellipse and returns to the same position. We therefore expect the same bounds to apply to the energy eigenkets- and similarly, the momentum eigenkets. However, this means we expect:
$$
\ket{p; x=x}\duac\frac{1}{\sqrt{2\pi\hbar}}e^{ipx/\hbar}= \ket{p; x=x+L}\duac\frac{1}{\sqrt{2\pi\hbar}}e^{ip(x+L)/\hbar}.
$$
We then expect 
$$
\frac{1}{\sqrt{2\pi\hbar}}e^{ipx/\hbar}=\frac{1}{\sqrt{2\pi\hbar}}e^{ip(x+L)/\hbar} \implies e^{i(p/\hbar)x}=e^{i(p/\hbar)x}e^{i(p/\hbar)L}
$$
and so we must have 
$$
e^{i(p/\hbar)L}=1
$$
This is rather dramatic. By Euler's equation we get:
$$
\begin{aligned}
e^{ix} = \cos(x) + i\sin(x)
\end{aligned}
$$
\\
Therefore
$$
\begin{aligned}
e^{i(p/\hbar)L} = 1 \Rightarrow\:\: \cos((p/\hbar)L) + i\sin((p/\hbar)L)=1
\end{aligned}
$$
\\
We need to get rid of the sine function part which is multiplied by an imaginary unit, since the cosine function is real-valued, the sine function is real valued, and $1$ is a real number, so we must have $\sin((p/\hbar)L)=0$ or we would get a complex but not real
%The term for ``complex but not real'' is imaginary, or purely imaginary, or similar.
left hand side and a real right hand side: a contradiction. Then to finish we need $\cos{(p/\hbar)L}=1$, which means 
$$
pL/\hbar=2n\pi, \:\: n\in\mathbb{Z}.
$$
\\
We see now that the above equation is quantized- indexed by the integers $n$. Therefore momentum $p$ is also quantized and therefore energy must also be quantised as the energy eigenvalues are equal to $p^2/2m$. Such is a very common theme in quantum mechanics: $\sin(n\pi) = 0$ and $\cos{2n\pi}=1$ for $n\in\mathbb{Z}$, which indexes by integers $n$ when there is boundary behaviour causing periodic recurrences. Now we can index the eigenvalues with $n$, and summarise that:
$$
\begin{aligned}
p_n= \frac{2\pi n\hbar}{L}, \:\: n\in\mathbb{Z}.
\end{aligned}
$$
Then, the eigenenergies must be:
$$
E_{n}=(p_n)^2/2m=\frac{2\pi^2n^2\hbar^2}{mL^2}.
$$
Now we want to evaluate the new propagator for this free particle on an ellipse. To do this we could solve the time independent Schr\"{o}dinger to find the energy eigenstates. However, we can also simply solve the momentum eigenvalue equations, which are a bit simpler, since we already know that 
$$
\hat{P}\ket{\psi_{p}}=p\ket{\psi_{p}}\implies \psi_{p}(x)=\frac{1}{\sqrt{2\pi\hbar}}e^{ipx/\hbar}.
$$
Plugging in the momentum eigenvalues $p_{n}$ and indexing by $n$, we have
$$
\psi_{p}^{(n)}(x)=\frac{1}{\sqrt{2\pi\hbar}}e^{2\pi nix/L}
$$
which are also the position space energy eigenstates corresponding to the energy eigenvalues $E_{|n|}$. The propagator can then be expressed:
$$
\prop{t}{0}=\sum_{n}e^{-iE_{n}t/\hbar}\ket{E_{n}}\bra{E_{n}}
$$
and we are working with discrete energy eigenkets so we can use this form of the propagator as well. The position space elements should be clear:
$$
\begin{aligned}
\bra{x}{\prop{t}{0}}\ket{x_{0}}&=\sum_{n}e^{-iE_{n}t/\hbar}\ip{x}{E_{n}}\ip{E_{n}}{x_{0}}\\
\bra{x}{\prop{t}{0}}\ket{x_{0}}&=\sum_{n}e^{-iE_{n}t/\hbar}E_{n}(x)E_{n}^{\ast}(x_{0}).
\end{aligned}
$$
Substituting in the energy eigenstates 
$$
\bra{x}{\prop{t}{0}}\ket{x_{0}}=\sum_{n}e^{-iE_{n}t/\hbar}\biggl(\frac{1}{\sqrt{2\pi\hbar}}e^{2\pi nix/L}\biggr)\biggl(\frac{1}{\sqrt{2\pi\hbar}}e^{-2\pi nix_{0}/L}\biggr)
$$
and substituting the energy eigenvalue $E_{n}$,
$$
\bra{x}{\prop{t}{0}}\ket{x_{0}}=\frac{1}{2\pi \hbar}\sum_{n}e^{-2i\pi^{2}n^{2}\hbar t/mL^{2}}e^{2\pi nix/L}e^{-2\pi nix_{0}/L}
$$
simplifying,
$$
\bra{x}{\prop{t}{0}}\ket{x_{0}}=\frac{1}{2\pi \hbar}\sum_{n}e^{(2\pi i n)/L[\pi n\hbar t/mL ]}.
$$
\\\\
%Have you lost a line here? Something seems to be missing.
\\\\
we see that the full set of solutions to Schrodinger's Equation for a free particle on a circle can be categorised by:
$$
\Psi_n(x)= Ne^{ik_n x}
$$
%I think it's well worth pointing out quite how obvious a result this is, given that it was reached by some quite complicated-looking mathematics. After all, this last equation is nothing more than the statement that the energy eigenkets are (snapshots of) standing helical waves. The reader might well have been able to guess this beforehand! It's always a good idea to offer the reader the chance to guess at a result before it is derived. Why? Because they'll enjoy it! And, having enjoyed it, they'll see more meaning in the mathematics, it having produced a result they already trust.
where N was the normalisation constant assumed earlier when we eliminated $\int_{0}^{L}\psi^\ast(x)\psi(x)$. Its value can calculated easily:
$$
\begin{aligned}
&\Psi_n(x)= Ne^{ik_n x}, \:\: \int_{0}^{L}\psi_n^\ast(x)\psi_n(x)=1\\
\Rightarrow\:\: &\int_{0}^{L}Ne^{ikx}Ne^{-ikx} = 1\\
\Rightarrow\:\: &N^2\int_{0}^{L}1 = 1\\
\Rightarrow\:\: &N^2\int\frac{d}{dx}(x+c)=1\\
\Rightarrow\:\: &N^2\biggl[x+c\biggr]_{0}^{L} = 1 \Rightarrow\:\: N^2L = 1 \Rightarrow\:\: N= \frac{1}{\sqrt{L}}
\end{aligned}
$$
\\
Overall, we have:
$$
\Psi_n(x)=\frac{1}{\sqrt{L}}e^{ik_nx}
$$
\\
and since $k_nL=2\pi n \Rightarrow\:\: k_n=\frac{2\pi n}{L}$, we can also write this as:
$$
\Psi_n(x)=\frac{1}{\sqrt{L}}e^{\frac{2\pi nix}{L}}.
$$
This is a normalised set of wavefunctions due to the normalisation coefficient, but is it an orthogonal set too? We can verify this.
\\
Furthermore, we also know that as $V(x)=0$ then associated energies indexed by integers $n$ are:
$$
E_n = \frac{\hat{p}^2}{2m} = \frac{\hbar^2k_n^2}{2m} = \frac{\hbar^2}{2m}\left(\frac{2\pi n}{L}\right)^2
$$
\\
so they are
$$
E_n=\frac{2\pi^2\hbar^2n^2}{mL^2}
$$
\\
Note that despite the fact $n$ can be negative, corresponding to negative momentum, the $n^2$ term in the equation  for $E$ ensures that $E > 0$ as we have earlier shown. Clearly there are infinite energy eigenstates as there are infinite integers $n$ to index $E_n$.
\\\\
However, since $E_n$ is a function of $n^2$ it is also clear that all the energy eigenstates for a free particle on an ellipse can correspond to both $\psi_n$ and $\psi_{-n}$ which both have energy $E_n$, save for $E_0$. We say that all these non-zero energy eigenstates are degenerate: they can correspond to multiple wavefunctions, here 2. The temptation would be to say that we have an issue herein since that would make $\psi_n$ and $\psi_{-n}$ indistinguishable from the point of view of energy: but of course we very easily know that what distinguishes $\psi_n$ and $\psi_{-n}$ is that they have different momenta.
\section{Bound States}
In the last problem, we made one modification-- or really, added one constraint-- to the free particle problem, and found this discretised the energy spectrum immediately. Physically, the constraint seems rather significant, given that we expected the particle's motion to loop around in infinite cycles. Mathematically, however, it was only equivalent to adding positionwise periodicity. There exists this question now of how that discretisation occurred in the first place, and specifically, we can point to a specific class of problems which will be valuable to our understanding of this: those problems of bound states.
\subsubsection{A General Discussion}
A bound state in quantum mechanics is a state where we have the relationship $$
|x|\to\infty\implies\Psi(x)\longrightarrow 0.
$$
That is, the wavefunction is focused on one point, which we can call position $0$, and the further away we get from that point the smaller the value $\Psi(x)$ (the probability density corresponding to that point) is. The most basic example of this would occur if we had some barrier which prevented or made it practically very difficult for a particle trying to escape outside it. Of course, on a microscopic scale we cannot speak of stone walls or physical barriers, but we will see that barriers nevertheless can exist based on the potential of the system. We start by introducing a fact which exists in classical mechanics and stays valid in quantum mechanics- that a particle cannot have energy less than the minimum value of the potential of the system at any point. The proof is as follows: we know that the Hamiltonian is given by the formula
$$
\hat{H}=\frac{\hat{P}}{2m}+V(x).
$$
%Check the above...
Now pick a normalised energy eigenket $\psi_{E}(x)$ which has energy $E$. We have:
$$
\optrip{\psi_{E}}{H}{\psi_{E}}=E\sip{\psi_{E}}=E
$$
but we also have
$$
\optrip{\psi_{E}}{H}{\psi_{E}}=\bra{\psi_{E}}{\hat{T}+V(x)}\ket{\Psi_{E}}=\optrip{\psi_{E}}{{T}}{\psi_{E}}+\bra{\psi_{E}}V(x){\ket{\Psi_{E}}}
$$
%Have you specified that the kinetic energy operator is notated $\hat{T}$, prior to this point? I don't remember it, if so. The reader is likely to require reminding at this point. Remember, you are not speaking down to a reader to remind them what notation means. Every time you do so, it is likely that there are many readers who will be grateful. The trick is to do it minimally, without fuss, so the reader doesn't even notice you've done it!
and for some minimum value of $V(x)$ we have 
$$
\optrip{\psi_{E}}{H}{\psi_{E}}=\optrip{\psi_{E}}{{T}}{\psi_{E}}+\bra{\psi_{E}}V(x){\ket{\Psi_{E}}}\geq\optrip{\psi_{E}}{{T}}{\psi_{E}}+V_{\min}\sip{\psi_{E}}
$$
that is,
$$
E\geq\optrip{\psi_{E}}{T}{\psi_{E}}+V_{\min}
$$
\\\\
using the algebraic form of the momentum operator, the Schr\"{o}dinger Equation is
$$
-\frac{\hbar^2}{2m}\snd{\Psi}{x}+(V(x)-E)\Psi=0.
$$
If $E\geq V(x)$ for some point $x$ then $(V(x)-E)$ will clearly be nonpositive. However, at the moment $E<V(x)$, we must get the condition that $V(x)-E$ is positive. If we put denote this difference as $\Delta>0$, then we get the equation
$$
-\frac{\hbar^2}{2m}\snd{\Psi}{x}+\Delta\Psi=0\implies\frac{\hbar^2}{2m}\snd{\Psi}{x}=\Delta\Psi.
$$
This is comparable to the differential equation , whose solution leaks out to infinity and does not fit the definition of a bound state. Therefore in any bound state we can never have energy less than the potential of the system; furthermore, this means that no energy eigenstates exist whose corresponding eigenvalues are less than the minimum value of the potential at some point.
\\\\
We can now see how we can create a bound state: by setting regions where the potential is higher than the energy of the state and where therefore the wavefunction must disappear. In particular, a region with infinite potential must immediately have the wavefunction vanish as it cannot have an energy greater than infinity. Let us use this to investigate a classic introductory problem in quantum mechanics which involves a bound state, the infinite square well.
\section{Particle in an Infinite Square Well}
Examine the following diagram:
$$
\\
\\
\\
$$
\begin{center}
\begin{tikzpicture}
\draw [-to, line width = 0.5mm] (-2.5,0) -- (-2.5,4);
\draw [-to, line width = 0.5mm] (2.5,0) -- (2.5,4);
\draw [line width = 0.5mm] (-3,0) -- (3,0);
\draw[thin,pattern={north east lines}, pattern color= black] (3,0) rectangle (2.5,3.85);
\draw[thin,pattern={north east lines}, pattern color= black] (-2.5,0) rectangle (-3,3.85);
\node at (-2.5, -0.5) {$x=0$};
\node at (2.5, -0.5) {$x=L$};
\end{tikzpicture}
\end{center}
At the shaded regions the potential is defined to be infinite. But between positions $x=0$ and $x=L$ it is defined as $V(x)=0$. Due to the fact that the energy of a particle must be greater than the potential it is experiencing, we have two conditions that within the well it must have positive energy, and outside its wavefunction must vanish, so the lines $x=0$ and $x=L$ act as hard walls which bind the state to being between positions $0$ and $L$.
\\\\
We can list the properties of the system given by the boundary conditions:
\begin{enumerate}
    \item $\Psi(x)$ does not exist if $x>L$ or $x<0$. 
    \item The new normalisation requirement for the system $\Psi$ is $\sip{\Psi}=1=\int_{0}^{L} \Psi^\ast(x)\Psi(x)\,dx$. This is because the boundary conditions require that the probability of the wavefunction being between $0$ and $L$ is $1$, not simply just the probability that it is between negative infinity and infinity.
\end{enumerate}
\\
We know that the wavefunction is continuous. This means actually that it must vanish at the walls $x=0$ and $x=L$, or otherwise it couldn't drop to $0$ straight afterwards. So $x=0$ and $x=L$ are \apos{hard} walls. In other words,
$$
\Psi(0) = 0 = \Psi(L) 
$$
\\
Finally, within the region $x\in(0,L)$, we have $V(x)=0$. Thus the problem reduces to the free particle inside the square well. 
\\\\
We split the problem into three components: the first, investigating the wavefunction inside the well, the second, investigating the wavefunction outside the well, and finally, investigating the wavefunction on the boundary lines $x=0$ and $x=L$; we will cover these in that order.
\\\\
The first part of the problem, the problem of the wavefunction inside the square well, is simply the free particle problem since the potential is $0$, within finite position bounds. So, we once again have
$$
\frac{d^2\Psi}{dx^2} = -{\frac{2mE}{\hbar^2}}\Psi
$$
which has the solution
$$
\Psi(x)=Ae^{ikx}+Be^{-ikx}
$$
for some constants $A,B$, where 
$$
k=p/\hbar=\frac{\sqrt{2mE}}{\hbar}.
$$
We cannot yet determine these constants $A,B$ until we look at the other boundary conditions.
\\\\
The second component of the problem is the problem of the wavefunction outside the well. We recall that the wavefunction
$$
\Psi(x)
$$
is a probability distribution function which returns a value given a position $x$ which is its component in the basis and therefore a probability amplitude for achieving that measurement of $x$. Given that outside the well we have infinite potential, we must have probability zero of finding the particle there. This means that the solution of the wavefunction must also be 
$$
\Psi(x)=0 \mtab \forall x\in(-\infty,0)\cup(L,\infty)
$$
\\
The most important part of the problem comes, however, with the third component of the problem- that of the wavefunction on the hard walls $x=0$ and $x=L$. We have
$$
\Psi(x) = Ae^{ikx}+Be^{-ikx}
$$
for some constants $A$ and $B$. We can rewrite this with Euler's formula:
$$
\begin{aligned}
e^{ix}&=\cos(x)+i\sin(x)\\
\implies Ae^{ikx}+Be^{-ikx}&=A\cos(kx)+iA\sin(x)+B\cos(-kx)+iB\sin(-kx).
\end{aligned}
$$
The cosine function is even and the sine function is odd. Therefore we have 
$$
Ae^{ikx}+Be^{-ikx}=(A+B)\cos(kx)+i(A-B)\sin(kx).
$$
Now, examining the wall conditions, we firstly have:
$$
\Psi(x=0) = 0 
$$
\\
But since $\sin(0)=0$ this means 
$$
(A+B)\cos(kx)=(A+B)\cos(0)=0.
$$ 
However, $\cos(0) \neq 0 $. Therefore we must conclude that $A+B=0$. However, if we conclude this fact then it must be true for all regions, not simply just the wall $x=0$, as the wavefunction cannot simply morph as soon as it reaches $x=0$. Thus for the wavefunction of the whole problem of the infinite square well, which we will now be covering in trigonometric form, we can omit the first coefficient entirely and therefore vanish the cosine term. We are left with
$$
\Psi(x) = i(A-B)\sin{kx}.
$$
\\
Note that we have 
$$
A+B=0
$$
so neither of the individual constants $A$ and $B$ can be $0$, or that would imply the other is $0$, and we would have (in exponential form)
$$
\Psi(x)=0e^{ikx}+0e^{-ikx}=0
$$
which is absurd.
%Haha! It's not absurd, because it's a perfectly good solution to the problem: a zero wave. The word for such a solution is ``trivial'', i.e. not absurd but quite the opposite: uninteresting!
Now we can use the other boundary condition.
$$
\psi(x=L)=0 \Rightarrow\:\: i(A-B)\sin{kL}=0.
$$
\\
If we had 
$$
A-B=0
$$
then we would have 
$$
A-B=A+B=0 \implies B=0 \implies A=0
$$
which is impossible as shown above. Therefore we must have 
$$
\sin(kL)=0.
$$
This means that we must have $kL=n\pi, \:\: n \in \mathbb{Z}$. This means the wave number $k$ is again quantized! Thus we can define.
$$
k_n=\frac{n\pi}{L}
$$
which therefore means we have discrete momenta:
$$
p_{n}=\hbar k_{n}=\frac{n\pi\hbar}{L}
$$
and discrete energy:
$$
E_{n}=\frac{p_{n}^2}{2m}=\frac{\hbar^2\pi^2 n^2}{2mL^2}\stab.
$$
Plugging in to our original general solution for the wavefunction, we have the complete set of solutions 
$$
\Psi_{n}(x)=N\sin\left(\frac{n\pi x}{L}\right).
$$
The constant $N$ is a normalisation constant, which incorporates $i(A-B)$, which we had before. It must be remembered that $A$ and $B$ are arbitrary constants which really are not very important, so rolling them into a new constant $N$ is not a big problem. We can restrict ourselves to considering $n \in \mathbb{Z}^+$ since the parity of the wavefunction is irrelevant in producing the same results. This time, we will solve for the normalisation constant $N$.
$$
\int_{0}^{L}\Psi^\ast(x)\Psi(x)\,dx = N^2\int_{0}^{L}\sin^2\left(\frac{n\pi x}{L}\right) \,dx
$$
since the sine function is real valued so its complex conjugate is itself. Then,
$$
N^2\int_{0}^{L}\sin^2\left(\frac{n\pi x}{L}\right) \,dx = 1
$$
\\
The integral requires some algebra to evaluate. Using the trigonometric identity $\sin^2{kx}=\frac{1-\cos{2kx}}{2}$, we get:
$$
\begin{aligned}
&\sin^2{kx}=\frac{1-\cos{2kx}}{2}\\
\Rightarrow\:\: &\int_{0}^{L}\sin^2{kx}=\int_{0}^{L}\frac{1-\cos{2kx}}{2}\\
\Rightarrow\:\: &\int_{0}^{L}\sin^2{kx}=\int_{0}^{L}\frac{1}{2}-\int_{0}^{L}\frac{1}{2}{\cos{2kx}}\\
\Rightarrow\:\: &\int_{0}^{L}\sin^2{kx}=\int_{0}^{L}\frac{d}{dx}\left(\frac{1}{2}x+c\right)-\frac{1}{2}\int_{0}^{L}\frac{d}{dx}\left({\frac{1}{2k}\sin{2kx}}\right)\\
\Rightarrow\:\: &\int_{0}^{L}\sin^2{kx}=\biggl[\frac{1}{2}x+c\biggr]_{0}^{L}-\biggl[\frac{1}{2k}\sin{2kx}\biggr]_{0}^{L}\\
\Rightarrow\:\: &\int_{0}^{L}\sin^2{kx}=\frac{L}{2}-\biggl[\frac{1}{2k}\sin{2kx}\biggr]_{0}^{L}\\
\end{aligned}
$$
%You'd render the above algebra clearer by cutting out the implies sign and the integral on lines 3,4,5,6, and aligning the = signs, in a similar manner to the algebra below.
\\
Now, plugging in $k=\frac{n\pi}{L}$, we get:
$$
\begin{aligned}
\int_{0}^{L}\sin^2\left(\frac{n\pi x}{L}\right) \,dx &= \frac{L}{2}-\biggl[\frac{L}{n\pi}\sin\left(\frac{2n\pi a}{L}\right)-\frac{1}{2k}\sin{0}\biggr]\\
&=\frac{L}{2}-\biggl[\frac{L}{n\pi}\sin{2n\pi}-0\biggr]\\
&=\frac{L}{2}
\end{aligned}
$$
\\
since $\sin{2n\pi}=0$. So for the normalisation constant we have evaluated the integral from $0$ to $L$ and get:
$$
N^2\int_{0}^{L}\sin^2\left(\frac{n\pi x}{L}\right) \,dx=\frac{N^2L}{2}=1 \Rightarrow\:\: N=\sqrt{\frac{2}{L}}
$$
as our normalisation constant. That gives us the following set of normalised solutions indexed by positive integers $n$:
$$
\Psi_n=\sqrt{\frac{2}{L}}\sin\left(\frac{n\pi x}{L}\right).
$$
We can confirm these are orthogonal to each other for different $n$ and thus create an orthonormal set. The normalised property is covered by the normalisation coefficient $\sqrt{{2}/{L}}$, and the orthogonality can be proved by considering the inner product- from $x=0$ to $x=L$ since this is the domain of the wavefunction without it vanishing.
    $$
    \begin{aligned}
    \int_{0}^{L} \psi^\ast_m(x)\psi_n(x) \,dx = \left(\sqrt{\frac{2}{L}}\right)^2\int_{0}^{L}\sin\left(\frac{m\pi x}{L}\right)\sin\left(\frac{n\pi x}{L}\right)\,dx
    \end{aligned}
    $$
    \\
    since the complex conjugate of the real valued sine function is itself. Then:
    $$
    \begin{aligned}
    \sin{x}\sin{y}&= \frac{1}{2}[\cos(x-y)+\cos(x+y)]\\
    &\Rightarrow\:\: \frac{2}{L}\int_{0}^{L}\sin\left(\frac{m\pi x}{L}\right)\sin\left(\frac{n\pi x}{L}\right)\,dx\\
    &=\frac{2}{L}\int_{0}^{L}\frac{1}{2}\biggl[\cos\left(\frac{(m-n)\pi x}{L}\right)+\cos\left(\frac{(m+n)\pi x}{L}\right)\biggr]\\
    &= \frac{1}{L}\int_{0}^{L}\biggl[\cos\left(\frac{(m-n)\pi x}{L}\right)+\cos\left(\frac{(m+n)\pi x}{L}\right)\biggr]\\
    \end{aligned}
    $$
    \\
    Using some chain rule:
    $$
    \frac{d}{dx}\sin\left(\frac{(m-n)\pi x}{L}\right)= \frac{d}{dg}\sin(g(x))*\frac{d}{dx}\frac{(m-n)\pi}{L}x
    $$
    \\
    where $g(x) = \frac{(m-n)\pi}{L}x$. Then this means:
    $$
    \frac{d}{dx}\sin\left(\frac{(m-n)\pi x}{L}\right)=\frac{(m-n)\pi}{L}\cos\left(\frac{(m-n)\pi}{L}x\right)
    $$
    And therefore,
    $$
    \cos\left(\frac{(m-n)\pi}{L}x\right)= \frac{L}{(m-n)\pi}\frac{d}{dx}\sin\left(\frac{(m-n)\pi x}{L}\right)
    $$
    \\
    so we doing something similar with the $\cos(\frac{(m+n)\pi x}{L})$ term we can write the integral as:
    $$
    \frac{1}{L}\int_{0}^{L}\biggl[\left(\frac{L}{(m-n)\pi}\right)\frac{d}{dx}\sin\left(\frac{(m-n)\pi}{L}x\right)+\left(\frac{L}{(m+n)\pi}\right)\frac{d}{dx}\sin\left(\frac{(m+n)\pi}{L}x\right)\biggr]
    $$
    \\
    then, pulling out constants and using the fundamental theorem of calculus to get rid of the integral, we get the expression
    $$
    \biggl[\frac{1}{(m-n)\pi}\sin\left(\frac{(m-n)\pi}{L}x\right)+\frac{1}{(m+n)\pi}\sin\left(\frac{(m+n)\pi}{L}x\right)\biggr]_{0}^{L}
    $$
    \\
    clearly at $x=0$ we get a bunch of $\sin(0)$ terms so the large bracket is 0. At $x=L$ the value of $x$ cancels with the denominator inside the sine function, and so we get the sine values of $m-n\in\mathbb{Z}^{+}$ lots of $\pi$, which also ends up with 0. So altogether we have proved the whole integral and therefore the inner product of $\Psi_m$ and $\Psi_n$ is $0$. Note that the whole process is invalid when $m=n$ since then the fraction $\frac{L}{(m-n)\pi}$ we see in the expression of $\cos(\frac{(m-n)\pi}{L}x)$ as a derivative is clearly invalid due to division by $0$. Now we can summarise:
    $$
    \ip{\Psi_{m}}{\Psi_{n}} = \delta_{n,m}
    $$
    which is our original definition of an orthogonal set of wavefunctions $\ket{\Psi}$. So indeed, the set of $\Psi_{n}$ indexed by integers $n$ is an orthonormal set of states which can then be linearly combined into any superposition of states with different probabilities.
\subsubsection{Analysing solutions}
We can summarise our numerical analysis of eigenstates as follows. We obtained:
$$
\psi_n=\sqrt{\frac{2}{L}}\sin(\frac{n\pi x}{a}), \:\:\:\:\: E_n=\frac{\hbar^2\pi^2 n^2}{2ma^2}, \:\:\:\:\: n \in \mathbb{Z}^+
$$
\\\\
We can now plot the first $\psi_n$ on an axis from $0$ to $a$. This is in shown in figure 1.
%Is there a figure 1.???
Clearly, we observe a few properties:
\begin{enumerate}
    \item We call a zero of $\psi$ inside the domain of $\psi$ a node. Then here we see that $\psi_n$ has $n-1$ nodes: the zeros at $x=0$ and $x=a$ do not count as they are not inside the domain. More importantly, for this set of solutions to $\psi$ for every $1$ that you increase the integer indexing $\psi_n$ you also increase the number of nodes by $1$. This is actually generally true as you go up from the ground state- the lowest energy state and nearly always $\psi_1$ in a set of $\psi_n, \:\: n \in \mathbb{Z}^+$: for any set of solutions $\psi$ we add 1 node every time we move to the next possible energy level.
    \\
    \item The $\psi_n$ are clearly alternately symmetric and asymmetric as you go up index numbers $n$. $\psi_1$ is symmetric, $\psi_2$ is asymmetric, $\psi_3$ is symmetric, etc. Crucially, we call symmetric solutions \textbf{even} and the asymmetric solutions \textbf{odd}. The definition of an even function $f(x)$ is that $f(x)=f(-x)$. The definition of an odd function is where $f(-x)=-f(x)$. Technically here there are neither even solutions nor odd solutions since this is not true for any $\psi_n$ in the question defined. But if we had taken the midpoint of the well to be $x=0$ then looking at the first few wavefunctions it is clear this would have been true, so the parity of functions we think in is analogous. We will see in our next problem how important discussion of function parity will be in breaking down questions as they get more and more complex and we seek to categorise more and more realistic situations and problems. But there exists a powerful fact: if a particle is in an even potential then all wavefunctions are either odd or even. In this problem again if we had set the middle of the potential well to be $0$ then we would have seen an example of this fact: the potential would be even since it is symmetric about $0$ and the solutions are alternately even and odd. Clearly this is an exceptionally meaningful fact. We can prove it fairly easily for our one-dimensional purposes:
    $$
    \text{By Schrodinger}, \:\: -{\frac{\hbar^2}{2m}\frac{d^2}{dx^2}}\psi(x)+V(x)\psi(x)=E\psi(x)
    $$
    \\
    when we substitute in $x=-x$, we get:
    $$
    -{\frac{\hbar^2}{2m}\frac{d^2}{dx^2}}\psi(-x)+V(-x)\psi(-x)=E\psi(-x)
    $$
    \\
    we are considering a question where the potential is even and thus $V(x)=V(-x)$ So then:
    $$
    -{\frac{\hbar^2}{2m}\frac{d^2}{dx^2}}\psi(-x)+V(x)\psi(-x)=E\psi(-x)
    $$
    \\
    we step back out of the algebraic manipulation and see that clearly this means that if $\psi(x)$ is a solution in an even potential then $\psi(-x)$ is also a solution. This in itself is very important, naturally. But since we are working in the same one-dimensional vector space of solutions to the same Schrodinger then it is impossible for these two functions to be linearly independent, or orthogonal in a one-dimensional space. Therefore we can say
    $$
    \psi(x)=a\psi(-x)
    $$
    \\
    for some constant $a$. Then $a\psi(-x)=\psi(x)$, so it satisfies the conditions, and therefore is normalised. So we have:
    $$
    |a|^2\int_{-\infty}^{\infty}\psi^{\ast}(-x)\psi(-x)\,dx=1
    $$
    \\
    But $\psi(-x)$ itself is a normalised solution so the integral is equal to 1. Therefore 
    $$
    |a|^2=1 \Rightarrow\:\:\:\: a=\pm1
    $$
    By definition when $a=1$ then $\psi$ is even since we get $\psi(x)=\psi(-x)$, and when $a=-1$ then $\psi$ is odd, since we get $\psi(x)=-\psi(x)$. Therefore for an even potential all solutions $\psi$, when normalised, must be even or odd.
\end{enumerate}
\section{Harmonic Oscillator \ast}
We now introduce a third less fundamental but perhaps more impactful problem which must be included in any discussion of quantum mechanics. We will take one system we know very well from classical mechanics, the spring, which has a potential of 
$$
V(x)=\frac{1}{2}kx^2
$$
for the spring constant $k$. This can also be written via the angular frequency, $\omega=k/m$, as 
$$
V(x)=\frac{1}{2}mX\omega^{2}.
$$
There is the standard method of solving this problem in the quantum mechanical version, where the Hamiltonian is 
$$
\hat{H}=\frac{\hat{P}^2}{2m}+\frac{1}{2}m\hat{X}^{2}\omega^{2}.
$$
We note that for the system the spring constant and mass are both constants, and so $\omega=k/m$ is a constant and therefore should not and cannot be replaced by an expression in the position and momentum operators for the quantum Hamiltonian. Now the conventional method would be to solve the problem in position space and implement strategies much like we have shown above. But there is a method, courtesy of Dirac, which shows that we can use the energy eigenbasis as well.
\\\\
In principle this should be difficult, as to find the energy eigenkets which form the energy eigenbasis is tantamount to solving for the propagator for time evolution, in which case it is unclear why we would ever need to be in the energy eigenbasis after that. However, elegance will show in a clever way this limitation is not quite concrete for all problems, and the energy eigenbasis can be very useful even before we know the eigenvectors.
%This last section 8.4 needs some sorting out! It starts entitled Harmonic Oscillator, for which you then introduce the counting operator, and thence creation and destruction operators. But, as far as I can tell, none of that has been related back to the harmonic oscillator, which you mention and then forget about. 
%I think it would be much better to entitle and introduce this last section as the road to quantum field theory, which is what almost all of it is.
\\\\
To start, we define the operator
$$
a=\sqrt{\frac{m\omega}{2\hbar}}\left(\hat{X}+\frac{i\hat{P}}{m\omega}\right).
$$
Then we consider its hermitian adjoint,
$$
a^{\dagger}=\sqrt{\frac{m\omega}{2\hbar}}\left(\hat{X}-\frac{i\hat{P}}{m\omega}\right).
$$ 
It is clear these operators are not hermitian. What relationship is held by the two operators? They turn out not to be unitary either. Instead, we get 
$$
[a,a^{\dagger}]=1
$$
which can be easily verified:
$$
\begin{aligned}
[a,a^{\dagger}]&=\frac{m\omega}{2\hbar}\left[\biggl(\hat{X}+\frac{i\hat{P}}{m\omega}\biggr)\biggl(\hat{X}-\frac{i\hat{P}}{m\omega}\biggr)\right]-\frac{m\omega}{2\hbar}\biggl[\left(\hat{X}-\frac{i\hat{P}}{m\omega}\right)\left(\hat{X}+\frac{i\hat{P}}{m\omega}\right)\biggr]\\
&=\frac{m\omega}{2\hbar}\biggl[\hat{X}^{2}-\frac{i}{m\omega}[\hat{X},\hat{P}]-\left(\frac{i\hat{P}}{m\omega}\right)^2-\hat{X}^2-\frac{i}{m\omega}[\hat{X},\hat{P}]+\left(\frac{i\hat{P}}{m\omega}\right)^2\biggr]\\
&=\frac{m\omega}{2\hbar}\biggl[-\frac{i}{m\omega}(2i\hbar)\biggr]=\frac{m\omega}{2\hbar}\frac{2\hbar}{m\omega}=1
\end{aligned}
$$
Finally, we can define an operator, $N$, to be $a^{\dagger}a:=N$. This operator is hermitian as $a^{\dagger}a$.
%Needs a little tweak.
This operator has the form, as seen above:
$$
\begin{aligned}
N&=\frac{m\omega}{2\hbar}\biggl[\hat{X}^2+\frac{i}{m\omega}[\hat{X},\hat{P}]-\left(\frac{i\hat{P}}{m\omega}\right)^2\biggr]=\left(\frac{m\omega}{2\hbar}\right)\left[\hat{X^2}+\frac{\hat{P}^2}{m^2\omega^2}+\frac{i}{m\omega}[\hat{X},\hat{P}]\right]\\
&=\frac{1}{\hbar\omega}\frac{\hat{P}^2}{2m}\frac{1}{2\hbar}+{\hat{X}^2}m\omega+\frac{i}{2\hbar}i\hbar
\end{aligned}
$$
Regarding the Hamiltonian of the system again, this is 
$$
N=\frac{\hat{H}}{\hbar\omega}-\frac{1}{2}\implies\hat{H}=\hbar\omega\left(N+\frac{1}{2}\right).
$$
A bit of clever compatibility thinking is again the step here. Technically $N$ is not an observable operator, but note that what we proved about compatible observables did not hinge on the fact that the operators were Hermitian or had real eigenvalues. Only the third component, that about successive measurements of different observables and whether or not they affect the eigenstate, was contingent on the discussion being about observables in the other place.
%I didn't understand this last sentence.
In other words, if two operators commute they possess a common eigenbasis; this is not simply limited to observable operators, as a review of the proof we gave will show.
\\\\
We have seen above that the operator $N$ is a linear combination of the Hamiltonian. This means that they will commute. This in turn means that they possess a common eigenbasis! Thus there exist energy eigenkets $\ket{E}$ which are also eigenkets of the operator $N$. Thus we have the relationship
$$
\hat{H}\ket{E}=E\ket{E}
$$
as usual, but also the relationship
$$
N\ket{E}=n\ket{E}
$$
for some eigenvalue $n$. We can now equate the eigenvalues!
$$
\hat{H}\ket{E}=\hbar\omega\left(N+\frac{1}{2}\right)\ket{E}=\hbar \omega N\ket{E}-\frac{\hbar\omega}{2}\ket{E}=\hbar\omega\left(n+\frac{1}{2}\right)\ket{E}
$$
So the energy eigenvalues are given by 
$$
E_{n}=\left(n+\frac{1}{2}\right)\hbar\omega.
$$
The author has tried to instil a justified suspicion in the reader for indexing by $n$ without knowing there is a discrete case which can be indexed by integers; indeed, we still do not know anything about the eigenvalues $n$ so energy could still well be continuous at this stage, and indexing by $n$ would be meaningless. However, we will soon prove that in this case this step is fine as the eigenvalues of $N$ are in fact nonnegative integers $n$! We will from now on refer to $N$ as the counting operator, as is widespread convention. Let us consider some more commutation relations.
\\\\
By the commutation relation 
$$
[AB,C]=A[B,C]+[A,C]B
$$
we have
$$
[N,a^{\dagger}]=[a^{\dagger}a,a^{\dagger}]=a^{\dagger}[a,a^{\dagger}]+[a^{\dagger},a^{\dagger}]a=a^{\dagger}(1)+0=a^{\dagger}.
$$
Now we also have
$$
[N,a^{\dagger}]=Na^{\dagger}-a^{\dagger}N\implies Na^{\dagger}=a^{\dagger}N+[N,a^{\dagger}]=a^{\dagger}N+a^{\dagger}.
$$
Thus applying the operator to an energy eigenket, now indexed by $n$,
$$
Na^{\dagger}\ket{E_{n}}=(a^{\dagger}N+a^{\dagger})\ket{E_{n}}=a^{\dagger}N\ket{E_{n}}+a^{\dagger}\ket{E_{n}}=(n+1)a^{\dagger}\ket{E_{n}}.
$$
The reader will recognise the above as another eigenvalue equation, which states that for the counting operator, we have 
$$
N\ket{E_{n}}=n\ket{E_{n}}
$$
as one eigenvalue equation, but also the ket $a^{\dagger}\ket{E_{n}}$ as an eigenket which has an eigenvalue $n+1$ such that as above, the equation
$$
N(a^{\dagger}\ket{E_{n}})=(n+1)(a^{\dagger}\ket{E_{n}})
$$
applies. We have mentioned that $n$ is a nonnegative integer, so what this is really saying is that if we apply the counting operator $\ket{E_{n}}$, we get the \apos{measurement} of an eigenvalue $n$, but if we then change the eigenket by applying the operator $a^{\dagger}$ to it first, and then apply the counting operator, we will now count the eigenvalue $n+1$. What is the physical meaning of $n$? Well, we have 
$$
E_{n}=\left(n+\frac{1}{2}\right)\hbar\omega.
$$
The meaning of $\hbar\omega$ is interesting: one unit we can choose for the Planck's constant $\hbar$ is joules per hertz, and the angular frequency, as suggested by the name, can be measured in hertz, which means that units of $\hbar\omega$ are measured in joules- thereby a measure of energy. Of course, the value of $\hbar\omega$ is not $1$ Joule, since the angular frequency would have to be unthinkable for this to be true, but we can think of $\hbar\omega$ as small \apos{packets} of energy (in a loose intuitive manner) which we can use to measure the energy of the system it turns out very appropriately on the microscopic scale. The basic analogy one has already seen in their own studies? Coulombs as a measure of charge, which are also technically \apos{packets} of electrons (again in a very loose manner) which help make the counting process easier!
\\\\
Thus every time we increase $n$ we increase the energy reading. When $n$ is measured that corresponds to when the counting operator acts on the energy eigenket $\ket{E_{n}}$, and the corresponding energy is $n+1/2$ in units $\hbar\omega$. However, if we apply $a^{\dagger}$ to $\ket{E_{n}}$ first, we replace $n$ with $n+1$, thereby replacing the corresponding energy $n+1/2$ with the energy $n+1+1/2$. Thus the operator $a^{\dagger}$ \textbf{creates} one unit of energy, measured in $\hbar\omega$- giving it is name. Similarly:
$$
[N,a]=[a^{\dagger}a,a]=a^{\dagger}[a,a]+[a^{\dagger},a]a=a^{\dagger}(-1)+0=-a.
$$
Now we also have
$$
[N,a]=Na-aN\implies Na=aN+[N,a]=aN-a.
$$
Thus applying the operator to an energy eigenket, now indexed by $n$,
$$
Na\ket{E_{n}}=(aN-a)\ket{E_{n}}=aN\ket{E_{n}}-a\ket{E_{n}}=(n-1)a\ket{E_{n}}.
$$
Which, by the same argument as above, shows that the operator $a$ acting on a energy eigenstate $E_{n}$ \textbf{annihilates} (dramatic, but conventional) one unit of energy $\hbar\omega$! Thus the names the creation operator for $a^{\dagger}$ and the annihilation operator for $a$.
\\\\
Next, we wonder if there is an easier way to label the kets $a^{\dagger}\ket{E_{n}}$ and $a\ket{E_{n}}$, which we have also noted are eigenkets of the counting operator. Well, of course there is-
$$
N\ket{E_{n+1}}=(n+1)\ket{E_{n+1}}, \btab Na^{\dagger}\ket{E_{n}}=(n+1)a^{\dagger}\ket{E_{n}}\implies a^{\dagger}\ket{E_{n}}\equiv \ket{E_{n+1}}.
$$
This is completely reasonable, because $a^{\dagger}$ on the state $\ket{E_{n}}$ raises the energy by $1$. However, by the formula
$$
E_{n}=\left(n+\frac{1}{2}\right)\hbar\omega,
$$
and the fact we have taken it as a given that $n\in\mathbb{Z}^{+}$, we must have $E_{n+1}=E_{n}+1\times\hbar\omega$, in which case the equivalence of the states $a^{\dagger}\ket{E_{n}}$ and $\ket{E_{n+1}}$ makes perfect sense. Similarly, of course, $a\ket{E_{n}}$ and $\ket{E_{n-1}}$ are equivalent states. Yet we know that equivalence is not algebraic equality: rather,
$$
a\ket{E_{n}}=c\ket{E_{n-1}}
$$
for some multiplicative constant $c$ is sufficient for them to be equivalent states. The norm of $a\ket{E_{n}}$ is 
$$
\nhoptrip{E_{n}}{a^{\dagger}\:a}{E_{n}}=\nhoptrip{E_{n}}{N}{E_{n}}
$$
by our correspondence 
$$
\Omega\ket{X}\duac\bra{X}\Omega^{\dagger}.
$$
This is then equal to 
$$
|c|^2\sip{E_{n-1}}=|c|^2
$$
given the definition of $c$ above and the assumption that $\ket{E_{n+1}}$ has already been normalised. So we have 
$$
\nhoptrip{E_{n}}{N}{E_{n}}=n\sip{E_{n}}=n=|c|^2\implies c=\sqrt{n}.
$$
Thus we have 
$$
a\ket{E_{n}}=\sqrt{n}\ket{E_{n-1}}.
$$
We can do the exact same thing for $a^{\dagger}$ to get 
$$
a^{\dagger}\ket{E_{n}}=\sqrt{n+1}\ket{E_{n+1}}. 
$$
By the semidefinite metric, the norm of any ket, including $a\ket{E_{n}}$ is positive. However, the norm of $a\ket{E_{n}}$ is $n$ as already shown. Thus, $n$ must be nonnegative. We next realise that all we have covered so far means that
$$
(a^{2})\ket{E_{n}}=a\sqrt{n}\ket{E_{n-1}}=\sqrt{n}a\ket{E_{n-1}}=\sqrt{n(n-1)}\:\ket{E_{n-2}}
$$
and so on- in other words, applying the annihilation operator means we should be able keep annihilating energy in units of $\hbar\omega$ until we reach a negative value of $n$, no matter how great the $n$ and therefore energy. The reason this is problematic, however, is that we have already proven that the counting operator eigenvalue $n$ corresponding to the energy and counting simultaneous eigenstate $\ket{E_{n}}$ must be nonnegative due to the semidefinite postulate. So we cannot expect a negative value of $n$ to appear. As we have 
$$
a\ket{E_{n}}=\sqrt{n}\ket{E_{n-1}}
$$
We conclude that if we reach a $E_{n}$ with $n<0$ the definition means we must have reached an eigenvalue $n<0$. The fact above with the sequence of square roots shown by repeatedly applying the annihilation operator shows that clearly we should be able to reach a state $E_{n}$ with $n<0$. This is unless we terminate at a value before we reach below $n=0$. If after annihilating a sufficient number of $\hbar\omega$ units of energy we had $n\in(0,1)$ then we would be able to apply the annihilation operator again to get a new $n'=n-1\in(-1,0)$ which is negative. If we had a value of $n>1$ which was not an integer then we conclude we must be able to keep applying the annihilation operator until we get a value in the region $(0,1)$, after which the same argument applies that we reach a negative $n$. Thus we must have the value $n=1$ occurring, after which reapplying the annihilation operator should give $n=0$. What if we now try to reapply the annihilation operator? Well then, we have
$$
a\ket{E_{n}}=\sqrt{n}\ket{E_{n-1}}\implies a\ket{E_{0}}=0\ket{E_{-1}}=0.
$$
In other words, all negative index $E_{n}$, and therefore negative $n$ disappear! This satisfies our condition of being able to apply the annihilation operator repeatedly but not ever having states where the counting operator eigenvalue $n$ is negative. Thus we have shown with this argument, which should be reread if one is not convinced, that all we have done thus far on the assumption that the eigenvalues $n$ are integers was done on the correct assumption: $n$ must be a nonnegative integer, and there is a terminating state $\ket{E_{0}}$ which has energy $E_{0}=(0+1/2)\hbar\omega=(1/2)\hbar\omega$.