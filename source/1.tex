\chapter{Exoskeleton}
\section{Preface}
The sudden, explosive rise of quantum physics is probably both one of the most significant events in human intellectual history. Born in the ashes of a Newtonian model believed for two centuries to have been so unquestionable it almost held its own divinity, and forged by a cohort of Physicists who lopsidedly dominate the historical halls of fame in both number and scope of achievement, quantum physics has never been uncontentious, never obvious and never immediately complete. Yet in just 50 years, it ascended to full prominence, and its ideas, practices and results took to centre stage. It was nothing short of a revolution.
\\\\
This revolution, speaking for a second from an academic point of view past just electron spins and fundamental particles, was radical for two main reasons. The first was that physics had to fully accept that validity was derived from result and result alone. No matter how offensive the theories of quantum mechanics were to human intuition, the truth, if it lay beyond human imagination, simply lay beyond human imagination. Common sense was discarded for practical correctness, and the priority was achieving the results corroborated empirically even when everything seemed wrong. This mindset is critical to any quantum mechanics student, especially when they are starting off and unable to grasp the nuances of quantum mechanics well enough to even fathom debating the philosophical side of things. Indeed, it is the approach I take in this book. When a theory can make predictions accurate to the nearest ten billionth, as seen in quantum electrodynamics, and is not invalidated in any obvious way, it must be at least relevant. Studying quantum mechanics is central to science, and it will be still for quite some time. That should and will suffice for us.
\\\\
Secondly, quantum mechanics was a revolution in the history of applied mathematics. When the earliest serious formulations of quantum mechanics came in the mid 1920's, what belied them was mathematics: and not only mathematics, but fringe, new and diverse mathematics being unified to a single purpose: to create a theory which could explain the unexplainable results Physicists were obtaining at the time. The genius of Dirac, Heisenberg, Born and other founders was that they could draw from a purely abstract discipline and stamp it onto the real, physical world. Mathematics running physics has always been a well-understood fact, but the level we are are working at will be wholly new to the reader. Even a surface incursion into quantum mechanics will start reaching towards the deepest vestiges of mathematics in all corners.
\\\\
This will all present a stiff challenge to the reader, if this is the first time they have handled any theory of this level of complexity. In this sea of randomly drawn abstractions and deeply anti-human results, confusion becomes a very serious threat to the learner. It is extraordinarily easy in quantum mechanics to read 400 pages of textbook and still have absolutely no clue what a state vector/ wavefunction -- the first central mathematical object in quantum mechanics -- really is. Quantum mechanics is a famous sinkhole for people trying to be more intelligent than they are capable of understanding, but even that cannot be chalked up solely to outsiders who have never studied the subject before. Assuming one can learn quantum mechanics as one might learn differential equations -- crunching problems and becoming an expert through practising the mathematics -- is naive. There are rules to learn, and concepts to understand; and therefore rules and concepts to fail to understand regardless of how well one might feel they follow the proofs.
\\\\
Thus the aim of this book is to avoid confusion at all costs, and to be pragmatic. Most, if not all, of quantum mechanics textbooks work on the basis that students reading those textbooks are propped up by a rigorous undergraduate trimester and experienced professors explaining the nuances in lectures and seminars and even practical experiments while working with the same textbook. Self-learning quantum mechanics from such books is dangerous, because they are far too brief on explanation in their bid to prioritise swathes of algebra and computations, assuming experienced professors will do the rest for their usual readers. It is by no means impossible, especially for those trained in undergraduate mathematics, but it is likely ill-advised for the rest of self-learners to jump straight into quantum mechanics via books like those. Conversely, to those for which all the mathematics here comes in the blink of an eye, or who have finished studying a quantum mechanics textbook or course already, this book will comparatively less useful, because it involves very detailed and explicit proofs and explanations which are optimised for those learning quantum mechanics completely from the book, rather then those looking for a reference textbook.
\\\\
In the end, this book is but a stepping stone. A stepping stone which I believe is critical for a new student, to set principles in order before entering the true deep end, but also a stepping stone which is written to be a stepping stone. This book won't give you everything, but, simply by painstakingly avoiding a whole lot of bad, I hope it can give a self-learning student new to quantum mechanics a whole lot of good.
\\\\
You of course, shall be the judge of that.
\\\\
Han-Sen Choong
\section{A Physical Model}
One of the most difficult things in starting to learn quantum mechanics is that it can be very difficult to tell between concrete rules (of reality), chosen assumptions and mathematical logic. It is my belief that a new student of quantum mechanics must be able to distinguish between these as early as possible, because otherwise there are dangers of confusions where one questions assumptions as if they are concrete logical consequences when in the very first place they cannot be subject to such inspection. Thus, I think, it is useful to understand quantum mechanics as a discipline before delving into its details so that there is at all times in our learning of quantum mechanics this awareness of why exactly its components are arranged as they are. So let us begin with a question. What is the difference between quantum mechanics and quantum physics?
\\\\
The question may have been in the mind of the reader with the loose treatment of both terms in the Preface, and indeed if they have heard both terms elsewhere without knowing their distinction. In fact, the term \sapos{quantum physics} is will soon become extinct in the rest of this book, because the book is a quantum mechanics textbook-- not a quantum physics textbook. The difference is uncontentious; every classification would put quantum mechanics as \textit{the rules} and quantum physics as \textit{the application}. The latter is more advanced and difficult, but it is also wholly predicated on the former. 
\\\\
Quantum mechanics, easiest thought of then as the pure mathematical component to the study of quantum phenomena (sans experimentation), is what we call a \textbf{model}. Specifically, quantum mechanics is a model of physical reality, and thus we might refer to it as a physical model. However, it is the rationale of modelling which is a key characteristic of quantum mechanics as an academic discipline. In modelling, we take a situation where we have some computation or prediction we want to make, and we take the tools we have available to us from mathematics to create a system which is to make such predictions without any additional predictions which conflict with our desired subject. We choose the rules of our model based on what makes sense -- one would never model the probability of a train arriving $x$ minutes late with an exponential, for example, because the probability of a train arriving 100 minutes late should not be exponentially greater than it arriving 10 minutes late. How perfect are our rules? That would depend on the purpose of the prediction we are trying to make. In the situation of a train's arrival time, knowing the train has a probability 0.498263 of arriving at a specific time is no more useful than knowing it has 50\% probability, rounded to the nearest percentage point. In quantum mechanics, on such microscopic scales, we would like to be as accurate as possible, but yet our greater priority is making sure there are no glaring discrepancies between predictions made by the quantum model and physical experimental results. This latter condition of being true to real world experimental results is clearly prerequisite for any physical model, and the reason why quantum mechanics needed to be created in the first place was because the previously widespread model of Newtonian classical mechanics was proven incompatible with the results of new and radical quantum experiments.
\\\\
Now we can understand our task in this book as engaging with the model of quantum mechanics. Such consists of working with and understanding the rules the original Physicists creating the quantum model set out. These rules are called the \textbf{postulates} of quantum mechanics. This word \sapos{postulate} is critical -- as it means that these rules are \textit{assumed}, just like any mathematical model would assume rules and be tested on how well the resultant model fit its requirements. Thus the postulates of quantum mechanics are fundamentally different to other \sapos{rules} the reader might have come across in earlier mathematical studies. Those rules -- like the Pythagoras theorem -- are hard and concrete, and arise \textit{logically} from the definitions of a right-angled triangle, and side length. These rules, the quantum mechanical postulates, are \textit{assumed} rules by which we choose to abide! The difference is of course critical. Pythagoras' Theorem is not an assumption: it drops out of a construction we hold to be true. These quantum mechanical rules, however, don't drop out of logic. They were chosen, by the founders of quantum mechanics, to predict reality, and they were kept mainly because they worked.
\\\\
Now we return to the ideas of the Preface. In all modelling, the rules we assume are simply validated by the results of the predictions of the model compared to the results obtained in that which we are modelling. If the model is accurate and performs its functions well, we can then trust its rules are at least reasonably sound, and that studying that model is useful and relevant. This is the story of quantum mechanics, which is indeed the most successful mathematical model ever created. The Postulates we will be learning, in this book about quantum mechanics, are what define this quantum model we will be working with.
\\\\
Now, it is useful to undergo an understanding of what physical modelling-- in many ways, essentially no different to theoretical physics-- really aims to do, because most students will not have learnt this explicitly before. Physical modelling (physics) has two aims-- problems to answer:
\begin{enumerate}
    \item The State Problem
    \item The Time-Evolution Problem
\end{enumerate}
If a physical model can fully answer both these problems, then it is considered complete.
\subsection*{The State Problem}
The state problem concerns our ability to describe any state at any given time. It is therefore a problem of description, encapsulation and extraction. 
\\\\
Description is the most obvious criterion in the solution of the state problem. It pertains to our ability to observe a state and be able to put it down into description-- usually, mathematical description. It is only after this that we can bring anything successful to practical experiment, as there is no point trying to find data on states if there is no standard system whereby states can be distinguished and results can be recorded down.
\\\\
Linked with that description component problem, then, is the extraction problem, which is another key component of the state problem. If we have some state we have observed, and we have recorded its information down as some solution to the problem of description, we have gone from an observation to an abstract representation. However, we also want to be able to take an abstract representation and be able to at least imagine or compute the characteristics of the physical state it represents. Consider a situation, for example, where we can isolate air molecules and work out their speed in a simulation, and we decide to colour any air molecule with velocity above 550m/s red in our simulation, all air molecules with measured velocity 525-550m/s yellow, and all below 525m/s blue. That would be an attempt to tackle the description component of the state problem. However, it would also clearly be incomplete as a model of reality. The reason for that is simple: given a physical state (specifically, an air molecule having a certain velocity), we can certainly describe it using our colouring system. However, given a certain colour, we cannot at all work out anything about its velocity, other than how its value relates in magnitude to the value of 550m/s. In this case, it is clear that we can describe states better than we can extract information from a description, and it is clear this loss in reversability of information is a problem. 
\\\\
That example, and the final component of the state problem in encapsulation of information, will become more important very soon. For now, it should be very clear that the state problem is quite relevant, not always straightforward (especially after the groundbreaking results of the Stern Gerlach experiment we are about to study) and deserving of a place as one of the two most important problems in Physics.
\subsection*{Time Evolution and Observables}
The definition of the time-evolution problem is luckily much easier. Given a particular state, we want to be able to calculate how that state will evolve in a certain period of time into a new state (subject perhaps to time-evolving external conditions). This is the physics problem all readers will be familiar with: whether it be through acceleration-time graphs or via other common time-based experiments. 
\\\\
Such time-evolution is usually governed by an equation. In quantum mechanics, this is, as we will discover in Chapter 5, the widely famous Schr\"{o}dinger Equation. Equally, we note that a time-evolution equation clearly creates a greater onus on the results of the state problem in our model to be mathematical objects, so that these can actually be run through the equations we have.

We here have built an understanding of the central aims our physical model needs to meet, and in framing this beforehad, such discussions should no longer seem to come out of nowhere when we reach them by Chapter 3.
\section{A Structure}
If the reader is a student who feels high--school mathematics is trivially within their grasp, as indeed most students who dare start learning quantum mechanics are, you may start reading this section immediately. If you are a high--school student and do not think you  have learnt every corner of the syllabus yet, do not worry-- move swiftly onto Chapter 1, where a list of prerequisite mathematics is found in reasonable detail. As described there, you should by no means deprive yourself of a chance to get an early foothold into this significant subject even if out of just interest, but you should also open up some simultaneous study of those areas listed using textbooks written to present those areas, rather than trusting 20 bullet-pointed pages of a quantum mechanics textbook to teach you those crucial foundational concepts.
\\\\
Now, this section will present an insight into how this book will set about accomplishing its task of providing a pragmatic entry into quantum mechanics which avoids chronic confusions, building off the Preface. 
\\\\
The first thing to make clear is that this is a mathematics textbook. Certainly, it is a lot more verbose than a traditional mathematics textbook, because I prioritise, as I have already stated, an understanding of the principles over the cleanliness of logical arguments. However, none of the mathematics here is watered-down, inaccurate or simplified. By the time we move to Chapter 3, this will become rapidly clear, and by the time we reach Chapter 6, the prerequisite knowledge of Chapter 1 will be all but a fond memory of the good and easy times. 
\\\\
The second thing to note is that there will be a notation switch in Chapter 6. This structure is rare in quantum mechanics books: most of the time, an author chooses one notation and sticks to it. The reason I have chosen to set up this book with a notation switch is because I believe the conventional quantum mechanical notation is too good to not be learnt, but at the same time too novel and therefore distracting to focus on right at the beginning. The reader should bear in mind that no area in this book is wrong, and the notation shift will only introduce new algebraic manipulations, rather than replacing or correcting ideas covered in preceding chapters. I simply prefer that the reader struggles a bit with some extra reformulation there than struggles with notation, mathematics and conceptual understanding all at the same time, and in any case the demand of that switch is as much aesthetic as anything else, which should be acceptable.
\\\\
The book can be split into two constituent parts, therefore: before and after Chapter 6. Chapter 2 starts by providing motivation for the developments of quantum mechanics with the exemplifying Stern Gerlach experiment, and will be critical in laying out the new physical results which had to be accounted for. Chapter 3, 4 and 5 (quantum states, quantum observables and time-evolution respectively) lay out the quantum mechanical model and its rules with detailed study of subsequent implications both physically and algebraically. 
\\\\
The second part of this book can be characterised by the remaining chapters, which treat the mathematics more seriously and vigorously than the first part of the book. It will take the model we built in Chapters 3, 4 and 5, and recodify it in a way which allows us to understand the algebraic manipulations accessible to us in a more lucid and detailed manner. Finally, after doing all these things, the reader will be ready to move onto physical problems for the first time, and with some practice, will meet the end of the content of this book in Chapters 8 and 9. The closing problem will be the solution of the hydrogen atom, and we will witness a number -- and a perfectly accurate number at that -- drop out of the quantum mechanical model for the energy levels of hydrogen, which, astonishingly, is corroborated by empirical evidence. That should provide a satisfying conclusion to the journey of starting off with quantum mechanics, before the final chapter, Ad Infinitum, has been included as a conclusion to the book which I believe best sets a student on a path moving onwards from this text to learn more advanced quantum mechanics.  Hopefully the reader will be in very robust shape after all is said and done to press onto more advanced textbooks when their mathematics allows it, and then I will have successfully completed my task in writing this pragmatic exposition.

\tableofcontents
