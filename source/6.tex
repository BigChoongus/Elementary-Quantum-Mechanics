\chapter{Dirac Notation and Matrix Mechanics}
The second part of this book, starting now, will be far more mathematically demanding. It seeks to complete the original aim of this pragmatic book: to allow a reader to segue onto mathematically advanced quantum mechanics texts without chronic confusions of what the mathematics \textit{means}. Particularly, we will now tackle the challenging case of extending our quantum mechanics to cover continuously distributed observables.
\\\\
We have covered now all of the quantum mechanical postulates we need to cover. On the other hand, our problem solving skills, which are very different in quantum mechanics to just standard algebra, are still very limited in the wider scope of things. There are many manipulative techniques we have not covered which are fundamental, but far from trivial. Furthermore, learning these techniques will require a reformulation of almost everything we have done so far into a new notation. This switch will be strange and unwelcome at first, but sooner than later the reader will find that the new notation is far more useful than confusing, and at that point they will be on a very good path indeed.
\\\\
The benefit of this somewhat counterintuitive approach of learning one way and then reformulating into a new notation is not only that we don't have several factors affecting our understanding at once and causing trouble, but that the reader gets training to an undoubtedly undergraduate level without wrecking their ability to appreciate nuances as they grapple with difficult problems. Without a wider context of measurements and probabilities and state evolution within which to frame the mathematics in this book, it would seem like an endless barrage of symbols for which we have no use and no appreciation. On the other hand, after reading the previous chapters, we are capable of understanding why components are important-- they are probability amplitudes; how the state vector is both a label and can be used as a substitute for a physical state; how operators are central to the measurements we achieve for different operators. With all this conceptual learning done, in many ways our task from hereon is relatively simple-- change notation and crunch algebra without any other concerns any more! Remember what started things off. That $\uspin$ notation, which was designed to show how unimportant symbols themselves are, was where our discussion started, and we now come full circle to this idea that notation is far from sacred and must be evaluated simply based on pragmatic convenience. This new Dirac notation we are switching too will certainly make things more convenient, even though the state vector is still a state vector and the wavefunctions are still wavefunctions and the operators are still operators. Subsequently, looking in greater detail at the mathematical processes of quantum mechanics will become easier with our increased ability to engage in more efficient algebra. Nothing has changed. We are just switching from more intuitive symbols to the proper quantum mechanical convention of Dirac notation, which one might not have seen before but which needn't be mystified.
\\\\
In addition, it is simply necessary to cover the conventional notation, called Dirac notation for its founder, and the pertinent techniques which follow it, because there does not exist a textbook whose level is a step up from this one and which does not use such conventions and even in many cases assume it is prior mastered. We will also see that it would have been difficult to understand what it means outright before we understood what a state vector is, for example, but on the other hand now that we do understand what the state vector is, it will make problem solving more powerful without any danger of confusions. 
\\\\
In this chapter, we will set forwards some new limitations so we can focus on the algebraic content (given that we understand the concepts of the postulates quite well now). The primary limitation will be the one maintained in previous chapters: that we are working in discrete cases. The next chapter will be all dealing with continuity, which contains some extremely difficult mathematics resulting from the mathematics in this chapter, so this limitation will be soon loosened. We will also be avoiding any thoughts about the wavefunctions being the primary \sapos{tangible} representations of abstract state vectors, because we will see that with Dirac's notation comes another natural and powerful idea of matrix representations! We can also feel free to ignore measurements and related probabilities so that we can focus on developing algebraic techniques which will make problem solving more efficient and workable, because we already understand the postulates we need to therein. So for now, the job is mainly to forget the wavefunction, and focus on the more explicitly state vector (and matrix) focused approach we will start below. We appreciate throughout that wavefunctions (and therefore measurement probabilities) are still obtainable, by taking the inner product of a state vector with the relevant input basis vectors in that component function transformation we have seen, so its absence is far from its demise.
\section{Dirac Bra-Ket Notation}
Our new notation begins here. Like any new notation, our job will be changing the way we label the objects we want to consider. The starting point will be the state vector, and its notation. We will not change any properties of the state vector! Things may look different because the notation looks nothing like we have ever seen before, but that does not mean at all that the characteristics are different.
\\\\
The state vector will now be written as $\ket{\Psi}$. This called \textbf{ket} psi. We simply call it a ket instead of vector because Dirac notation is so ubiquitous in quantum mechanics that its authority over the discourse of quantum mechanics has turned the word ket into a  common noun in itself. The ket and the state vector are one and the same! The only difference is that we now use the notation $\ket{\Psi}$ instead of the notation $\Psi$. 
\\\\
The space of possible physical states will now be called the ket space instead of the state space. Obviously, since kets are the same as state vectors, kets are still in bijection with physical states. The vectors which span the ket space are also kets themselves. Therefore, an arbitrary ket can be expressed with components $\setof{c_{i}}$ as $$\ket{X}:=\sum_{i=1}^{n}c_{i}\ket{\alpha_{i}}$$ for a basis set $\setof{\ket{\alpha_{i}}}$. We can define scalar multiplication exactly as we have before:
$$
k\times\ket{X}=k\sum_{i=1}^{n}c_{i}\ket{\alpha_{i}}=\sum_{i=1}^{n}(kc_{i})\ket{\alpha_{i}}
$$
where the scalar multiplies each component of the vector to give it new components. Addition is defined by summing corresponding components as well:
$$
\ket{X}:=\sum_{i=1}^{n}c_{i}\ket{\alpha_{i}}, \ket{Y}:=\sum_{i=1}^{n}c'_{i}\ket{\alpha_{i}}\implies \ket{X}+\ket{Y}=\sum_{i=1}^{n}(c+c'_{i})\ket{\alpha_{i}}
$$
\\\\
An extremely important point, again, is that the ket is an abstract object. Recall that the state vector could not be given a form until we chose a specific basis. Since the ket is the same, we will here need to chose a basis for the ket to take a form. The key novelty of our new way of approaching the same ideas is that, in our bid to make our state vector, now ket, more \sapos{tangible}, we will this time try to divorce the ket from the idea of a wavefunction representation and replace this with yet another representation in bijection to the state vector-- a matrix representation! This is because ideas of matrices will be very useful to us in our work and manipulation. To do this, recall that in any basis a constituent of any vector space has unique components in that basis, most often referred to as its unique expansion in that basis. This allows us to create a unique matrix expression of any ket: for any
$$
\ket{X}:=\sum_{i=1}^{n}c_{i}\ket{\alpha_{i}}
$$
in the basis $\setof{\ket{\alpha_{i}}}$, we can express it as a column vector of its components
$$
\ket{X}=\begin{bmatrix}
c_{1}\\
c_{2}\\
\vdots\\
c_{n}\\
\end{bmatrix}
$$
and this is a unique way to express the ket in that basis. As all kets have the same dimensions in this matrix expression (since the dimension of the ket space is the same so the number of components specifying them are the same), we can form any linear combinations of them since we are just summing matrices of the same dimensions, which is acceptable, and this matrix expression will not be meaningless.
\\\\
Another important point is for the inner labelling of kets: i.e-
$$
\ket{\textbf{Inner Label}}.
$$
The inner label of a ket never has any mathematical meaning. We can label kets 
$$
\ket{1}, \ket{2}, ... \ket{5}
$$
if we want to. They are absolutely nothing to do with the numbers $1$ to $5$. Instead, the inner label serves to \textbf{organise} different kets: above, we have ordered ket $1$, ket $2$, ket $3$, and so on in some arbitrary way which is meaningful to us. We could also label them
$$
\ket{\text{Dirac}}, \ket{\text{Heisenberg}}, \ket{\text{Feynman}}
$$
except this provides no logical order for us and is long to write. Sometimes one will see some long-winded labels, such as 
$$
\ket{P=\sqrt{{2mE}/{\hbar}}\:}
$$
which is an example taken from Chapter 8, where we needed to label a ket by its momentum value $\sqrt{{2mE}/{\hbar}}$ to distinguish it from another ket with another momentum value (but otherwise identical characteristics). One should not be confused if there are numerical numbers as inner labels of kets: they are part of a taxonomy we have chosen ourselves. For example, in this book and ubiquitously in all books, convention is to denote eigenkets by the eigenvalue they represent. This is not to say that the label is physically meaningful-- we could easily label each eigenket by the square of its eigenvalue, for example- but it is useful to remember and use because it makes things clear for us when we have several lines of algebra simultaneously. Next, we need to complete the new notation for the inner product.
\subsection{Bra space and Inner Products}
One operation we cannot perform with a ket alone is the all-crucial inner product. To define this, we first define a new vector space. This vector space is the bra space, and is made by a simple transformation: for every ket $\ket{X}$, the corresponding bra is written $\bra{X}$ and is defined by
$$
\bra{X}=\ket{X}^{\dagger}.
$$
%#ADD TRANSPOSE CONJUGATE TO MATRIX PRELIM
The \apos{dagger} notation means the \textbf{Hermitian adjoint} of the ket $\ket{X}$. This is the formal way to refer to the transpose conjugate of any matrix (see the matrix preliminary) -- that is, complex conjugate all the matrix element values and then transpose the matrix. For the column vector kets, this therefore turns them to row vectors with the complex conjugate entries:
$$
\ket{X}=\begin{bmatrix}
c_{1}\\
c_{2}\\
\vdots\\
c_{n}\\
\end{bmatrix}\Rightarrow\stab\bra{X}=\begin{bmatrix}
c^{\ast}_{1}\stab c^{\ast}_{2} \dots c^{\ast}_{n}.
\end{bmatrix}
$$
One sees that the relationship exhibited between the two vector spaces of the ket space of kets and the bra space of bras is bijection. This is because every ket is unique as a column vector-- due to its unique expansion in any basis set-- and therefore taking the Hermitian adjoint will give us one single unique bra. For every element of the ket space there is therefore a single element in the bra space, and vice versa. This exists precisely because of the way we have defined the bra space.
\\\\
Why define such a bra space? Bras never represent physical states directly- even though they are in bijection with the state vectors (kets) which are in bijection with physical states so each bra is in bijection with physical states. The answer is simple-- because we need the inner product to be a real scalar given two input states: as we have already seen for the last few chapters. We cannot simply multiply two kets because we cannot multiply two column vectors by the rules of matrix multiplication! Therefore, we define the bra, which is a row vector with the complex conjugate entries, and now if we multiply them:
$$
\bra{X} \times \ket{X}
$$
we should get a $1\times n$ row vector multiplying an $n\times 1$ column vector, and therefore a $1\times 1$ result: eg, a scalar, as desired. 
\\\\
We should be convinced that this new inner product is exactly the same as the old inner product. Take any two state vectors
$$
\forall\stab\ket{\beta}:=\sum_{i=1}^{n}b_{i}\ket{\alpha_{i}}, \:\: \ket{\gamma}:=\sum_{i=1}^{n}c_{i}\ket{\alpha_{i}},
$$
By the rules of matrix multiplication, the  $i$'th column component of the row vector multiplies the $i$'th row component of the column vector. Thus we have
$$
\bra{\beta}\times\ket{\gamma}:=\ip{\beta}{\gamma}=\sum_{i=1}^{n}b^{\ast}_{i}c_{i}.
$$
as the components of the ket column vector, here $\ket{\beta}$ are replaced by their complex conjugates in the bra row vector. There is absolutely no difference to the inner product operation: only, we consider it from a new perspective due to the matrix representations we have introduced.
\\\\
Several properties we have established previously can now be written in the new notation:
\begin{itemize}
    \item $\ip{X}{Y}=\ip{Y}{X}^{\ast}$
    \item $\ip{X}{X}\geq0$
    \item A normalised ket $\ket{\Tilde{\alpha}}$ is such that $\ip{\Tilde{\alpha}}{{\Tilde{\alpha}}}=1$.
    \item Two kets $\ket{\alpha}$ and $\ket{\beta}$ are orthogonal if $\ip{\alpha}{\beta}=0$.
\end{itemize}
To further the bijection between the ket space and bra space, as bras will prove more useful than only to appear in inner products, we have, using the conventional $\leftrightarrow$ symbol to mean \apos{corresponds to}:
$$
\begin{aligned}
\ket{X}&\duac\bra{X}\\
c\ket{X}&\duac c^{\ast}\bra{X}\\
\ket{X}+\ket{Y}&\duac \bra{X}+\bra{Y}.\\
\end{aligned}
$$
The way we obtain this is an important algebraic point: by performing the operation \apos{dagger} on any linear combination of kets, we immediately create a new set of bras in bijection with linear combinations of kets of that type, in much of a similar argument to the one we used originally to show the bra space is in bijection to the ket space.
\\\\
The expansion of bras in their basis should be clear following the above facts. For 
$$
\ket{X}:=\sum_{i=1}^{n}c_{i}\ket{\alpha_{i}}
$$
we have 
$$
\bra{X}=\sum_{i=1}^{n}c_{i}^{\ast}\bra{\alpha_{i}}.
$$
Thus completes our investigation of the bra space, and now we can continue to reformulate old ideas in this new notation to familiarise ourselves with it.
\\\\
Further with inner label conventions: in general, for scalars we write
$\ket{aX}$ to mean $a\ket{X}$, and $\bra{aX}$ to mean $a^{\ast}\bra{X}$. It is up to the reader to distinguish which are the scalars and which the placeholder letters for kets and bras in question, though the context should never make this in any real doubt if one follows the steps through. Similarly, we want to ensure brevity with this notation as one (but not the only) of the benefits of employing bra ket notation, so we also often write $\ket{aX+bY}$ instead of $\ket{aX}+\ket{bY}$ or $a\ket{X}+b\ket{Y}$. Using the ket $\ket{aX+bY}:=\ket{aX}+\ket{bY}$ is somewhat easier than for example defining
$$
\ket{Z}:=\ket{aX}+\ket{bY}
$$ 
every time we sum two kets, where a new letter in the inner label would just be confusing.
\\\\
The inner product short-form facts are always helpful; indeed, we have \textit{linearity in ket}:
$$
\ip{V}{aX+bY}\equiv\bra{V}\times(\ket{aX}+\ket{bY})\equiv\ip{V}{aX}+\ip{V}{bY}\equiv a\ip{V}{X}+b\ip{V}{Y}.
$$
This fact is one we have seen already (fact S7 in Chapter 3) for inner products, but expressed in Dirac notation. The comparable idea exists in bra:
$$
\ip{aX+bY}{V}=(\bra{aX}+\bra{bY})\times\ket{X}=\ip{aX}{V}+\ip{bY}{V}=a^{\ast}\ip{X}{V}+b^{\ast}\ip{Y}{V}.
$$
\\
Next, we can also recall the ubiquitous expansion. Take an arbitrary ket $\ket{X}$ in the orthonormal basis $\setof{\ket{\tilde{\alpha}_{i}}}$ and the following will always hold: 
$$
\begin{aligned}
\ket{X}&:=\sum_{i=1}^{n}c_{i}\ket{\tilde{\alpha}_{i}} \implies \forall j, \stab \ip{\tilde{\alpha}_{j}}{X}=\bra{\tilde{\alpha}_{j}}\times\sum_{i=1}^{n}c_{i}\ket{\tilde{\alpha}_{i}}.
&
\end{aligned}
$$
By linearity in ket, the outside bra is absorbed into the sum notation so we get
$$
\ip{\tilde{\alpha}_{j}}{X}=\sum_{i=1}^{n}c_{i}\ip{\tilde{\alpha}_{j}}{\tilde{\alpha}_{i}}=\sum_{i=1}^{n}c_{i}\delta_{ij}
$$
by the orthonormality of the basis. Then, this is simply
$$
\ip{\tilde{\alpha}_{j}}{X}=c_{j},
$$
which should certainly be familiar now, though perhaps not before in Dirac notation. It tells us that, given a basis, if we orthonormalise it (with the Gram-Schmidt process), the expansion coefficients of any ket are not random: they are the inner products of the ket being expanded with the corresponding basis kets. So for any ket $\ket{X}$,
$$
\ket{X}=\sum_{i=1}^{n}\ket{\tilde{\alpha}_{i}}\ip{\tilde{\alpha}_{i}}{X}.
$$
It is important to represent it explicitly here because the above will be used universally for problems to follow, and without explicitly stating that this is the expansion we already know it could otherwise be confusing. The reader should find it simple to produce a bra expansion in an orthonormal bra basis, though this is seldom as useful or commonly used.
\subsection{Operators and Eigenkets}
The action of an operator $\Omega$ on a ket $\ket{X}$ is written $\Omega\ket{X}$. This is perhaps why we incorporate scalars into the ket inner label, as noted above-- so that there is no mixup between scalar multiplication and the action of an operator on a ket. The action of an operator on a bra is conversely written $\bra{X}\Omega$.
\\\\
We always focus on linear operators which map one ket onto another ket in the vector space. This makes sense: the application of observable operators for example we do not expect to make an impossible state, and any possible state is represented by some superposition of the state vectors which constitute the ket space. It is good to keep in mind  very explicitly the notion that an operator acting on a ket always produces a new ket. More properties of linear operators can be expressed in bra-ket notation:
\begin{itemize}
    \item $a\Omega\ket{X}=\Omega a\ket{X}$
    \item $\bra{X}a\Omega=\bra{X}\Omega a$
    \item $\Omega_{1}\ket{X}=\Omega_{2}\ket{X} \forall\ket{X}\implies\Omega_{1}=\Omega_{2}$
    \item $\Omega\{a\ket{X}+b\ket{Y}\}=a\Omega\ket{X}+b\Omega\ket{Y}$
    \item $\Omega\{a\bra{X}+b\bra{Y}\}=a\bra{X}\Omega+b\bra{Y}\Omega$
\end{itemize}
The product of two operators means to apply the \apos{closer} operator to the ket or bra respectively and then apply the second, \apos{further} operator to the resulting ket or bra. We already know from our study of compatibility in particular that the assumption that two given operators will commute is false. The commutator is denoted the same way:
$$
[\Omega, \Lambda]:=\Omega\Lambda-\Lambda\Omega
$$
and so is the anticommutator:
$$
\{\Omega, \Lambda\}:=\Omega\Lambda+\Lambda\Omega.
$$
The eigenvectors of the operators will now often be called eigenkets, for obvious reasons, and the name eigenbras follows for bras in bijection with eigenkets. 
\\\\
We must now introduce the unique way to specify an operator as a matrix. This will allow us to perform some important operations in the future. 
\\\\
We start by noting that the action of an operator on the basis vectors of a ket space (any basis, not just its own eigenbasis!) is sufficient knowledge to specify its actions on all kets in that basis:
$$
\Omega\ket{\alpha_{i}}=\ket{\alpha'_{i}}\imp\:\: \forall \stab\ket{X}:=\sum_{i=1}^{n}c_{i}\ket{\alpha_{i}}, \mtab \Omega\ket{X}=\Omega\sum_{i=1}^{n}c_{i}\ket{\alpha_{i}}
$$
and by the linearity of the operator, this is just 
$$
\Omega\ket{X}=\sum_{i=1}^{n}\Omega c_{i}\ket{\alpha_{i}}=\sum_{i=1}^{n} c_{i}\Omega\ket{\alpha_{i}}=\sum_{i=1}^{n}c_{i}\ket{\alpha'_{i}}.
$$
However, things are not over from the perspective of a matrix formulation. Not only does this specification not give any clue as to how to express the operator as a matrix, but the resulting ket after the transformation is made is also not in any way representable as a column vector. This is because we must remember that a linear transformation on the basis vectors of a space by no means produces another basis which still spans the space at all (in which case a vector of components makes no sense anymore). The best example of this, of course, would be if $\Omega=\Omega_{0}$, the null operator, in which case all $\ket{\alpha'_{i}}$ would be null kets and certainly span no space at all.
\\\\
So the above informs us that to complete the task we need to return all the above to the original basis, which we know is serviceable for matrix representations. We thus start by expressing the components of the transformed kets $\setof{\ket{\alpha'_{i}}}$ in the original basis. If
$$
\ket{\alpha'_{j}}:=\sum_{i=1}^{n}c^{(j)}_{i}\ket{\alpha_{i}}
$$
and we assume the starting basis $\setof{\ket{\alpha_{i}}}$ is orthonormal, since otherwise it could be orthonormalised, then we clearly know the components $c_{i}^{(j)}$ are simply the inner products $\ip{\alpha_{i}}{\alpha'_{j}}$. This component is the component of the $j$'th transformed ket corresponding to the $i$'th basis ket. We can then define the entities
$$
\Omega_{ij}:=\ip{\alpha_{i}}{\alpha'_{j}}=\bra{\alpha_{i}}{\Omega}\ket{\alpha_{j}}
$$
and these are the components of the transformed kets in the original basis before they were transformed. The original question can be reposed. If we define 
$$
\Omega\ket{X}:=\ket{x_{0}}:=\sum_{i=1}^{n}c'_{i}\ket{\alpha_{i}}
$$
for some components $c'_{i}$, then these components are
$$
c'_{i}=\ip{\alpha_{i}}{x_{0}}=\bra{\alpha_{i}}\Omega\ket{X}=\bra{\alpha_{i}}\Omega\biggl(\sum_{i=1}^{n}c_{i}\ket{\alpha_{i}}\biggr).
$$
By the linearity of the operator, this becomes
$$
c'_{i}=\bra{\alpha_{i}}\times\biggl(\sum_{j=1}^{n}c_{j}\Omega\ket{\alpha_{j}}\biggr),
$$
and then by linearity in ket this becomes 
$$
c'_{i}=\sum_{j=1}^{n}c_{j}\bra{\alpha_{i}}{\Omega}\ket{\alpha_{j}}.
$$
Thus using the same notations defined above this is simply
$$
c'_{i}=\sum_{i=1}^{n}c_{i}\Omega_{ij}.
$$
The notation $\Omega_{ij}$ is clearly meant to hint that these can be placed in some matrix where each value $\Omega_{ij}$ (note these are inner products, so they are indeed a scalar values) is the entry in the $i$'th row and $j$'th column of the matrix. And as each $\Omega_{ij}$ is the component of the $j$'th transformed ket corresponding to the $i$'th basis ket, we say that the upper limit of $i$ and $j$ are both $n$ since there are $n$ original basis kets, and therefore $n$ transformed kets as well. Thus we can create an $n\times n$ matrix for all the entries $\Omega_{ij}$. The relationship we get is that 
$$
\begin{bmatrix} 
c'_{1} \\ 
c'_{2} \\
\vdots \\ 
c'_{n} \\ 
\end{bmatrix}
=
\begin{bmatrix}
\langle{\alpha_{1}}|\Omega|{\alpha}_{1}\rangle & \langle{\alpha_{1}}|\Omega|{\alpha}_{2}\rangle &
\dots & \langle{\alpha_{1}}|\Omega|{\alpha}_{n}\rangle\\ 
\langle{\alpha_{2}}|\Omega|{\alpha}_{1}\rangle & \ddots &
\:\dots\: & \vdots \\
\vdots & \vdots & \ddots & \vdots\\ 
\langle{\alpha_{n}}|\Omega|{\alpha}_{1}\rangle & \dots & \dots & \langle{\alpha_{n}}|\Omega|{\alpha}_{n}\rangle
\end{bmatrix}
\begin{bmatrix} 
c_{1} \\ 
c_{2} \\
\vdots \\ 
c_{n} \\
\end{bmatrix}
$$
as a way to relate the components of the transformed vector $\Omega\ket{X}:=\ket{x_{0}}$ to the original components of $\ket{X}$ before it was transformed. We see that the left hand side is the matrix representation of $\ket{x_{0}}$ in the basis we have been using, since it specifies the components of $\ket{x_{0}}$ as a column vector. Meanwhile, the right column vector clearly does the same for the original $\ket{X}$. And therefore this whole matrix equation is clearly the matrix form of the definition $\ket{x_{0}}=\Omega\ket{X}$, which then means that the $n\times n$ matrix in the middle is the matrix representation of $\Omega$. So to conclude this discussion we restate the fact that 
$$
\begin{bmatrix}
\langle{\alpha_{1}}|\Omega|{\alpha}_{1}\rangle & \langle{\alpha_{1}}|\Omega|{\alpha}_{2}\rangle &
\dots & \langle{\alpha_{1}}|\Omega|{\alpha}_{n}\rangle\\ 
\langle{\alpha_{2}}|\Omega|{\alpha}_{1}\rangle & \ddots &
\:\dots\: & \vdots \\
\vdots & \vdots & \ddots & \vdots\\ 
\langle{\alpha_{n}}|\Omega|{\alpha}_{1}\rangle & \dots & \dots & \langle{\alpha_{n}}|\Omega|{\alpha}_{n}\rangle
\end{bmatrix}
$$
is the way to represent the operator $\Omega$ in the $n$-dimensional ket space spanned by the orthonormal basis vectors $\setof{\ket{\alpha_{i}}}$.
\section{A Rudimentary Manipulation Toolbox}
The introduction of Dirac notation is surprisingly powerful because its unorthodox aesthetic form allows for the formulation of some identities and operators which under normal circumstances would look like a long amalgamation of random algebraic entities, but look simple and orderly (and therefore easy to memorise and use) in Dirac notation. This section will introduce essential identities and axioms which will be used without second thought for advanced problem solving, and in doing so, I hope the reader can also continue the process of becoming fluent in Bra Ket.
\subsection{General Solution of the Eigenvalue Problem}
The first tool we will see is a general solution of the eigenvalue problem, for which we have been waiting. The eigenvalue condition is always
$$
\Omega\ket{\omega} = \lambda\ket{\omega}.
$$
for some eigenvalue $\lambda$ and eigenvector $\ket{\omega}$
We know that all operators in the space can be represented by an $n\times n$ matrix (where $n$ is the dimensionality of the space). Just to convert the eigenvalue $\lambda$ into matrix form as well, we will multiply both sides by the $n\times n$ identity operator $I$:
$$
\begin{aligned}
\Omega\ket{\omega} &= \lambda I\ket{\omega}\\
\Rightarrow\:\: (\Omega-\lambda I)\ket{\omega} &= 0
\end{aligned}
$$
where 0 is the null matrix. Considering the whole in matrix form, we write:
$$
\begin{bmatrix}
\Omega_{1,1}-\lambda & \Omega_{1,2} & \dots  & \Omega_{1,n} \\ 
\vdots  &   \Omega_{2,2}-\lambda & \vdots & \vdots \\ 
\vdots &  \dots & \ddots & \vdots \\
\Omega_{n,1} & \dots & \dots & \Omega_{n,n}-\lambda\\ 
\end{bmatrix}
\begin{bmatrix}
c_{1} \\ 
c_{2} \\ 
\vdots \\ 
c_{n} \\ 
\end{bmatrix}
= 
\begin{bmatrix}
0 \\
0 \\
\vdots \\ 
0 \\
\end{bmatrix}.
$$
Looking at the matrix representation of the equation above, we can deduce that
$$
\forall i, \:\: \sum_{j}c_{j}(\Omega_{ij}-\lambda\delta_{ij})=0
$$
%# CRAMER'S RULES OF MATRIX DETERMINANTS
by the rules of matrix multiplication. This is a linear system of equations, with coefficients $\Omega_{ij}-\lambda\delta_{ij}$ and unknowns $c_{i}$. Therefore the logic behind Cramer's Rules apply: and since the $c_{i}$ are the components of $\ket{\omega}$ in the basis they are not all zero since $\ket{\omega} \neq 0$, and thus we need nontrivial solutions and therefore the determinant of the leftmost term must be zero. In other words,
$$
\det(\Omega-\lambda I) = 0 \iff
\forall i, \:\: \sum_{j}c_{j}(\Omega_{ij}-\lambda\delta_{ij})=0
$$
This is practically helpful, especially in problems with fewer dimensional spaces, and as a theoretical concept which shows us we can always solve the eigenvalue problem. There will be other methods to solve eigenvalue equations too, but these often rely somewhat more on inspection and then proof that one has all the solutions by inspection; these inspection methods only arise as shortcuts based on specific conditions we get.
\subsection{The Associative Axiom}
Dirac introduced in his paper on Dirac notation a critical axiom called the Associative Axiom. This axiom will save a huge amount of confusion for those unfamiliar with Bra-Ket notation, and is possibly the most powerful idea of them all. This axiom states that all (legal) \apos{multiplication} operations in Dirac notation are always associative.
\\\\
Put in scalar terms, this word associative is usually seen as
$$
(a\times b)\times c = a \times (b\times c).
$$
In other words, so long as the order of the terms $a,b,c$ are kept (here, $a$ before $b$ before $c$), the two multiplication operations can be performed in either order. With numbers this is completely natural (as also is additive associativity, but not subtractive or divisive associativity), but with Dirac notation this is much less trivial and also much more powerful.
\\\\
The first example we have seen but not explained explicitly is the inner product 
$$
\bra{X}{\Omega}\ket{Y}.
$$
This is an inner product since it is the product of a bra, $\bra{X}$ with a ket $\Omega\ket{Y}$ (which is another ket in the ket space the operator $\Omega$ has mapped the original ket $\ket{Y}$ to). However, it can also be seen as a bra, $\bra{X}\Omega$, multiplied by a ket $\ket{Y}$. This is the associative axiom: in fact it does not matter which way the multiplication goes and only because of this can we write the concise $\bra{X}{\Omega}\ket{Y}$.
\\\\
The ramifications of this fact are profound, as beginners with Dirac notation often get confused about which products are \apos{bunched together} and must be performed first, leading to a great state of bewilderment when a term might have several bras, operators and kets next to each other. The answer is very simply that, so long as operations are being performed left to right, the order does not matter and the reader may read the multiplications in any way they like. It also immediately leads to theorems for entities in Dirac notation, by defining them through the perspective of other known products. The first example of this we will see is the following completeness relation.
\\\\
\textbf{\underline{The Outer Product}}
\\\\
We discussed above how the associative axiom holds for all legal multiplications between bras, kets and operators. What constitutes an illegal multiplication? Well, to start we can give the example of the orthonormal expansion of any ket $\ket{X}$,
$$
\ket{X}=\sum_{i=1}^{n}\ket{\talpha_{i}}\ip{\talpha_{i}}{X}.
$$
Contrast this to the naive expansion
$$
\ket{X}=\sum_{i=1}^{n}c_{i}\ket{\talpha_{i}}
$$
which we might use if we were not sure the basis was orthonormal, for example. We see that the position of the component shifted from the leftmost sum term to the rightmost sum term when $c_{i}$ was replaced with $\ip{\talpha_{i}}{X}$. One might ask why this is. The best answer for this is simply the associative axiom: if we had written 
$$
\ket{X}=\sum_{i=1}^{n}\ip{\talpha_{i}}{X}\ket{\talpha_{i}},
$$
then we might think this is okay as we have a scalar $\ip{\talpha_{i}}{X}$ multiplying a ket $\ket{\talpha_{i}}$. However, by the associative axiom we also expect the above to be equally well expressed as 
$$
\ket{X}=\sum_{i=1}^{n} \bra{\talpha_{i}}\times\biggl(\ket{X}\times\ket{\talpha_{i}}\biggr).
$$ 
This is where our problem is-- two kets cannot be multiplied together, as they are both $n\times 1$ matrices! Similarly, two bras cannot be multiplied together as they are both $1\times n$ matrices. So this is why we cannot write
$$
\ket{X}=\sum_{i=1}^{n}\ip{\talpha_{i}}{X}\ket{\talpha_{i}}.
$$
This however implies that for the expansion
$$
\ket{X}=\sum_{i=1}^{n}\ket{\talpha_{i}}\ip{\talpha_{i}}{X}
$$
it is correct and therefore not only is the operation $\ket{\talpha_{i}}\times\ip{\talpha_{i}}{X}$ legal (it is a scalar multiplying a ket, so it should be)- but also that the operation
$$
\left(\op{\talpha_{i}}{\talpha_{i}}\right)\times \ket{X}
$$
should be possible. This product $\op{\talpha_{i}}{\talpha_{i}}$ is called the outer product between the bra $\bra{\talpha_{i}}$ and ket $\ket{\talpha_{i}}$. We can verify it should be possible, as it is an $n\times 1$ matrix multiplied by a $1\times n$ matrix, which should give an $n\times n$ matrix. As it produces an $n\times n$ matrix it clearly does not result in a scalar like the inner product; rather, we might posit that it is an operator, as operators are represented  by $n\times n$ matrices. This is in fact a true assumption: an outer product is fundamentally meant to be treated as an operator. However, before we discuss that we should establish and prove a fundamental theorem in quantum mechanics which will be surprisingly useful for manipulation. This is the completeness relation, which results from the associative axiom, and deserves its own section for its importance even though the proof requires no considerable thought if we have the associative axiom in mind.
\\\\
\textbf{\underline{The Completeness Relation}}
\\\\
We have seen that the representation
$$
\ket{X}=\northexp{X}{\talpha_{i}}
$$
is allowed because the sum term $\sop{\talpha_{i}}\times \ket{X}$ is just the action of the outer product on the ket ${X}$. However, by the rule of linear operators that 
$$
(\Omega_{1}+\Omega_{2})\ket{X}=\Omega_{1}\ket{X}+\Omega_{2}\ket{X}
$$
and the fact that the outer products are operators, we can use the same principle to write 
$$
\ket{X}=\biggl(\sum_{i=1}^{n}\sop{\talpha_{i}}\biggr)\ket{X}.
$$
And this is very revealing, of course, because this is true for any arbitrary ket $\ket{X}$, and therefore we come to the conclusion that 
$$
\sum_{i=1}^{n}\sop{\talpha_{i}}=1.
$$
This is the most commonly written form of the completeness relation, but one should be aware that the 1 here represents the identity operator, rather than a scalar-- since the sum of $n\times n$ matrices will give us an $n\times n$ matrix rather than a scalar.
\\\\
On the other hand, the more we foray into problem solving the more we will see of this seemingly completely useless relation. One example can be in the proof that the sum of the modulus squared components of a normalised ket is also equal to $1$:
$$
\sip{X}=\bra{X}\times\ket{X}=\bra{X}\times (1\ket{X})=\bra{X}\times \sum_{i=1}^{n}\sop{\talpha_{i}}\times\ket{X}
$$
where the number $1$ is the identity operator again and the kets $\setof{\talpha_{i}}$ are the orthonormal basis vectors. Then, by linearity in ket, we can write this as 
$$
\sip{X}=\sum_{i=1}^{n}\bra{X}\times\sop{\talpha_{i}}\times\ket{X}
$$
and by the associative axiom this is just
$$
\sip{X}=\sum_{i=1}^{n}\ip{X}{\talpha_{i}}\ip{\talpha_{i}}{X}=\sum_{i=1}^{n}\ip{\talpha_{i}}{X}\times(\ip{\talpha_{i}}{X})^{\ast}=\sum_{i=1}^{n}|\ip{\talpha_{i}}{X}|^{2}.
$$
However, as the inner products $\ip{\talpha_{i}}{X}$ are the components of $\ket{X}$ in the basis $\setof{\talpha_{i}}$,  if the ket $X$ is normalised then the left side is $1$ so we get 
$$
1=\sum_{i=1}^{n}|c_{i}|^2.
$$
So the modulus squared of the components of a normalised ket in a basis sum to $1$. We know these modulus squared components are probabilities if we are working in an observable space by the measurement postulate! So what we have just proved, in a flick of the pen when working with Dirac notation, is that the reason we like working with discrete wavefunctions is because their components in this way perfectly represent probabilities summing in total to $1$! We have already proved this fact, at the end of Chapter 4 on normalised discrete wavefunctions being interpreted as probability mass functions, but, if one takes a quite glance at that, the proof is neither so short, nor so elegant: it is positively clunky in comparision. The purpose of showing this is to exhibit that, just like in normal algebra, the world of Dirac notation can belie some extremely unusual and ingenious manipulations which without the notation would not be so easy to express, but with the fluency in the notations, can be done in just a few lines or seen in a matter of seconds. 
\section{Change of Basis}
One of the most important concepts in matrix mechanics is that of diagonal matrices. A square matrix is diagonal if all its elements are $0$ except those in the major diagonal: that is, elements $M_{ii}$, which can be anything. One example of a diagonal matrix is the identity matrix, which has all zero entries except the major diagonal, which is filled with $1$'s.
\\\\
\textbf{{Theorem: Every Hermitian operator has at least one orthonormal}} \\
%#INCOMPLETE
\textbf{{eigenbasis in which its matrix representation is diagonal. The}}\\ 
\textbf{{diagonal entries are the eigenvalues of the operator.}}
\\\\
This proof is quite tough and not particularly illustrative for our current needs, and therefore will not be shown for now. However, we will find that the implications of this theorem are hugely powerful, because it means that if we can diagonalise the operator matrix then we immediately solve the eigenvalue problem as the diagonal matrix values give us the eigenvalues from which deriving the eigenvectors should be easy: clearly, a great problem-solving step.
\\\\
So we want to diagonalise the operator. By the theorem, this means we want to take one matrix in any basis to the eigenbasis which diagonalises the operator. Thus we will often be  interested in a \textbf{change of basis}.
\\\\
A change of basis is performed by applying an operator to the original basis kets and mapping them onto the new kets. Such is a common idea in quantum mechanics: to transform one ket to another we also try to see if there is a way we can formulate it as an operator equation. Here, we certainly can; such operators are usually denoted $U$. So we guess that for some original basis $\setof{\ket{\alpha_{i}}}$ and operator $\Omega$ with eigenvalues $\setof{\beta_{i}}$ and eigenvectors $\setof{\ket{\beta_{i}}}$,
$$
\forall i, \stab U\ket{\alpha_{i}}=\ket{\beta_{i}}.
$$
Assuming without loss of generality that these basis sets are both orthonormal, the operator which works is
$$
U:=\sum_{j}\op{\beta_{j}}{\alpha_{j}}.
$$
We can prove this. For any original basis ket $\ket{\alpha_{i}}$,
$$
\biggl(\sum_{j}\op{\beta_{j}}{\alpha_{j}}\biggr)\ket{\alpha_{i}}=\sum_{j}\ket{\beta_{j}}\ip{\alpha_{j}}{\alpha_{i}}=\sum_{j}\ket{\beta_{j}}\delta_{ij}=\ket{\beta_{i}}
$$
and this holds for any $i$ since we have orthonormal sets. What is much more important, however, is that the operator $U$ satisfies a very interesting condition. Consider, for the operator defined, the product
$$
U^\dagger U=\biggl(\sum_{k}\op{\alpha_{k}}{\beta_{k}}\biggr)\times\biggl(\sum_{j}\op{\beta_{j}}{\alpha_{j}}\biggr)
$$
As the basis sets are orthonormal, we see that all the terms disappear on account of the inner product $\ip{\beta_{k}}{\beta_{j}}$, except when $k=j$. In those cases, we get 
$$
\sum_{k}\sop{\alpha_{k}}=1.
$$
So for this \apos{transformation operator}, we have $U^{\dagger}U=1$. Similarly, $U U^{\dagger}=1$. Such an operator is called \textbf{unitary}. Now consider some arbitrary operator which is unitary and its action on a state ket which is normalised. We know that for the ket $\ket{\Psi
}$
\section{Exercises from Chapter 6$\ast$}
\begin{enumerate}
    \item 
    \item
    \item
    \item
    \item
    \item
    \item
    \item
    \item
    \item
\end{enumerate}